
\documentclass{book}
\usepackage{a4,makeidx,fancyheadings}
\usepackage{graphicx}
\usepackage[francais]{babel}
\usepackage[utf8]{inputenc}
\usepackage{fancyvrb}
\usepackage{makeidx}
\usepackage{tabularx}
\usepackage{eurosym}
\usepackage{color}
\usepackage{url}
\usepackage{pdfpages}
\usepackage{xspace}
\usepackage{geometry}
\geometry{ 
      hmargin=3cm,
      vmargin=3cm
}

\makeindex
	
\begin{document}
\newcommand{\motcle}[1]{\index{#1}\emph{#1}}
\newcommand{\sre}{service relations entreprises\xspace}
\newcommand{\sop}{service organisation et process\xspace}
\newcommand{\scom}{service communication\xspace}
\newcommand{\sconc}{service concours\xspace}


\thispagestyle{empty}
%-------- fancy headings setting -----------
\rhead[]{}
%\footrulewidth 0.1pt
\rfoot[]{\small\sc S. Genaud}
\pagestyle{fancy}
%------------------------- Page de Garde -------------------
\setlength{\parindent}{0mm}
\setlength{\parskip}{0mm}
\vspace*{\stretch{1}}
\rule{\linewidth}{1mm}
\begin{center}
\Large{Audit du Système d'Information}\\[5mm]
\Large{de l'Ecole de Management de Strasbourg}\\[5mm]
\large{Stéphane Genaud}
\rule{\linewidth}{1mm}
\vspace*{\stretch{2}}
\end{center}
\begin{center}
juin \oldstylenums{2011} \\
\textrm{
$Revision$\\
$Id$\\
$Date$\\
}
\end{center}
%------------------------------------------------------------

\tableofcontents
\newpage

 
 

\chapter{Audit des Services}
 

\section{Responsable Informatique}

\paragraph{Christos Karakostas} (Responsable)


\section{Recherche}
\paragraph{Karine Bouvier (KB), Maxime Merli (MM), Thierry Nobre (TN)}

\subsection{Processus fonctionnels}

Le besoin quasi-exclusif est le recensement des publications des chercheurs.
Cette information sert à deux types processus:
\begin{itemize}
\item[$\bullet$] RC-1) L'évaluation de la production scientifique individuelle ou 
collective (AERES, accréditations, classements, etc).
\item[$\bullet$] RC-2) La communication sur les centres d'intérêts et compétences 
vers l'extérieur (sites web, faculty book, etc).
\end{itemize}

\bigskip
L'information doit être structurée de façon à répondre aux exigences des 
différentes instances d'évaluations (\motcle{AERES}, organismes d'accréditation).
Cette structuration des informations bibliographique est classique et des
formats existent depuis longtemps (par exemple bibtex). Une spécificité
à prendre en compte est les classements des revues établi annuellement par 
l'AERES.

D'autres processus peuvent être identifiés, comme par exemple l'organisation
d'une conférence, où la collecte d'information concernant le devenir des
doctorants. Cependant, l'essentiel de ces actions sont ponctuelles et ne 
nécessitent pas un support récurrent du système d'information. La seule 
activité récurrente annexe identifiée est la consultation de l'état des 
finances. Dans ce cas, l'interface web avec \motcle{SIFAC} ainsi que les 
contacts nécessaires avec le gestionnaire des comptes apparaissent satisfaisants.



\subsection{Existant}

L'existant est essentiellement constitué de:
\begin{itemize}
\item la base de données recherche (MS Access), propre à l'EM,
\item l'application, partie de l'\motcle{intranet}, qui gère cette base : saisie, 
modification, consultation de la base.
\end{itemize}
Un premier projet a été mené en 2010 pour constuire cette base de données des
publications des enseignants-chercheurs de l'EM Strasbourg. Il a été demandé à 
CK de développer une fonctionnalité de l'intranet pour cela. L'interface de 
saisie développée n'étant pas assez contrainte, ce premier projet n'a pas 
permis de collecter des données d'une qualité suffisante pour être exploitables
pour produire les statistiques requises (e.g noms de revues orthographiées 
différement). Sur la base de ce constat, l'outil de collecte a été amélioré en 
2011. Du point de vue des responsables du projet (MM, KB, TN) l'outil donne 
satisfaction en ce qui concerne la qualité des données. Des fonctionnalités 
supplémentaires sont envisagées de manière non urgente (fichier PDF téléchargeable 
de la publication, remontée automatique du papier sur des archives ouvertes comme SSRN%
\footnote{\url{ http://www.ssrn.com/}}).
La façon dont les informations sont présentées sur les pages web personnelles
des enseignants-chercheurs%
\footnote{\url{http://www.em-strasbourg.eu/enseignants/em-strasbourg-enseignants}}
est également jugé satisfaisant.


\subsection{Analyse}

Il ressort de la discussion que les principaux problèmes auxquels ont eu à 
faire MM et KB pour la mise en place du projet sont:
\begin{itemize}
\item un problème d'interlocuteur : ils n'ont pas trouvé de personne référente 
capable de les orienter sur les ressources internes ou externes, capables de fournir 
la prestation dont ils avaient besoin.
\item un problème de conduite de projet : une fois la ressource chargée de la 
réalisation identifiée (CK), leur demande n'a pas fait l'objet d'une conduite
de projet. La planification était hasardeuse, le cahier des charges et
les validations peu formalisées.
\end{itemize}

\paragraph{Qualité des applications}
Il est à noter que l'interface construite ne permet pas de faire d'import (bibtex, 
endnotes,~...). 

\paragraph{Bilan qualité des données}
~\\


	\begin{tabular}{|l|l|l|l|l|l|l|}
	\hline
	efficacité	& efficience &	confidentialité	& intégrité & disponibilité & conformité & fiabilité \\
	\hline
	
	\hline
	\end{tabular}
	

%--------------- Relations Entreprises -----------------------------------

\section{Relations Entreprises et EM Strasbourg Partenaires}

\paragraph{Marie-Hélène Brémont (MHB), Hélène Heintz (HH), Francis Schillio (FS)}
~\\

\textit{Le service Relations Entreprise et Francis Schillio pour 
EM Strasbourg Partenaires ont été audités séparémment mais leurs 
besoins sont confondus. La description est donc regroupée en une 
seule section.}


\subsection{Processus fonctionnels}

Les processus principaux sont:
\begin{itemize}
\item[$\bullet$] RE-1) La recherche ponctuelle d'information sur une entreprise 
			     ou sur un contact professionnel particuliers.
\item[$\bullet$] RE-2) Le filtrage d'un ensemble d'entreprises ou de contacts
			     selon des critères de façon à obtenir un tableau de bord
			     ou envoyer une sollicitation. 
\item[$\bullet$] RE-3) L'enrichissement de la base de données entreprises.
\end{itemize}

\bigskip

L'activité RE-1) permet à FS ou MHB de trouver les coordonnées d'une personne
contact dans une entreprise, d'avoir une vision d'ensemble des personnes 
présentes dans l'entreprise, de voir l'état des cotisations, des versements
effectués, de vérifier à quelles chaires l'entreprise participe, ou toute
information permettant d'avoir une vision globale de l'historique de la 
relation avec cette entreprise.\\

Dans la catégorie RE-2) on trouve le besoin de générer des états récapitulatifs,
comme par exemple lister les entreprises pour lesquelles il y une différence
entre promesse de don et versement réalisés, extraire les entreprises dont 
l'activité est dans tel domaine, etc. Ces actions de filtrage des entreprises
ou des contacts rattachés est également nécessaire pour des actions de 
type publi-postage.\\

L'activité RE-3) consiste à enrichir la base de façon à améliorer les deux
processus précédents. MHB aimerait intégrer les personnes (et leurs entreprises)
participant au concours. L'import des prospects acquis sous une forme quelconque
fait également partie de cette catégorie.
Un point particulier est l'import des diplômés dans cette base. Bien que l'intérêt 
de les y inclure soit évident, l'obsolescence des données est difficile à déterminer.
Seuls les enregistrements des cotisants à jour constituent une information fiable.
Il s'ajoute à cette considération de qualité des données, une difficulté technique
d'identification des entreprises auxquelles ces diplômés doivent être rattachés car
l'entreprise dans la base diplômés n'a pas d'identifiant unique (e.g le numéro SIRET).



\subsection{Existant}

Bernadette Fischbach travaille avec un outil développé en 2001 par 
une étudiante (Cécile Henner) du PGE IECS. L'outil est une base 
Access interfacée. La base comporte de l'ordre de 10000 fiches 
entreprises dont environ 2000 réellement actives. La qualité
des données est maintenue par le travail quotidien de Bernadette
Fischbach. De plus, une opération d'ajout et de vérification des
numéros SIRET a été confié à un prestataire extérieur (environ
60-70\% des entreprises ont été ``SIRETé''.\\

En septembre 2010, la direction de l'EM (Michel Kalika) a confié
à l'auteur de cet audit une mission pour mettre en place un véritable
CRM. Cette demande émanait du service relation entrepise (Francis 
Schillio). Il a été décidé d'acheter cette prestation et de ne rien 
développer en interne étant donné la charge de travail de Christos 
Karacostas. Le calendrier initial prévoyait une mise en production 
à la fin du printemps 2011. A la demande du service relations 
entreprises, le calendrier du projet a été reporté pour une mise en 
production fin 2011.
Le cahier des charges après étude de marché et audition de
plusieurs offres a été établi d'après l'expression des besoins
centrée autour du service relations entreprises, stages et apprentis.
Ce cahier des charges, base de l'appel d'offre figure en annexe~\ref{ch:annexe-crm}.\\

\begin{figure}[hbt]
\begin{center}
\includegraphics[width=.9\linewidth]{figs/crm_overview.pdf}
\end{center}
\caption{Positionnement du CRM dans le système d'information de l'EM}
\label{fg:crm_overview}
\end{figure}

Le périmètre du CRM  a été volontairement limité au \sre. L'analyse 
des besoins a montré que le CRM ne pouvait pas prendre en charge 
toutes les fonctionnalités de gestion des stages et de l'apprentissage. 
Ceci nécessiterait d'ne faire un outil de gestion de scolarité également
(par exemple mémorisation des notes de stage). Comme le montre la 
figure~\ref{fg:crm_overview}, il a été décidé que la liste des étudiants 
en stage et apprentissage serait régulièrement exportée (fréquence 
mensuelle environ) vers le CRM afin de l'enrichir.\\

Etant donné que cet outil est un produit commercial dont le coût est
proportionnel au nombre de licences utilisateurs achetées, et qu'il
donne des informations financières sensibles sur les versements des
entreprises, cet outil reste confiné au \sre. D'autres services, comme
les scolarités, pourront profiter de cet outil indirectement en adressant
des requêtes au \sre.

\subsection{Analyse}
Le \sre travaille depuis dix ans avec une base de données qui lui est
propre. Cet outil est bien maîtrisé par Bernadette Fischbach qui en
a une longue expérience et possède une connaissance profonde des 
entreprises avec lesquelles l'EM a des relations régulières. \\

Cet outil est cependant limité de par sa conception et ses connexions
avec les autres bases de données du S.I. Sa conception rend l'extraction
de données assez difficile car des requêtes spécifiques nécessitent 
l'intervention de CK. La base de données entreprises est autonome, ce
qui explique la longévité du système, mais qui s'avère limitante aujourd'hui.
Le nouveau projet CRM doit améliorer cette situation. D'une part, l'outil
sera multi-utilisateurs, accessible en ligne (il est prévu qu'il soit 
consultable de l'antenne Parisienne), permettra à chaque utilisateur de 
construire ses reqûetes à travers son interface, et sera paramétrable. 
D'autre part, des données externes (stagiaires et apprentis, prospectivement 
données du concours et diplômés) seront importées pour enrichir la base.\\

Ce projet CRM, qui ne peut s'insérer complètement dans le S.I. global
(car il faudrait y gérer aussi la scolarité) est destiné au moyen terme,
en attendant d'analyser le produit CRM qui sera adopté dans le schéma
numérique directeur de l'UdS et probablement disponible à l'horizon 2014. 

  

%---------------  Communication  -----------------------------------
\section{Service communication}

\paragraph{Michèle Schmidt (MS), Isabelle Suhr (IS), Marine Julien (MJ), 
Nicolas Beyhurst (NB).}

\subsection{Processus fonctionnels}
Les processus nécessitant le support du système d'information peuvent 
être rangés en quatre catégories:
\begin{itemize}
\item[$\bullet$] SC-1) La communication d'informations à des publics ciblés.
\item[$\bullet$] SC-2) La gestion technique d'évènementiels.
\item[$\bullet$] SC-3) La collecte d'informations pour répondre aux enquêtes de type 
      palmarès.
\item[$\bullet$] SC-4) L'analyse de l'attractivité de chaque formation académique.
\end{itemize}
\bigskip

Dans la catégorie SC-1) le besoin est de constuire des \textbf{listes 
pertinentes et ciblées d'adresses electroniques}. La difficulté
est bien sûr différente selon que la communication est à destination
d'un public interne ou externe à l'établissement.

En interne, il est nécessaire de disposer d'un \textbf{annuaire}
fiable et précis du personnel et des étudiants. 

Pour la communication vers l'extérieur, le mailing est souvent 
fait de manière indirecte : l'information à communiquer transite par
le service en charge du public visé, avec un contrôle informel des 
destinataires par le service en question. Ainsi, une invitation à un 
colloque académique sera transmise par Karine Bouvier (Recherche),
une invitation à un carrefour métier sera transmise par Bernadette 
Fischbach (Relations Entreprises) ou une information aux étudiants
est transmise à la scolarité concernée pour redifusion. La communication 
vers l'extérieur nécessite donc une \textbf{base de prospects} fiable 
qui puisse être consolidée avec les données d'autres services (scolarité, 
relation entreprises, recherche), ainsi que des \textbf{procédures 
de collectes} de nouveaux prospects le plus automatisées possibles.\\

Dans la catégorie SC-2) on trouve en particulier l'activité de création
de mini-sites web par NB permettant de gérer chaque évènement (inscriptions,
programmes, informations pratiques, etc) ou du site des diplômés (alumnis).
Cette activité ne dépend pas du système d'information excepté pour
la mise à jour de l'annuaire des diplômés. Elle est au contraire 
support à la catégorie SC-1) en fournissant les données déposées par les
visiteurs sur les sites web.\\ 

La catégorie SC-3) nécessite de rassembler des statistiques provenant de 
nombreuses sources. Les informations à collecter et les services qui les 
produisent sont présentés en figure~\ref{fg:comm_flux}.
\begin{figure}[hbt]
\begin{center}
\includegraphics[width=.75\linewidth]{figs/comm_flux.pdf}
\end{center}
\caption{Les flux d'informations nécessaires pour répondre aux enquêtes de 
type palmarès}
\label{fg:comm_flux}
\end{figure}
Dans ces enquêtes, MS sépare les différentes questions posées et délègue
dans les services appropriés la charge de trouver l'information.\\


Dans la catégorie SC-4), on compte les deux enquêtes intégrés (PGE et Master 
Universitaires, formation continue) qui apportent de l'information sur la 
typologie des étudiants nouvellement recrutés et la façon dont ils ont eu 
connaissance de l'établissement et de la formation. Cette information permet 
au service de mieux comprendre l'impact des différents supports de 
communication.

Rentre également dans cette catégorie les statistiques sur les formations
Master et EE à tous les stades du recrutement. IS a construit au cours 
des années un historique de ces statistiques, formation par formation,
et peut ainsi mieux analyser l'attractivité de chacune des formations.
Ces données permettent aussi d'ajuster les opérations de communication.
Par exemple, un nombre plus faible de demandes de dossier d'inscription 
à une formation peut suggérer une opération de communication ciblée.
Pour collecter ces statistiques, une synthèse des données était auparavant
demandée (via fichier partagé en réseau) à la scolarité aux différents 
étapes du recrutement (nombre de dossiers demandés, dossiers reçus, 
dossiers acceptés). 
L'accroissement du nombre de formations et la mise en place d'ARIA rend 
cette procédure caduque. Aujourd'hui, ces données dont récupérées
directement d'ARIA mais la précision est plus faible et ARIA ne 
conserve que les pré-inscriptions pour l'année en cours. Il faudrait
donc disposer d'un stockage auxiliaire pour fabriquer un historique.


\subsection{Existant}
\begin{itemize}
\item Pour les processus de catégorie i) pour la communication interne
	les données d'annuaire fournies par l'\motcle{intranet} EM %
	\footnote{\url{http://intranet.em-strasbourg.eu/adm/fr/grh/coordonnees.html}}
	sont utilisées de manière efficace pour communiquer par mail.
	En revanche, il n'existe pas d'annuaire plus complet permettant
	par exemple de connaître les fonctions et les status (problème 
	d'édition des cartes de visites en français et anglais).


	Pour les processus de catégorie SC1) pour la communication externe,
	il existe des fichiers prospects dispersés et conservés dans des 
	formats techniques de stockage différents. Le tableau en figure 
	\ref{fg:comm_prospects} liste ces différents fichiers.
	L'utilisation de fichiers excel est la plus courante et considérée 
	la plus pratique actuellement.

\begin{figure}[hbt]
\begin{center}
	\begin{tabular}{llll}
	\hline
	\hline
	public	& source	 & format de stockage \\
	\hline
	prospects par évènements & inscription sur  mini-sites &  base de données, export csv (excel)\\
	prospects formation continue  & pré-inscriptions ARIA & import csv \\
	prospects diplômés      & service diplômés &  import csv, puis base de \\
					&			 &  données gérée dans le service Comm. \\
	étudiants après CRD 	& données APOGEE & import csv \\
	journalistes 		& contact personnel	& carnets d'adresse \\
	salons, contacts divers	& contact mail, carte visite & carnets d'adresse, fichiers excel\\ 
	VIP institutionnels	& contact mail, carte visite & carnets d'adresse, fichiers excel\\ 
	\hline
	\hline
	\end{tabular}
\end{center}
\caption{Fichiers prospects utilisés pour communiquer vers l'extérieur, source 
	d'acquisition et format de stockage}
\label{fg:comm_prospects}
\end{figure}
	Ceux-ci  ont pu être constitués par la collecte des informations saisies 
	par un participantà un évènement (inscription sur mini-site), ou lors de 
      salons (saisie manuelle). L'application ARIA de pré-candidature 
	(application de l'UdS) peut aussi être utilisée pour produire des 
	fichiers prospects (personnes ayant entamé une démarche d'inscription 
	sans finaliser).

\item Dans la catégorie SC-2) NB développe de manière réactive des sites couplés 
	à des base de données selon des formats ouverts (logiciels libres PHP, 
	MySQL) qui répondent aux besoins ponctuels de gestion d'un évènement, 
	permettant de conserver de manière pérenne ces données. Son action est 
	cependant intrinsèquement limitée par la difficile connexion à d'autres 
	sources d'informations (\motcle{intranet} EM et applications du S.I. UdS).

\item Dans la catégorie SC-3) les données ne sont pas stockées mais simplement 
	reportées aux demandeurs dans le format souhaité.

\item Dans la catégorie SC-4) les processus sont manuels et les données 
	enregistrées dans des fichiers personnels.
\end{itemize}

\paragraph{Analyse}

D'une part, le \scom a globalement besoin d'informations réparties dans la 
quasi-totalité des applications du système d'information. Aujourd'hui, la 
collecte des informations nécessaires est déléguée aux services les plus 
proches de ces applications. Cependant, il est probable que ce fonctionnement 
serait recommandé même si le système d'information était directement 
interrogeable par le \scom.

D'autre part, le \scom possède et produit ses propres données, principalement 
des contacts. La gestion de ces données est hétéroclite en raison des supports 
divers. La disponibilité de l'information n'est pas totalement maîtrisée car 
certaines données peuvent échapper aux sauvegardes car non entreposées dans des 
espaces de stockage sauvegardés par les services informatiques). Les outils 
bureautiques utilisés font que l'information est peu efficiente et peu efficace. 


Du point de vue des applications, la présence d'un informaticien au sein du 
service aide considérablement à la résolution de problèmes techniques ponctuels. 
Ceci ne peut cependant pas remplacer la mise en place se solutions applicatives 
plus globales. Il manque en particulier un outil pour faire du mailing de masse 
compatible avec les règles de sécurité anti-spam de l'UdS. L'emploi d'outils 
collaboratifs adaptés permettrait également de gagner en efficacité. 

	

%------------------------ R H -------------------------------------------------

\section{Service Ressources Humaines}

\paragraph{Laurie Walbrou (LW), Emilie Kaës (EK)}


\subsection{Processus fonctionnels}
\label{sc:rh-process}
Les processus nécessitant le support du système d'information peuvent 
être rangés en deux catégories:
\begin{itemize}
\item[$\bullet$] RH-1) La gestion du personnel administratif et enseignant
\item[$\bullet$] RH-2) L'organisation d'évènements
\end{itemize}
\bigskip

La catégorie RH-1) représente l'immense majorité de l'activité.

Le besoin est de \textbf{collecter des informations pertinentes
sur le personnel} de l'EM. La pertinence se décline selon
deux besoins plus spécifiques:
\begin{itemize}
\item RH-1.1) les besoins pour répondre aux accréditations, enquêtes 
	palmarès, au label diversité, écrire le bilan social.
\item RH-1.2) les besoins issus du dialogue de gestion avec l'Université.
\end{itemize}

\bigskip

Le besoin RH-1.1) nécessite des informations très nombreuses, précises
et changeantes d'années en années. Ainsi l'accréditation AACSB demande
par exemple pour chaque enseignant combien d'années il a passé à l'étranger 
les cinq dernières années. Ces contraintes spécifiques \textbf{obligent 
l'établissement à posséder une base propre} de ces informations étant donné 
que les logiciels de gestion de RH disponibles à l'UdS ne peuvent en aucun
cas fournir d'informations aussi spécifiques. Il est probable que cette
contrainte perdure après la mise en place du futur SI RH du schéma numérique
directeur de l'UdS (horizon 2014), à moins que celui-ci ne soit extrêmement 
paramétrable et qu'un support important de la part de l'UdS soit consacré 
aux demandes spécifiques.\\

Le besoin RH-1.2) est beaucoup moins exigeant. Il fait appel à toutes
les informations administratives contenues dans HARPEGE et SOSIE
concernant les personnels.\\


Concernant les activités de la catégorie RH-2) le besoin exprimé 
est la mise en place d'un agenda partagé permettant d'éviter
les conflits de date lors d'organisations d'évènements, et de 
diffuser l'information évenementielle sans faire d'invitation
par mail mais par un mécanisme de publication/souscription
(e.g fil RSS, synchronisation d'agenda).  



\subsection{Existant}

Le processus correspondant au besoin RH-1.1) a conduit à mettre en
place une procédure pour le recrutement des vacataires. Cette procédure
est décrite par le diagramme de séquence en figure \ref{fg:rh_seq_vacataires}.
Elle vise à enregistrer les informations perinentes concernant les enseignants
au moment de la candidature en analysant le CV d'un candidat retenu. Ces
informations sont enregistrées dans un fichier Excel maintenu par EK.

\begin{figure}[hbt]
\begin{center}
\includegraphics[width=\linewidth]{figs/rh_seq_vacataires.pdf}
\end{center}
\caption{Diagramme UML de séquence pour la procédure d'enregistrement d'un vacataire.}
\label{fg:rh_seq_vacataires}
\end{figure}

Le reste de la procédure vise à enregistrer administrativement l'enseignant
dans le S.I. de l'UdS: application SOSIE pour les heures effectuées et payées
et application HARPGest pour mémoriser les informations d'état civil et celles
nécessaire au paiement des heures. 
Cependant, ces \textbf{applications se révèlent totalement inefficaces} dans 
le reste du processus. Ainsi, les extractions d'HARPGest ne peuvent fournir
que les noms, date de naissance et nationalité d'une personne. SOSIE ne liste
que les heures de cours \emph{payées} et sur des intitulés de matière qui
peuvent avoir changé dans APOGEE.
Pour connaître l'état des paiements des heures, la procédure en place prévoit
que Sylvie Amar édite le fichier excel (partagé en réseau) pour y inscrire 
les heures payées pour chaque enseignant.\\

Les processus de la catégorie RH-1.2) subissent les mêmes manquements en terme
de support du système d'information. Le dialogue de gestion annuel implique
de faire à minima un état des lieux précis des personnels permanents de 
l'établissement. Cependant, les applications en place ne permettent même pas
d'obtenir les numéros de postes sur lesquels les personnes sont en place.
LW maintient son propre fichier pour ces postes. Elle obtient cette information
par contact téléphonique auprès d'une responsable au service RH de l'UdS, et
non sur la base d'une importation automatique de données.\\

Concernant le besoin RH-2.), aucun outil n'est aujourd'hui en place.

\subsection{Analyse}
\label{sc:rh-analyse}
Le support du S.I. aux RH est en train d'évoluer. Avant la mise en place
de la procédure décrite (figure~\ref{fg:rh_seq_vacataires}), et relativement 
aux processus décrits ici, le service RH fonctionnait sans aucun support réel 
du point de vue des systèmes d'information. La perspective de créer un outil 
spécifique pour gérer les candidatures et CV a été évoquée dès 2006 mais ne 
s'est pas réalisée. 
Confrontés aux demandes des accréditations, depuis 2010, les service RH et
Organisation et Process ont passé de l'ordre de 10 mois hommes pour analyser 
les CV papiers des enseignants intervenant (de l'ordre de 500 intervenant). 
Un nouveau projet a été conçu, intitulé \textit{Base de données CV}, dont 
le document de travail est copié en annexe~\ref{ch:rh-cvtheque}.
Idéalement, ce projet devrait mis en production en décembre 2011. 
Cependant, le manque général de planification des projets et la faible 
formalisation du cahier des charges pourrait rendre cette échéance improbable.\\

La procédure actuelle est adaptée aux demandes et devrait permettre de 
répondre assez efficacement aux questions. Cependant, le fichier excel
dans lequel figure quasiment toutes les données est le point névralgique
de ce processus, et à ce titre représente un risque certain. Etant sur
un espace commun, il est censé%
\footnote{Vérification non effectuée lors de l'audit.}
être sauvegardé par le service informatique mais la gestion des droits
étant difficile, on peut imaginer qu'un autre administratif corrompe 
(intentionnellement ou non) les données y figurant.\\

Concernant les applications, il faut retenir que les applications
du S.I. de l'Université ne fournissent quasiment aucune information en
retour (SOSIE et HARPGest). L'ADE permet de vérifier les cours effectuées.
L'\motcle{intranet} EM n'est pas du tout utilisé par le service RH.
 



% --------------- A C C R E D I T A T I O N S ----------------------
\section{Service Organisation et Process}

\paragraph{Fatiha Bouterâa (FB), Maxime Merli (MM)} 
~\\

\textit{Pour l'audit dans ce service, le sujet prioritaire concerne les 
accréditations, d'où la présence de MM pour l'entretien. Cependant, pour 
garder la cohérence du document, cette section reprend les besoins du service.}

\subsection{Processus fonctionnels}
\label{sc:sop-process}

Une très grosse partie de l'activité du service est aujourd'hui en rapport 
avec les accréditations. 
\begin{itemize}
\item[$\bullet$] OP-1) La production de statistiques pour répondre aux 
			     accréditations.
\item[$\bullet$] OP-2) Le criblage du contenu pédagogique pour répondre 
			     aux accréditations.
\end{itemize}
La gestion de la qualité est une autre activité que le SI peut contribuer à
améliorer, notamment à travers le processus suivant:
\begin{itemize}
\item[$\bullet$] OP-3) La gestion des fiches d'amélioration.
\end{itemize}
\bigskip
Les activités pour OP-1) se décomposent selon les flux d'information 
nécessaires au document à produire (différents pour chaque accréditation). 
Les flux sont comparables à ceux sollicités par le \scom. Ils sont présentés 
en figure~\ref{fg:accred_flux}.
\begin{figure}[hbt]
\begin{center}
\includegraphics[width=.75\linewidth]{figs/accred_flux.pdf}
\end{center}
\caption{Les flux d'informations nécessaires pour les accréditations}
\label{fg:accred_flux}
\end{figure}
En plus des informations que le \scom doit aggréger, le \sop doit obtenir des 
données précises sur les programmes pédagogiques. Bien que cohérentes ces 
informations nécessitent de les demander à plusieurs sources: par exemple
le nom des diplômes, des listes d'étudiants inscrits, les volumes horaires,
seront obtenus via les scolarités et la base de l'intranet EM (intervention de 
CK), les taux de sélectivité à l'entrée des formations seront obtenus du \sconc 
pour le PGE ou des responsables de formation pour les autres formations, 
tandis que l'année de création d'une formation nécessite
de faire une recherche auprès du créateur de la formation (la mémoire n'étant
pas infaillible, un recours à des documents annexes comme les plaquettes peut
être nécessaire).
Une difficulté supplémentaire s'ajoute du fait des questions posées qui 
impliquent souvent un \textbf{historique} sur les 3 à 5 dernières années, comme 
par exemple nombre de diplômés sur les trois dernières années.

Obtenir ces données se révèle encore plus compliqué pour les diplômes hébergés
hors Strasbourg (Maroc, Nancy, Paris).\\


L'activité OP-2 concerne la gestion de l'\emph{Assurance of Learning} (AOL).
Un document d'accréditation AACSB implique, outre les données quantitatives,
d'être capable de présenter une maquette pédagogique en conformité avec des
objectifs pédagogiques (\emph{learning goals}) définis au préalable. 
Les besoins sont ici de pouvoir extraire automatiquement des vues matricielles
de synthèse sur l'ensemble des formations. Cela peut être des tableaux montrant
quels \emph{learning goals} sont satisfaits par quels cours. 
En plus de ces indicateurs de la cohérence de la proposition pédagogique, il
faut pouvoir mettre en place des indicateurs sur l'efficacité de la formation,
c'est-à-dire définir des critères de réussite formulés pour chaque formation
et mesurer les taux de réussite et les taux d'atteinte des objectifs
pédagogiques.

L'évaluation des enseignements est également nécessaire dans ce processus.

Ces activités OP-2, qui agissent sur la présentation structurée des 
programmes pédagogiques, sont menées en étroite collaboration avec les directeurs
délégués des programmes dans leur activité PP-2, qui agit elle sur le contenu
même de la formation (voir section~\ref{sc:pp-process}, page~\pageref{sc:pp-process}).
\\

L'activité OP-3 devrait être collaborative, et ouverte aux utilisateurs à 
travers une interface web. Une fiche d'amélioration devrait pouvoir
être suivie en temps réel: personne en charge du dysfonctionnement constaté, 
statut de la résolution du dysfonctionnement (en cours, traité), et son 
efficacité testée après résolution (délai programmé après 
lequel le test d'efficacité est fait). Un système de relance automatique
par mail doit y être adjoint pour s'assurer de l'avancement des traitements.



\subsection{Existant}

Le processus OP-1 contient le processus RH-1.1) décrit précédemment dans la 
section~\ref{sc:rh-process} (partie qui concerne les renseignements sur les
personnels enseignants et administratifs). Comme pour RH-1.1., le même constat 
s'applique pour toutes les autres données à collecter: le S.I. ne permettait pas 
d'obtenir ces données initialement et des fichiers ad-hoc ont été construits quand 
il a fallu produire le rapport la première fois. Ces fichiers sont aujourd'hui vitaux pour
rédiger les dossiers d'accréditations à venir.\\

Pour OP-2 la gestion de l'AOL peut utiliser l'\motcle{intranet} de l'EM qui
permet à chaque enseignant de saisir les données relatives à chacun de ses 
cours\footnote{Exemple: \url{http://tinyurl.com/3ek7rus}}. Ces données 
constituent une base précieuse pour le travail de structuration de l'ensemble
de la maquette. De nouvelles fonctions pourraient aussi être ajoutées facilement
en cas d'urgence.\\

Concernant OP-2 également, une procédure d'évaluation existe depuis une dizaine 
d'années dans le programme PGE et a été généralisée à tous les programmes. Elle 
propose aux étudiants d'évaluer anonymement chaque cours auquel il était inscrit. 
Cette procédure utilise exclusivement l'\motcle{intranet} EM.
L'évaluation de la formation dans sa globalité est proposé par un autre service à l'UdS%
\footnote{\url{http://sondages.unistra.fr}, (logiciel limesurvey)}.\\

Pour OP-3 seul un support sous forme d'un fichier excel non partagé (possédé par Angèle 
N'zinga) existe actuellement. Cependant, le besoin est
bien connu dans le monde du logiciel et de nombreux outils existent%
\footnote{\url{http://tinyurl.com/3eyvb4y}}
et pourraient être transposés ici.


\subsection{Analyse}
\label{sc:sop-analyse}

Dans son travail sur les dossiers d'accréditation, le \sop et les autres
personnels impliqués ont effectué un travail considérable pour produire
les informations quantitatives exigées par ces dossiers. Ces procédures
de collectes d'information sont très faiblement automatisées. Les données
les plus facilement exploitables sont celles provenant de l'\motcle{intranet}
ou que l'EM contrôle : base entreprises, stages, apprentis, concours. 
L'exploitation des données nécessite quand même l'intervention de CK. 
Pour les données qui étaient hors de ce champ, le \sop a enregistré de
nombreuses informations dans des fichiers à part, avec les mêmes risques
que ceux mentionés précédemment (e.g section \ref{sc:rh-analyse}). 
Il serait donc judicieux de ré-injecter toutes ces informations dans le S.I.
global mais ceci implique des développements conséquents dans l'intranet.\\


L'AOL pourrait faire l'objet d'un projet démarrant immédiatement dans
l'objectif d'une accréditation AACSB à l'automne 2014. Il est en effet 
nécessaire de déployer le processus pendant deux à trois ans (durée
d'un cycle de formation pour un étudiant), car l'accréditation impose de
présenter les résultats obtenus sur au moins un cycle. Le démarrage de
l'exploitation devrait dans ce cas démarrer au premier semestre 2012,
ce qui impliquerait une défintion des besoins et une réalisation logicielle
à l'automne 2011.

Le S.I. devrait alors être repensé pour recueillir puis formatter les 
données décrites précédemment. Il devrait également prendre en compte 
les possibilités qui seront offertes par le futur S.I. \motcle{Alisée} 
de l'UdS. Etant donné que ces questions sont transversales à toutes les 
formations, concevoir un modèle de données capable de s'adapter à la 
variabilité des angles de vues, et capable de s'articuler avec les 
enseignants est un projet complexe qui nécessite, après une mise en 
{\oe}uvre rapide, une phase d'ajustement assez longue, probablement sur
plusieurs semestres. 




% --------------- C O N C O U R S ----------------------

\section{Concours }
Aida Saïd, responsable


% --------------- P R O G R A M M E S  ----------------------
\section{Programmes Pédagogiques}
\paragraph{
Géraldine Broye (GB), %
Pia Imbs (PI),
Babak Mehmanpazir (BM)} 
~\\

\textit{Les trois directeurs délégués, pour les Master Universitaires (MU), 
Executive Eductation (EE) et Programme Grandé \'Ecole (PGE) ont été audités 
séparément.}

\subsection{Processus fonctionnels}
\label{sc:pp-process}

\begin{itemize}
\item[$\bullet$] PP-1) Faire connaître les formations et solliciter des candidatures.
\item[$\bullet$] PP-2) Construire et structurer la maquette pédagogique.
\item[$\bullet$] PP-3) Monitorer le parcours de l'étudiant lors de sa formation.
\item[$\bullet$] PP-4) Evaluer les enseignements. 
\end{itemize}
\bigskip

Pour PP-1) le besoin diffère dans sa forme selon le programme. Pour EE il faut 
être capable de faire connaître la formation auprès d'un public professionnel 
ciblé. Les actions peuvent être du mailing vers des entreprises d'un secteur 
d'activité précis, ou encouragés les diplômés des promotions précédentes à être 
des prescripteurs. Pour le PGE, on cherche à recenser les étudiants ayant manifesté 
un intérêt pour le programme lors de visites dans les lycées, des journées portes 
ouvertes, salons, etc. Une nouveauté est le besoin d'étendre la prospection pour 
le recrutement d'étudiants étrangers. 
De manière idéale, le S.I. devrait permettre d'établir puis maintenir \textbf{un 
lien} permanent (mais non intrusif) entre des prospects et l'EM. Dans cette optique, 
les modes et usages de communication sont aujourd'hui bien codifiés. Il faut proposer 
un site web ou le prospect peut s'abonner à une source d'information (mailing list,
RSS ou SMS), et que cet abonnement soit complètement contrôlé par l'abonné lui-même.
~\\


La catégorie PP-2 regroupe deux activités différentes mais complémentaires. La 
construction des contenus pédagogiques est essentiellement faite par les responsables de 
formation. Elle nécessite que ces responsables recrutent des intervenants ou mobilisent 
des personnels permanents sur des cours, que les enseignants formalisent les contenus de 
cours et les modalités d'administration de ces cours. La structuration de la maquette 
pédagogique consiste à avoir une vue globale des formations offerte au niveau d'un 
programme, et d'ajuster les objectifs ou contenus de formation pour montrer la cohérence 
de l'ensemble. Cette dernière activité est une conséquence directe des exigences 
d'accréditation (voir OP-2 section \ref{sc:sop-process}). Ces deux activités ont été
regroupées dans cette catégorie car idéalement, la phase de construction des contenus 
pédagogiques devrait prévoir la saisie des éléments d'appréciation de la bonne structuration
de la maquette. Par exemple, chaque enseignant lors de la saisie de son cours, devrait se
voir présenter les \emph{learning goals} globaux du programme, et il devrait pouvoir 
indiquer lequels sont remplit par chacun des cours qu'il dispense.
~\\

Répondre au besoin de  PP-3) est pertinent pour toutes les formations, mais
a la plus forte value pour des parcours de formation complexes, comme celui du PGE.
Le souhait est d'obtenir immédiatement l'historique d'un étudiant. Cet historique
est constitué des informations pertinentes permettant de rendre compte de son pédigrée
(bac, boursier, etc)  et de son parcours scolaire (ECTS acquis, étape du diplôme, etc).
Cette fonctionnalité permet un gain de temps en supprimant la nécessité de consulter
le dossier d'inscription papier ou des sources diverses (stages, etc) pour juger
de situations particulières.
~\\

PP-4) L'activité d'évaluation des enseignements est un outil destiné à fournir un 
retour aux enseignants et responsables pédagogiques dans un objectif d'amélioration
des enseignements. L'évaluation de la formation est destinée aux responsables de
formation.
L'évaluation est également  un élément obligatoire dans le cadre des accréditations. 
Produire des évaluations est donc aussi un besoin pour le \sop (voir OP-2, 
section \ref{sc:sop-process}). 


\subsection{Existant}

PP-1) Dans l'activité de publicité et prospection, les base d'information 
utilisées sont très peu efficaces. 
\begin{itemize}
\item Pour EE, des fichiers de contacts spécifiques 
pour certains domaines de formation, comme par exemple développement durable en
2008, ont été construits à la main à partir d'annuaires d'entreprisesen (Kompass, 
Regioner). Ces fichiers ont pu être complétés par des données diverses et ponctuelles
comme des cartes de visites ou des inscriptions à des évènements thématiques.
Ces fichiers sont conservés par le secrétariat du programme EE.

Pour EE, le nouveau projet CRM représente l'espoir de mieux cibler les
démarches de prospections, si les informations sur les personnes contacts au sein 
des entreprises sont de qualité. Le fichier des diplômés est également d'un 
grand intérêt, les anciens d'une formation étant de bons prescripteurs de cette 
formation au sein de leur entreprise.

\item
La prospection pour le PGE n'est pas plus rationalisée. Les étudiants envoyés
en opération promotion dans les lycées rapportent des compte-rendus papier
inexploité(ables). Quelques contacts pris sur des salons ou aux journées portes
ouvertes sont saisis par Aida Saïd dans des fichiers excel. Pour la prospection
à l'étranger, l'EM bénéficie de la plate-forme \motcle{Pass-world} de Passerelle
qui permet de receuillir des candidatures communes aux 8 écoles françaises 
participantes. \\
\end{itemize}


PP-2) L'outil central pour cette construction est l'\motcle{intranet} avec 
la base des cours saisis par les enseignants. Dans sa forme actuelle, il 
ne permet pas d'intégrer des élements de structuration globale de la pédagogie. 
Il a cependant permis, vis-à-vis des exigences AACSB, d'exporter des listes 
exhaustives de cours avec leur contenu pédagogique, facilitant ainsi le
travail de synthèse. Une extension de la fonctionnalité existante est 
envisageable mais une réflexion plus générale sur les besoins est préférable 
étant donné l'enjeu sur le long terme. C'est la même préconisation que celle 
faite en section \ref{sc:sop-analyse}.

La construction d'un contenu de formation peut également nécessiter pour le 
responsable de formation de chercher des compétences dans des domaines sépcifiques. 
Le projet de base de données de CV évoqué en section \ref{sc:rh-analyse} pourrait 
aussi aider dans cette recherche.\\



PP-3) Les possibilités offertes aujourd'hui pour obtenir une vue synthétique
d'un dossier étudiant sont quasi-inexistantes. Il est nécessaire de consulter 
la version papier entreposée dans les scolarités. L'\motcle{intranet} EM 
 possède quelques informations mais aucune fonctionnalité n'a été développée
pour donner cette vue. Dans le cas où un tel développement était envisagé,
il serait nécessaire de prévoir le rappatriement de toutes les données 
saisies lors de la pré-inscription par l'étudiant (application \motcle{Aria}).\\


PP4) L'évaluation des enseignements est en place --le PGE pratique cette évaluation 
depuis  une dizaine d'années-- et a été généralisé à tous les programmes après
la création de l'EM (2009). Une procédure de distribution des rapports d'évaluation
sous format papier est utilisée pour éviter une dissémination qui serait contraire
aux exigences de confidentialité. Cependant, à condition que l'intranet respecte 
les standards de sécurité, cette procédure pourrait être dématérialisée en n'autorisant
que la personne concernée à consulter ses rapports à partir de son compte intranet.

L'évaluation des formations est également en place depuis un an à travers 
un outil différent (service de l'UdS%
 \footnote{\url{http://sondages.unistra.fr}, (logiciel limesurvey)}).
Un travail de communication pour inciter les étuiants à répondre, ou un ajustement 
des modalités d'amdinistration du questionnaire sont à envisager pour améliorer
le taux de réponse, parfois trop faible pour obtenir un retour significatif (par
exemple 5/40 pour MAE).
\\



\subsection{Analyse}


PP-4)
L'EM a moins de recul sur l'évaluation de la formation que sur celle des enseignements
mais l'objectif d'avoir un processus efficace sur ce point est facilement atteignable.


 
\section{Scolarités}


\section{Service International}

\section{Université numérique}


%----------------------- ----------------------------------------------

\chapter{Analyse}

 - pb interlocuteur

- pb planification projets / travailler mode projet 

- doubler les personnes sur les tâches critiques

- les personnels informatique se sentent entravés vis-à-vis de la DI

- les personnels informatique ne travaillent pas ensemble

Dépendent de l'\motcle{intranet} historique de l'IECS (CK)
- annuaire personnel EM + vacataire


%----------------------- BIBLIO ----------------------------------------------
\bibliographystyle{alpha}
\bibliography{biblio}
%----------------------- ANNEXES -------------------------------------
\appendix

\chapter{Cahier des charges du CRM}
\label{ch:annexe-crm}

Le document suivant est le cahier des charges pour le CRM.
Ce document constitue le cahier des clauses techniques particulières
utilisé dans l'appel d'offre (procédure marché à procédure adaptée)
lancé en décembre 2010. 

\includepdf[pages=1-5]{figs/crm_cctp.pdf}


\chapter{Base de données des CV}
\label{ch:rh-cvtheque}

Le document suivant décrit l'expression des besoins (dernière version du 9 mai 2011)
concernant le projet d'une base de donnée des CV.


\includepdf[pages=1-4]{figs/rh-bdcv.pdf}

\printindex
\end{document}

 
  
 

