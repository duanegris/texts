%
% Guide des themes:
% http://mcclinews.free.fr/latex/beamergalerie.php

\documentclass{beamer}
%\documentclass[handout]{beamer}
\usepackage{beamerthemesplit}

\beamertemplatetransparentcovereddynamic

%\usepackage{beamerthemeshadow}
\usepackage{pgf,pgfarrows,pgfnodes,pgfautomata,pgfheaps,pgfshade}

% pas les symboles de navigation
\setbeamertemplate{navigation symbols}{}

\mode<presentation>
%\mode<handout>{\beamertemplatesolidbackgroundcolor{black!5}}

%--- gris et rouge----
\usetheme{Boadilla}
%\usecolortheme{beaver}
%\useinnertheme[shadow]{rounded}
%--- /gris et rouge----

%albatross -> fond bleu
%crane -> barre orangé
%beetle 
%dove fly seagull wolverine beaver

%fond rosé dégradé
%\beamertemplateshadingbackground{red!10}{blue!10}

%\usepackage{pgf,pgfarrows,pgfnodes,pgfautomata,pgfheaps,pgfshade}

\usepackage[utf8]{inputenc}
\usepackage{eurosym}
\usepackage{pifont}
\usepackage{graphicx}
\usepackage{tabularx}
\usepackage{amsmath}
\usepackage{amsfonts}
\usepackage{comment}
\usepackage{wasysym} % emoticons
\usepackage{fancyvrb}
\usepackage{pgf}
\usepackage{calc}
%\usepackage{fp} %floating point
% hyperref already used in beamer : \usepackage[colorlinks=true]{hyperref}
\hypersetup{colorlinks=true} 

\definecolor{darkgreen}{rgb}{0,.7,0}
\definecolor{darkred}{rgb}{.7,0,0}

% custom Emoticon
\newcommand{\Frownie}{{\textcolor{darkred}\frownie}}
\newcommand{\Smiley}{{\textcolor{green}\smiley}}

\graphicspath{{fig/}}

%%%%%% New handy commands


\def\shorttitle{Clusters-Grids-Clouds}
\title[]{Infrastructures Parallèles de Calcul\\ {\large Clusters -- Grids -- Clouds}}

\author[~~~Stéphane Genaud]{Stéphane Genaud} \date{11/02/2011} 


\begin{document}

\frame{\titlepage}


\begin{frame}
\frametitle{Clusters - Grids - Clouds}
  \begin{block}{}
    \begin{itemize}
    \item<+-> \textcolor{darkred}{Clusters} : assemblage de ``PCs''\\+ interconnexion rapide + espace disque partagé  
    \item<+-> \textcolor{darkred}{Grids} : mutualisation de ressources à large échelle\\ (middleware + VOs)
    \item<+-> \textcolor{darkred}{Clouds} : externalisation des ressources\\ (coût=$f($temps$)$, virtualisation, SaaS/IaaS/PaaS)
    \end{itemize}
 \end{block}
%\hfill
\end{frame}



\begin{frame}
\frametitle{Clusters}
  \begin{block}{Matériel}
    \begin{itemize}
    \item Calcul: serveurs $\times$ processeurs $\times$ c{\oe}urs
    		\begin{itemize}
		\item<+->hpc (UdS) 2007 : 32 $\times$ 1 $\times$ 2 = 64cores , Opteron 2.4GHz, 4GB RAM/server
		\item<+->hpc (UdS) 2009 : 68 $\times$ 2 $\times$ 4 = 544 cores, ....
		%\item<+->griffon (G5K) : 92 $\times$ 2 $\times$ 4 = 736 cores , 16GB RAM/server
		\item<+->jade (CINES) : 2880 $\times$ 2 $\times$ 4 = 23040 cores , 34GB RAM/server (267 Tflop/s)
		\item<+->\textcolor{darkred}{Hybride} Tianhe-1H (Chine): 2560 $\times$ 2 $\times$ 4=20480 cores (Xeon) + AMD GPU : 2560 $\times$ 1600= 4096000 SPU 
    		\end{itemize}
    \item<+-> Mémoire : distribuée sur les serveurs, caches sur les processeurs/coeurs
    \item<+-> Réseau : TCP ou spécialisé (Infiniband, Quadrics, ...)
    \item<+-> Stockage : scratch, NFS, GPFS
    \end{itemize}
  \end{block}
\hfill
\end{frame}


\begin{frame}
\frametitle{Grids}
  \begin{block}{Matériel}
    \begin{itemize}
    	\item<+->`grilles ``maison'' et \href{http://desktopgridfederation.org/}{Desktop Grid}, Volunteer Computing (e.g \href{http://www.boinc.org}{Boinc})
    	\begin{itemize}
		\item Calcul, Mémoire : ordinateurs de bureau (complètement hétérogène)
    		\item Réseau : TCP, NREN ou ADSL
		\item Stockage : local et parfois distribué à large échelle. 
    	\end{itemize}
	\item<+-> Grille production (e.g \href{http://www.egi.eu/}{EGI}, \href{http://www.teragrid.org}{TeraGrid}): clusters
    	\begin{itemize}
    		\item Calcul, Mémoire : clusters
    		\item Réseau : local=clusters et  global=NREN
		\item Stockage : type cluster et parfois distribué à large échelle
    	\end{itemize}
    \end{itemize}
  \end{block}
\hfill
\end{frame}


\begin{frame}
\frametitle{Clouds}
  \begin{block}{Matériel (\Frownie souvent secret industriel) }
    \begin{itemize}
		\item Calcul, Mémoire: dépend du prix : [machine de bureau $\leadsto$ cluster]  
    		\item Réseau : local=dépend du prix, global=NREN
		\item Stockage : type cluster 
    	\end{itemize}
\pause
	\begin{small}
	Amazon EC2
	\begin{tabular}{|l|l|l|l|l|l|l|}
		\hline
		instance		& CPU			& RAM (GB) & disque (GB) &  I/O \&réseau	& \$\\
		\hline
		\hline
		small			& mono-proc		 & 1.7		& 160		&  modérée 	& 1 \\
		large			& 2 cores		& 7.5		& 850		&  élevée  		& 2\\
		xtra large		& 4 cores		& 15		& 1690	&  élevée		& 4\\
		$\vdots$		&			&		&		&			& $\vdots$ \\
		clstr compute	& 2{$\times$}8 cores &23	& 1690	&  10 Gbps		& 33.5\\
		\hline
	\end{tabular}
	\end{small}

  \end{block}
\hfill
\end{frame}

%__________________________ Pros & Cons ______________________%

\begin{frame}
\frametitle{Clusters:Pros \& Cons }

  \begin{block}{Objectifs, Avantage}
    \begin{itemize}
		\item HPC (Performances)
    		\item Fiabilité de l'infrastructure
		\item Homogénéité des systèmes/logiciels installés
		\item Facilité d'accès aux données
    	\end{itemize}
  \end{block}
  \pause
  \begin{block}{Obstacles, Inconvénients}
    \begin{itemize}
		\item Problèmes de coût et dimensionnement
    		\item Difficulté de tirer toute la performance
		\item Environnement déporté
    	\end{itemize}
  \end{block}
\end{frame}


\begin{frame}
\frametitle{Grille: Pros \& Cons }

  \begin{block}{Objectifs, Avantage}
    \begin{itemize}
		\item Mutualisation des ressources  
    		\item Meilleure utilisation des ressources ($\downarrow$ coût)
		\item Ressources nombreuses
    	\end{itemize}
  \end{block}
  \pause
  \begin{block}{Obstacles, Inconvénients}
    \begin{itemize}
		\item Instabilité de l'infrastructure (pannes)
    		\item Hétérogénéité des matériels et logiciels
		\item Recensement difficile des ressources
    	\end{itemize}
  \end{block}
\end{frame}


\begin{frame}
\frametitle{Clouds IaaS: Pros \& Cons }

  \begin{block}{Objectifs, Avantage}
    \begin{itemize}
		\item Dimensionnement facile de l'infrastucture
		\item Pas de coût de possession pour les Clouds publics 
    		\item Environnement contrôlé et homogène grâce à la virtualisation
		\item Variété des types d'infrastructure et des modèles de programmation
    	\end{itemize}
  \end{block}
  \pause
  \begin{block}{Obstacles, Inconvénients}
    \begin{itemize}
		\item Variance des performances dûes à la virtualisation 
    		\item Matériel utilisé non divulgué dans les clouds publics
		\item Perte des compétences sur l'infrastructure mtérielle
    	\end{itemize}
  \end{block}
\end{frame}



%__________________________ Domaines d'applications ______________________%
\begin{frame}{}
\frametitle{Domaines d'applications}



	\resizebox{\textwidth}{!}{
	\begin{tabular}{|p{2cm}|p{2.8cm}|p{2.2cm}|p{2.2cm}|p{2.2cm}|}
		\hline
					& cluster		&  grid    			& grid prod. 		&  cloud (IaaS) \\
		\hline
		\hline
		resources		& homogène		& hétérogène		&  fédér. homogènes	&   fédér. homogènes \\ 
		\hline
		gestionnaire	& batch		& middleware		& meta-batch		&   manuel+transparent \\
		\hline
	      marché dominant	& HPC		 			&  calcul			& calcul + I/O 		  &  web servers 	\\
		\hline
		parallélisme	& div. domaine			&  distribution tâches	& distribution programmes &  distribution systèmes\\
		\hline
		modèle prog.	& SMPD				&  client/server		& séquentiel, workflow	  &  tout  		\\
		\hline
		exemples		& MPI [+OpenMP,] [+opencl], PGAS & Boinc, Condor		& tout lang.		  &  tout lang.	\\ 
		\hline
		\hline
	\end{tabular}
	}

\end{frame}


\end{document}
