

Besoin de parallélisme pour traiter les données.


Exemple: 20 milliards pages web x 20KB = $4* 10^{14}$ (400 TB)
• vitesse lecture disque 30-35 MB/sec
$\Righarrow$ 4 mois de lecture
• nécessite de l'ordre de 1000 disques
• Traitement de ces données encore plus long et fastidieux


MapReduce
=========
Un modèle de programmation

* avec un schéma très contraint

Mais permet
* parallélisation automatique
* de l'équilibrage de charge
* des optimisations sur les transferts disques et réseaux
* de la tolérance aux pannes


Applications typiques
=====================


Base: un problème qui nécessite la lecture de très 
nombreuses données uniformes.

* Map : extraction d'une caractéristique particulière présente dans chaque donnée,
puis tri.
* Reduce : agréger les résultats obtenus: opérande, tri ou transformation, ...

Le schéma de l'application reste constant, ce sont les 
fonctions Map et Reduce qui changenet selon le problème.


Exemple: fréquence de mots dans des fichiers
=============================================

f1: "le soleil brille sur le fleuve"
f2: "L'or brille plus que l'argent"
f3: "...."

Map count f1 : [('le',1);('soleil',1);('brille',1);('sur',1);('le',1);('fleuve',1)]
    count f2 : [('L',1);('or',1);('brille',1);('plus');('que',1);('l',1);('argent',1)]

Reduce :
     'le',[1;1]     ->  'le',2
     'soleil',[1]   ->  'soleil',1
     'brille',[1;1]   ->  'brille',2










