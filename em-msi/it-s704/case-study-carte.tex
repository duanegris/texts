\documentclass[12pt]{article}
\usepackage{a4wide}
\usepackage[francais]{babel}
%\usepackage{times}
\usepackage{geometry}
\usepackage[utf8]{inputenc} 
%\setlength\textwidth{15.5cm}
%\setlength\textheight{20cm}
%\setlength\rightmargin{2cm}
\pagestyle{empty}
\geometry{a4paper}
\begin{document}

\begin{center}
\begin{Large}
\'Etude de Cas \\
\textit{Carte d'étudiant nouvelle génération (CENG)}
\end{Large}

\begin{Large}
\emph{} (IT-S704)\\
\'Ecole de Management Strasbourg\\
\end{Large}

\vspace{.3cm}
\begin{large}
Romaric David, Stéphane \textsc{Genaud}, Novembre 2009 
\end{large}

\end{center}







\vspace{.5cm}
%----------------------------------------------------------------------------------
\section{Contexte}
L'Université de Strasbourg veut mettre à disposition des étudiants une carte d'étudiant
leur donnant accès à des services supplémentaires que ceux traditionnellement proposés
avec ce type de carte. Elle souhaite par exemple que la carte donne immédiatement
accès aux transports en commun de la ville et aux services universitaires de restauration 
gérés par le CROUS.
Les services de la carte sont également étendus par rapport à ceux proposés précédemment
à l'intérieur de l'Université: la carte doit donner accès à l'étudiant à ses relevés de notes,
et permettre également un accès sécurisé à certaines bornes en libre-service (BLS) disposant
d'écrans tactiles.\\

\noindent %
Ci-dessous est présenté un ensemble d'éléments extraits du cahier des charges.


\section{Fonctionnalités}

\begin{itemize}
\item  Identification visuelle du porteur : la carte comportera les noms, prénoms, 
numéro national étudiant (INE), le nom de la composante où est inscrit l'étudiant
et le porteur de la carte pourra être identifié par la  photo figurant sur la carte.\\

\item  Accès réseau transport : la carte pourra être utilisée dans tous les automates
des transporteurs ayant signé une convention avec l'université. Dans l'immédiat, seul
la Compagnie des Transports Strasbourgeois (CTS) est concernée.\\

\item  Accès service de restauration : la carte pourra être utilisée dans tous les 
restaurants du CROUS.\\

\item  Consultation relevé de notes :  à partir de BLS, l'étudiant devra pouvoir consulter
	son relevé de notes.\\

\item  Accès à l'Espace Numérique de Travail (ENT) sur borne libre service :  
      en enfichant sa carte sur la borne, l'étudiant pourra accéder à son compte ENT.

\end{itemize}


\section{\'Eléments du cahier des charges}

\subsection{Université}
\begin{itemize}
\item 
L'ENT utilise un annuaire LDAP.
La création de la carte étudiant doit entrainer la création du compte informatique.
Le LDAP contient les nom, prénom, identifiant, et mail de l'étudiant.


\item
Le service central de scolarité utilise une base de données Oracle.
La base de données de la scolarité contient le nom, prénom, INE, 
diplôme étudiant,  composante (EM, autre école, ...).
Le système de gestion de base de données du service central de scolarité autorise
certaines requêtes provenant de machines localisées dans le réseau de 
l'Université et déclarées au préalable.
Parmi ces requêtes, il est possible de demander la création d'un numéro INE:
le système d'information de la scolarité est en effet synchronisé avec des 
serveurs nationaux, ce qui lui permet de déterminer un numéro INE unique.

\item 
Lors de la création d'une carte, il faut donc demander la génération
d'un numéro INE par l'intermédiaire d'un première requête. Celle
ci créé un enregistrement dans le système d'information de la scolarité.
Le numéro INE généré est récupéré et mémorisé par le système d'information de la
CENG.


\end{itemize}

\subsection{CROUS}
\begin{itemize}
\item 
Le CROUS et les compagnies de transports ont leur propre système d'information, qui stocke l'ensemble
de leurs clients, chacun sous un format propriétaire auquel on ne peut avoir accès.

\item
Le CROUS fait une mise à jour quotidienne et à heure fixe de ses usagers, en prenant en compte
les nouveaux inscrits et les usagers dont l'inscription périme.
À tout usager inscrit est associé une période de validité, mémorisée dans la base de données du CROUS.

\item
Le CROUS ne souhaite pas que ses bases soient accédées en écriture.
Par contre, l'étudiant doit pouvoir accéder à un restaurant universitaire dès le lendemain de son inscription.
Il est prévu que le CROUS fournisse annuellement à l'Université une plage d'identifiants réservés pour attribution
par l'Université.
\end{itemize}


\subsection{Transporteurs}
\begin{itemize}
\item 
Avant l'introduction de la carte CENG, la création d'un client à la CTS est uniquement faite par un opérateur 
dans un point de vente. Depuis son terminal, l'opérateur créé instantanément l'enregistrement dans les 
bases de la CTS. 
Avec le projet CENG, la création de la carte étudiant doit automatiquement enregistrer l'étudiant comme nouveau
client du transporteur de son choix (à l'heure actuelle, le seul choix possible est la CTS).

Immédiatement après délivrance, la CENG fait office de support d'abonnement auprès du transporteur.
L'abonnement n'est pas crédité. L'étudiant doit le créditer par les moyens habituels prévus par le
transporteur (bornes, guichets, ...). 


\item  L'Université a obtenu l'accord du transporteur pour faire des statistiques sur le nombre de trajets 
      effectués par les étudiants afin de mesurer l'attractivité du transporteur.
	 Une fois par mois, le transporteur s'engage à transmettre un fichier indiquant combien de trajets
	 ont effectués chacun de ses clients porteurs de la CENG.

\end{itemize}



\subsection{Modalités de délivrance}
\begin{itemize}
\item 
La délivrance de la CENG se fait en présence de l'étudiant. Elle doit être immédiate.

\item 
La base de données CENG doit contenir nom, prénom, photo, 
identifiant chez le transporteur, identifiant au CROUS, numéro INSEE, 
numéro INE, composante d'inscription.

\item
La création de la carte remplace l'inscription administrative à l'Université
(l'étudiant ne passe plus dans les bureaux de la scolarité comme les années précédentes).
Elle n'est faite que la première fois que l'étudiant s'inscrit à l'Université.
Par conséquent, avant la création de la carte, l'étudiant ne figure pas dans
le système d'information de l'Université.

\end{itemize}


\section{Travail à réaliser}

\noindent
Le travail démandé consiste à répondre à cet appel d'offres carte 
d'étudiant nouvelle génération, en proposant une nouvelle organisation des S.I. 
permettant le bon fonctionnement de cette carte. 
Votre proposition est libre dans la forme. 
L'exhaustivité de la modélisation des flux d'information, du modèle conceptuel 
des données et des traitements, et la clarté avec laquelle elle est présentée 
fera la qualité de votre réponse.

Pour vous aider dans votre réponse, vous pourrez répondre aux points suivants.

\begin{enumerate}
\item  Représenter schématiquement les différents systèmes d'information et les flux entre ces systèmes.
\item  Lister les cas d'utilisation qui vous semblent pertinents.
\item  Proposez un processus pour la délivrance de la carte.
      Pensez que L'étudiant doit obtenir login, mot de passe et adresse mail
 Décrire les flux correspondant. 

\item Décrivez ce qui devra être prévu (matériel, logiciel) sur le poste de travail où se réalise
la délivrance de la CENG.
%% Le poste de travail devra comporter :
%% - Un ordinateur de bureau
%% - Le nécessaire pour numériser une photo. Pour la Carte multi-services, ils ont prévu un scanner et une webcam pour faire des photos
%% - L'application de saisie de la CENG, la connexion à la BD

\item  Précisez la forme que prend la création de l'étudiant dans la base de donnée scolarité.
\item  Montrez que votre modèle conceptuel des données permet de faire des statistiques sur les trajets effectués. 
	 Par exemple, pouvez vous indiquer le mois de l'année où le nombre de trajets en transports en commun est
	le plus important.
\item  Proposez un processus de transmission d'informations vers le CROUS
\item  Afin de recalculer les subventions au CROUS, l'établissement doit connaitre
le montant des dépenses effectuées au restaurant universitaire. Votre modèle conceptuel des données permet-il
de le calculer ?
\end{enumerate}

Pensez à représenter les aspects statiques et dynamiques du système. Vous 
pourrez par exemple représenter les données stockées en utilisant un diagramme de 
classes UML. Les processus pourront être représentés avec des diagrammes
de séquences ou des diagrammes d'états.


\end{document}
