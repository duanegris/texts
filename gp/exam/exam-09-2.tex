\documentclass{report}
\usepackage[francais]{babel}
%\usepackage{times}
\usepackage{geometry}
\usepackage[latin1]{inputenc} 
%\setlength\textwidth{15.5cm}
%\setlength\textheight{20cm}
%\setlength\rightmargin{2cm}
\pagestyle{empty}
\geometry{a4paper}
\begin{document}
 
\begin{center}
\begin{Large}
	Sujet d'examen
\end{Large}

\begin{Large}
\emph{Outils pour la gestion de projet} (IT-S601)\\
Ecole de Management Strasbourg\\
\end{Large}

\vspace{.3cm}
\begin{large}
St�phane \textsc{Genaud}, d�cembre 2009 
\end{large}
 
\end{center}


\noindent {\footnotesize
dur�e : 2 heures\\
documents non autoris�s\\
calculatrice autoris�e
}
\vspace{1cm}\\
%

%----------------- Question -------------------

\vspace{.5cm}
 
\noindent  $\diamond$ {\em Question 1} (6pts)

Pour la r�alisation de l'intranet d'une �cole, vous �tes
charg�s par le chef de projet d'analyser une partie du
projet nomm� sous-projet \textsf{�tudiants}.
On vous demande de lister cinq rubriques d'information qui seront 
publi�s sur le site (un exemple de rubrique pourrait �tre
"la consultation des notes obtenues").
Ensuite, fa�tes la liste compl�te des acteurs qui interviendront
sur le sous-projet (services, sous-traitants, etc..) et
proposer un d�coupage WBS et OBS du sous-projet.



\vspace{.5cm}
%----------------- Question -------------------
\noindent  $\diamond$ {\em Question 2}  (14 pts)

\noindent L'analyse du projet vous fournit le tableau suivant.
Il liste les t�ches (non-pr�emptibles) et leurs d�pendances (en indiquant les successeurs)
ainsi que les dur�es vraisemblables (\textrm{vrai}), pessimistes (\textrm{pess})
et optimistes (\textrm{opt}) de chacune des t�ches. 
\[\begin{array}{|l|l|c|c|c|}
\hline
\hline
\textrm{t�che}& \textrm{successeur} & \textrm{vrai} &\textrm{pess} & \textrm{opt}\\
\hline
t_1		&   t_5,t_4	& 2 &   3 & 1    \\
t_2		&   t_4	& 5 &   5 & 5      \\		
t_3		&   t_4,t_8	& 3 &   6 & 3      \\
t_4		&   t_6    	& 5 &   5 & 5      \\
t_5		&   t_6 	& 7 &   8 & 6	\\
t_6		&   t_7 	& 5 &   7 & 3      \\
t_7		&   fin 	& 10 &  12  & 8     \\
t_8		&   t_7  	& 5  &  8 & 4\\
\hline
\hline
\end{array}
\]\\

\noindent
\begin{itemize}

\item[\textit{a)}] \emph{Graphe PERT} \\
Tracer le graphe PERT correspondant en utilisant les dur�es vraisemblables
pour calculer et reporter sur le graphe, les dates au plus t�t et au plus
tard, ainsi que les marges. Noter �galement le chemin critique.\\

\item[\textit{b)}] \emph{Dur�e}\\
Donner la dur�e du projet en hommes $\times$ jours en utilisant le graphe PERT.\\


\item[\textit{c)}] \emph{Deux personnes}\\
Dire si l'on peut finir le projet dans le d�lai minimum donn� par
le graphe PERT si l'on fait l'hypoth�se que l'on dispose de deux
personnes travaillant � temps complet comme ressources.
Si ce n'est pas possible, en combien de jours pourrait on finir
au mieux ?
Dans tous les cas, dessiner le diagramme Gantt justifiant votre
r�ponse.\\

\item[\textit{d)}] \emph{$n$ personnes}\\
Faire un diagramme de Gantt avec autant de personnes que vous le
souhaitez. Bien s�r, le projet doit s'achever dans les d�lais donn�s
par le graphe PERT.\\

\item[\textit{e)}] \emph{Variance du chemin critique}\\
Calculer la dur�e probable $D$ (somme des dur�es probables des t�ches) 
ainsi que l'incertitude $E$ du 
chemin critique. Diriez vous que la dur�e estim�e 
l'est avec une certitude faible ou elev�e ?\\



\item[\textit{f)}] \emph{Finir avant la date}\\
Soit ${\cal D}(p)  = D_C + E_C \cdot G(p)$ 
la dur�e d'un chemin $C$ avec une probabilit� $p$.
Donner un encadrement de la probabilit� de finir le projet en 24,5 jours. 




\end{itemize}

\vspace{1cm}

\noindent \textbf{Annexe} : pour le PERT probabiliste, 
la loi normale utilis�e associe en particulier les valeurs suivantes :
$$\begin{array}{cc}
 p & G(p) \\
 \hline
  90\% & 1,28 \\
  87\% & 1,19 \\
  70\% & 0,52 \\
  50\% & 0\\
  34,5\% & -0,4\\
  27,4\% & -0.6
\end{array}$$

\end{document}



%\begin{comment}
%----------------------------------------------------------
\begin{tabbing}
xxxxx\=xxxx\=xxxxxx\=xxxxxxxxxx\=xxxx\=xxxxxxxxx\=xxxxxxxxxxxxxxxxxxxxxx  \kill
R1   \> t2 \>      \> t4       \>mmmm\> t6      \> t7 \\
R2   \> t1 \> t3   \> t5       \>    \> t8      \> mmmmmmmmmmmmmmmmmmmmm\\
\end{tabbing}
%\end{coment}
%\begin{comment}
%----------------------------------------------------------
$$
D(C) = prob_2 + prob_4 + prob_6 + prob_7
      = 5   +  5  + 5   + 10
	 = 25
$$

$
d_i = \frac{pess_i - opt_i}{6}\\
$
$$
\begin{array}{lll}
E_C  & = & \sqrt{d_2^2 + d_4^2 + d_6^2 + d_7^2} \\
     & = & \sqrt{0   + 0   + 4/9 + 4/9}  \\
    &  = & \sqrt{8/9}= \frac{2 \sqrt{2}}{3} = .942809\\ 
\end{array}
$$


%-------------------------------------------------------
Finir en 24,5 jours :
  
   D(p) = 24,5
   D_C + E_C * G(p) = 24.5
   25  + 0.942.G(p) = 24.5
   G(p) = -0.5 / 0.942 = -.53
   =>
   27,4\% < p < 34,5\% 
%\end{coment}
