\documentclass{report}
\usepackage[francais]{babel}
%\usepackage{times}
\usepackage{geometry}
\usepackage[T1]{fontenc} 
\setlength\textwidth{15.5cm}
\setlength\textheight{20cm}
\setlength\rightmargin{2cm}

\geometry{a4paper}
\begin{document}
 
\begin{center}
\begin{Large}
	Sujet d'examen
\end{Large}

\begin{Large}
\emph{Introduction � la gestion de projet} (NTC-211)\\
I.E.C.S., 2\ieme ann�e\\
\end{Large}
\begin{large}
Nicolas \textsc{Cytrynowicz}, 
St�phane \textsc{Genaud}, Janvier 2000 %\oldstylenums{1999}
\end{large}
 
\end{center}


\noindent {\footnotesize
dur�e : 2 heures\\
documents non autoris�s
}
\vspace{1cm}\\
%----------------- Question -------------------

\vspace{.5cm}
 
\noindent  $\diamond$ {\em Question 1} (8 points)

\begin{itemize}
\item[\textit{a)}]
<<Il faut communiquer plus et planifier moins>>.  Qu'en pensez-vous ?\\


\item[\textit{b)}]
D�crire les acteurs du projet, en les d�finissant et en donnant leurs r�les respectifs.\\

\end{itemize}

%----------------- Question -------------------
\noindent  $\diamond$ {\em Question 2} (12 points)   
Dans le cadre d'un projet, on a identifi� les t�ches suivantes, leurs dur�es en jours (j) et leurs d�pendances (toutes de type fin-d�but). 

\[\begin{array}{|l|c|l||c|c|c|}
\hline
\hline
\textrm{t�che}		& \textrm{dur�e}	& \textrm{successeur(s)} & opt & prob & pess \\
\hline
\textrm{d�but}	&	0	& t_1,t_2,t_3	& & &\\
t_1		& 	3	& t_4	&               2&3&5\\
t_2		&	6	& t_5   &				5&6&9\\		
t_3		& 	2	& t_5,t_6	&			2&2&2\\
t_4		&	6	& t_7		&		4&6&8\\
t_5		&	3	& t_7 		&		3&3&4	\\
t_6		&	5	& \textrm{fin} &	4&5&6\\
t_7		&	1	& \textrm{fin} &	1&1&1 \\
\hline
\hline
\end{array}
\]

\begin{itemize}
\item[\textit{a)}] 
Dessinez le graphe Pert correspondant au tableau ci-dessus, puis calculer les dates au plus t�t, au plus tard et les marges (on suppose que la marge sur le chemin critique est nulle). 
Indiquer les t�ches appartenant au chemin critique.\\

\item[\textit{b)}] 
Pour accomplir les t�ches du projet, les ressources � disposition sont \emph{John}, \emph{Cyntia}, et \emph{Kevin}. 
Etablissez un diagramme Gantt du projet, sachant que \emph{John} et \emph{Kevin} ne peuvent pas commencer avant le quatri�me jour.\\

\item[\textit{c)}]
Etablissez un diagramme Gantt du projet, avec seulement \emph{John} et \emph{Cyntia} (\emph{John} est cette fois enti�rement disponible). 
En combien de temps finirait on au plus t�t dans ces conditions ?


\end{itemize}
\end{document}
