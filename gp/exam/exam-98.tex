\documentclass{report}
\usepackage{times}
\usepackage{geometry}
\usepackage[francais]{babel}
\usepackage[T1]{fontenc} 
\setlength\textwidth{15.5cm}
\setlength\textheight{20cm}
\setlength\rightmargin{2cm}

\geometry{a4paper}
\begin{document}
 
\begin{center}
\begin{Large}
	Sujet d'examen
\end{Large}

\begin{Large}
\emph{Introduction � la gestion de projet} (NTC-211)\\
I.E.C.S., 2\ieme ann�e\\
\end{Large}
\begin{large}
St�phane Genaud, Janvier 1999 %\oldstylenums{1999}
\end{large}
 
\end{center}


\noindent {\footnotesize
dur�e : 1 heure 30\\
documents non autoris�s
}
\vspace{1cm}\\
%----------------- Question -------------------
\noindent  $\diamond$ {\em Question 1} :  
Dire quels sont les acteurs que l'on peut trouver dans un projet
important. D�crire leurs r�les et dire quels peuvent �tre leurs int�r�ts et
motivations.

\vspace{.5cm}
 
 
%----------------- Question -------------------
\noindent  $\diamond$ {\em Question 2} :  
Dans le cadre d'un projet, on a identifi� les t�ches suivantes, leurs dur�es en jours (j) et leurs d�pendances (toutes de type fin-d�but). 

\[\begin{array}{|l|c|l|}
\hline
\hline
\textrm{t�che}		& \textrm{dur�e}	& \textrm{successeur(s)}\\
\hline
\textrm{d�but}	&	0	& t_1,t_2	\\
t_1		& 	6	& t_3,t_4	\\
t_2		&	3	& t_3		\\
t_3		& 	2	& t_5,t_7	\\
t_4		&	6	& t_8		\\
t_5		&	3	& t_6 		\\
t_6		&	5	& \textrm{fin} \\
t_7		&	1	& \textrm{fin}	\\
t_8		&	6	& \textrm{fin} \\
\hline
\hline
\end{array}
\]

\begin{itemize}
\item[\textit{a)}] 
Dessiner le graphe Pert correspondant au tableau ci-dessus, puis calculer les dates au plus t�t, au plus tard et les marges (on suppose que la marge sur le chemin critique est nulle). 
Indiquer les t�ches appartenant au chemin critique.\\

\item[\textit{b)}]
L'entreprise dispose de quatre personnes capables de r�aliser n'importe laquelle des t�ches du projet : Pierre, Michel, Marie et Sophie. Sophie ne travaille qu'� mi-temps.\\
On estime que ces personnes ont un co�t en francs (F) pour chaque jour travaill� : Pierre et Michel co�tent � l'entreprise 100 F/j, Marie 110F/j et Sophie 45 F/j.\\

Dire quel est le nombre minimum de personnes n�cessaires, pour l'ach�vement du projet dans le d�lai donn� par le r�seau Pert. 
Dans ce cas, quelles sont les personnes qu'il faut faire travailler pour que le projet co�te le moins cher possible.\\

% r�ponse 2 personnes: 
%	Michel : {t1;t4;t8} (chemin critique)
% 	Pierre : {t2;t3;t5;t6;t7} (t5 et t7 faites en s�quence) 
%	Michel : cout = (6 + 6 + 6) * 100 = 1800F
%	Pierre : cout = (3 + 2 + 3 + 1 + 5) * 100 = 1400 F
%		 Total  cout = 3200 F
 
\item[\textit{c)}]
Etablir un diagramme de Gantt du projet avec ces personnes
et indiquer le co�t du projet dans ce cas.\\

\item[\textit{d)}] 
Sophie aimerait intervenir sur le projet.
Peut on, sans allonger la dur�e du projet, lui confier certaines t�ches ?
Si oui, quelles t�ches doit on lui confier et cela serait il avantageux en 
termes de co�t du projet (indiquer le co�t du projet dans ce cas) ?
S'il n'est pas possible ou pas souhaitable de la faire intervenir, 
donnez vos raisons.
%r�ponse : oui, mais elle ne peut �tre affect�e qu'aux t�ches dont la dur�e
% est >= � la marge.
%	Sophie : {t2;t3;t7} -> cout = (6 + 4 +2 )*45 = 540 F
%	Michel : {t1;t4;t8} -> cout = (6 + 6 + 6) * 100 = 1800F
%	Pierre : {t5;t6}    -> cout = (3 + 5) * 100 = 800 F
%			       Total  cout = 3140

\end{itemize}
\end{document}
