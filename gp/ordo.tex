
\subsection{Ordonnancement}

\frame{
\frametitle{Ordonnancement}

\begin{itemize}
\item
Affecter des tâches à des ressources dans un ordre,
c'est \textbf{ordonnancer}.

\item
De nombreux résultats proviennent de la recherche opérationnelle.\\
(exemple: Flow job, Job shop.)

\item
Beaucoup des problèmes d'ordonnancement sont NP-complets. \\
Des heuristiques donnant de "bons" résultats souvent utilisées.
\end{itemize}
}


\frame{
\frametitle{Ordonnancement}


Il faut connaître les hypothèses. 
\begin{itemize}
      \item Tâches connues à l'avance ? 
		Si oui, ordonnancement \emph{statique}, si non ordonnancement \emph{online}.
      \item Tâches dépendantes les unes des autres ?
	\item Tâches préemptibles ? préemptible: peut être interrompue.
	\item Ressources hétérogènes ? connaître la capacité de travail de la ressource. 
	\item $\dots$ 
\end{itemize}
}

%---------------------------- Heuristics independent tasks ----------------------------

%----------------------------  Min-min --------------
\subsection{Heuristiques tâches indépendantes}

\frame{
\frametitle{Heuristiques tâches indépendantes}

	Heuristiques tâches indépendantes, Ressources hétérogènes
	\begin{itemize}	
	\item Min-min
	\item Max-min
	\item Sufferage
	\end{itemize}	
}
\frame{
\frametitle{Min-min}
	
\begin{itemize}
\item
	Pour chaque tâche $T_i$, 
     calculer son temps de fin sur toutes les machines et retenir le minimum, appelé $\beta_i$.

\item garder la tâche qui a le temps le plus court: $\stackrel{min}{i}$ $\beta_i$

\item affecter la tâche sur la machine qui donne ce temps

\item recommencer jusqu'à ce que toutes les tâches soient ordonnancées

\end{itemize}	

}



\frame{
\frametitle{Min-min}


%---1--
\only<1,2>{
$$
\begin{array}{|l|l|l|l|}
\hline
		& T_1 & T_2 & T_3  \\
\hline
R_1		& 140	& 20	& 60   \\
\hline
R_2		& 100 & 100 & 70 	\\
\hline
\end{array} $$
}

%---2--
\only<3>{
$$
\begin{array}{|l|l|l|l|}
\hline
		& T_1 &  \textcolor{red}{T_2}  & T_3  \\
\hline
R_1		& {\bf 160}	&  \textcolor{red}{20} &  {\bf 80}   \\
\hline
R_2		& 100 &  --			& 70 \\
\hline
\end{array} $$
}

%---3--
\only<4>{
$$
\begin{array}{|l|l|l|l|}
\hline
		& T_1 &  \textcolor{red}{T_2}  & \textcolor{blue}{T_3}  \\
\hline
R_1		& 160	&  \textcolor{red}{20} &  --   \\
\hline
R_2		& 100 &  --				& \textcolor{blue}{70} \\
\hline
\end{array} $$
}
%---3--
\only<5>{
$$
\begin{array}{|l|l|l|l|}
\hline
		& T_1 &  \textcolor{red}{T_2}  & \textcolor{blue}{T_3}  \\
\hline
R_1		& 160	&  \textcolor{red}{20} &  --   \\
\hline
R_2		& {\bf 170} &  --				& \textcolor{blue}{70} \\
\hline
\end{array} $$
}

%---7--
\only<6>{
$$
\begin{array}{|l|l|l|l|}
\hline
		& \textcolor{brown}{T_1} &  \textcolor{red}{T_2}  & \textcolor{blue}{T_3}  \\
\hline
R_1		& \textcolor{brown}{160}	&  \textcolor{red}{20} &  --   \\
\hline
R_2		& 170 &  --				& \textcolor{blue}{70} \\
\hline
\end{array} $$
}

\pause
\begin{itemize}
\item<+-> $T_2$ sur $R_1$ (20)
\item<+-> tâches sur $R_1$ vont terminer 20 unités de temps plus tard
\item<+-> $T_3$ sur $R_2$ (70)
\item<+-> tâches sur $R_2$ vont terminer 70 unités de temps plus tard
\item<+-> $T_1$ sur $R_1$ (160)
\end{itemize}
}


%----------------------------  Max-min --------------

\frame{
\frametitle{Max-min}

Soit $T$ l'ensemble des tâches à ordonnancer.
\begin{itemize}
\item
Pour chaque tâche $T_i \in T$ 
     calculer son temps de fin sur toutes les machines et retenir le minimum, appelé $\beta_i$.

\item garder la tâche qui a le temps de fin maximum: $\underset{i}{max}$ $\beta_i$

\item affecter la tâche sur la machine qui donne ce temps

\item recommencer avec l'ensemble $T \backslash T_i$,
       jusqu'à ce que toutes les tâches soient ordonnancées

\end{itemize}	

}



\frame{
\frametitle{Exemple Max-min}

%---1--
\only<1,2>{
$$
\begin{array}{|l|l|l|l|}
\hline
		& T_1 & T_2 & T_3  \\
\hline
R_1		& 140	& 20	& 60   \\
\hline
R_2		& 100 & 100 & 70 	\\
\hline
\end{array} $$
}

%---2--
\only<3>{
$$
\begin{array}{|l|l|l|l|}
\hline
		& \textcolor{red}{T_1} &  T_2  & T_3  \\
\hline
R_1		& --	&  		20 &   60 \\
\hline
R_2		& \textcolor{red}{100} &  200	& 170 \\
\hline
\end{array} $$
}

%---3--
\only<4>{
$$
\begin{array}{|l|l|l|l|}
\hline
		& \textcolor{red}{T_1} &  T_2  & \textcolor{blue}{T_3}  \\
\hline
R_1		& --	&  		20 &   \textcolor{blue}{60} \\
\hline
R_2		& \textcolor{red}{100} &  200	& 170 \\
\hline

\end{array} $$
}


%---3--
\only<5>{
$$
\begin{array}{|l|l|l|l|}
\hline
		& \textcolor{red}{T_1} &  T_2  & \textcolor{blue}{T_3}  \\
\hline
R_1		& --	&  		{\bf 80}  &   \textcolor{blue}{60} \\
\hline
R_2		& \textcolor{red}{100} &  200	& 170 \\
\hline
\end{array} $$
}

%---7--
\only<6>{
$$
\begin{array}{|l|l|l|l|}
\hline
		& \textcolor{red}{T_1} &  \textcolor{brown}{T_2}  & \textcolor{blue}{T_3}  \\
\hline
R_1		& --	&  \textcolor{brown}{80}	  &   \textcolor{blue}{60} \\
\hline
R_2		& \textcolor{red}{100} &  200	& 170 \\
\hline
\end{array} $$
}

\pause
\begin{itemize}
\item<+-> $T_1$ sur $R_2$ (100)
\item<+-> tâches sur $R_2$ vont terminer 100 unités de temps plus tard
\item<+-> $T_3$ sur $R_1$ (60)
\item<+-> tâches sur $R_1$ vont terminer 60 unités de temps plus tard
\item<+-> $T_2$ sur $R_1$ (80)
\end{itemize}
}


\begin{comment}
\subsection{Heuristiques tâches dépendantes}

\frame{
\frametitle{}
	Heuristiques tâches dépendantes, Ressources hétérogènes
	\begin{itemize}	
	\item HEFT 
	\item 
	\item 
	\end{itemize}	
}

\end{comment}

