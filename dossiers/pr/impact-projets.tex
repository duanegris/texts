\subsection{Impact des projets} 

\paragraph{Parall�lisation pour la g�ophysique}
La suite logicielle \textsc{ray2mesh} con�ue, d�velopp�e et test�e pour des architecures parall�les 
a �t� utilis� dans plusieurs travaux. 
Les caract�ristiques favorisant l'utilisation de ce logiciel sont, d'un point de vue de la conception,
la modularit� des biblioth�ques et la portabilit� (test�e sur de nombreuses plateformes Unix), et d'un point de vue scientifique, une application de tomographie sismique produisant des r�sultats pertinents d'un point de vue g�ophysique.
Trois �tudiants en th�se l'utilisent actuellement pour confronter leurs propositions � ce qui est consid�r� 
comme une application ``r�aliste''. Ludovic Hablot (�quipe Projet INRIA RESO), �tudie l'influence des tailles de buffers
sur la performance des �changes TCP longues distance quand l'application s'ex�cute sur plusieurs sites, 
Camille Coti (INRIA GrandLarge, Orsay) compare les temps de d�marrage de l'application � travers plusieurs middleware, et Matthieu Cargnelli (EADS + INRIA GrandLarge, Orsay) propose un workflow (OpenWP) et utilise \textsc{ray2mesh} comme exemple.
Lors des pr�sentations de Grid5000, \textsc{ray2mesh} est l'exemple cit� comme exp�rimentation d'une application
scientifique sur grille \cite{cappello07}.  



\paragraph{\pmpi} 
{\pmpi} est un middleware relativement r�cent mais qui a attir� l'attention de plusieurs chercheurs
n'ayant aucun lien avec notre communaut� habituelle. 
Ce travail est cit� dans des contextes tr�s divers. Pour citer quelques exemples, 
certains chercheurs utilisent une partie de {\pmpi}, avec une partie des sources de SIP-communicator
pour d�velopper leur propre outil d'�change de messages \cite{Norbisrath08}.
D'autres chercheurs ont ajout� un m�chanisme de s�lection des ressources � {\pmpi} pour en faire leur propre
environnement \cite{p6}. Un dernier exemple est son utilisation dans une application de calcul de flots \cite{p5}. 

