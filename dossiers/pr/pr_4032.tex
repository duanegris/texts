
% PROjet d'enseignement

% Détails responsabilités filière

%   20% 

% Motivations pour l'enseignement

% 3 lettres reco
% copie des contrats

\documentclass[11pt]{article}
%\usepackage{makeidx}
%\usepackage{graphicx}
\usepackage[utf8]{inputenc}
\usepackage[OT1]{fontenc}
\usepackage[francais]{babel}
%\usepackage{a4wide}
\usepackage{fancybox}
%\usepackage{hyperref}
\usepackage{comment}
\usepackage{geometry}
\usepackage{eurosym}
\usepackage{amssymb}
\usepackage[pdftex]{graphicx}
\usepackage{pdfpages}

\usepackage{pgf}
\usepackage{pdfswitch}
\hypersetup{urlcolor=black,linkcolor=black,citecolor=black} % Keep printer friendly
%\geometry{hmargin=3.5cm, bottom=22mm,vmargin=2.5cm }
\geometry{top=23mm,bottom=27mm,left=17mm,right=27mm,headsep=0pt}
\setlength{\parindent}{1cm}


\usepackage[twoside,citeonce(page)]{footbib}
\usepackage{todonotes}

\footbibliography{generalbiblio}
\footbibliographystyle{plain}

\newcommand{\SG}[2][inline]{\todo[color=yellow!50,#1]{\sf \textbf{SG:} #2}}

\newcommand {\burl}[1]{\url{http://icps.u-strasbg.fr/~genaud/courses/#1}}

\newcommand {\rubrique}[1]{
\begin{flushleft}
{\large\bf #1}
\rule{\linewidth}{1mm}
\end{flushleft}
}
\newcommand{\pmpi}{\mbox{\textsc{P2P-MPI}}}


\makeatletter
\def\thebibliography#1{\subsection*{\rule{\linewidth}{1mm}\@mkboth
%\def\thebibliography#1{\subsection*{\rubrique{\large\bf Références biliographiques}\@mkboth
  {PUBLICATIONS}{PUBLICATIONS}}\list
      {[\arabic{enumi}]}{\settowidth\labelwidth{[#1]}\leftmargin\labelwidth
      \advance\leftmargin\labelsep
      \usecounter{enumi}}
      \def\newblock{\hskip .11em plus .33em minus .07em}
      \sloppy\clubpenalty4000\widowpenalty4000
      \sfcode`\.=1000\relax}
    \makeatother


\begin{document}
%------------------------- Page de Garde -------------------
%\setlength{\parskip}{0mm}
\thispagestyle{empty}
\vspace*{\stretch{1}}




\rule{\linewidth}{1mm}
\begin{center}
\Large{\textbf{Dossier de Candidature\\
aux fonctions de \\
Professeur des Universités\\
Poste 4032 - UFR de Mathématiques-Informatique}}\\[5mm]
\Large{Stéphane \textsc{Genaud}}\\[1cm]

\rule{\linewidth}{1mm}
\vspace*{\stretch{2}}\\
\vspace{3cm}
%\DeclareGraphicsExtensions{.jpg}
%\includegraphics[width=5cm]{inria.jpg}
\end{center}
\begin{center}
Date: \today\\
\end{center}

\newpage
\mbox{}%blank page 

\setlength{\parindent}{5mm} %% restore a decnt value
\setlength{\parindent}{0mm}
%\setlength{\itemsep}{1mm}
\newpage


\begin{center}
\huge{\textsc{Table des matières}}
\end{center}
\vspace{2cm}


\tableofcontents

\noindent
%\rule{\linewidth}{0.3mm}


\newpage


%-------- C V -----------------------------------

\section{Curriculum Vit{\ae}}

\setlength{\tabcolsep}{5pt}

\subsection{\'Etat civil}

\medskip

\noindent
\begin{tabular}{llp{\linewidth}}

				%& \rule{\linewidth}{.2mm} \\
\hspace{1cm}	& Prénom et nom :			&\textbf{Stéphane GENAUD}\\
		 	& Date et lieu de naissance :	& Né le 10 mars 1969 à Forbach (57).\\
			& Nationalité :			& Française. \\
			& Situation familiale :		& Marié, 2 enfants. \\
			& Adresse professionnelle :	& LSIIT, Pôle API, Boulevard S. Brant,\\
			&      				& F-67400 Illkirch\\
			&					& Tél. : +33 (0)38 92 77 449\\ 
			&					& Fax  : +33 (0)3 90 24 45 47\\
			& Adresse électronique :	& \texttt{genaud@unistra.fr}\\
			& Page personnelle :		& \texttt{{\url{http://icps.u-strasbg.fr/members/genaud}}}\\[5mm]
				%& \rule{\linewidth}{.2mm} \\

\end{tabular}



\subsection{Formation Universitaire}
\medskip
\noindent
\begin{tabular}{lp{14.8cm}}
		%		& \rule{\linewidth}{.2mm} \\
	\textbf{2009} &  \textbf{Habilitation à diriger des recherches} en informatique \\
			  &	Université Henri Poincaré, Nancy. \textit{Soutenue le 8/12/2009}\\
      		  &	Titre du mémoire : {\em Exécutions de programmes parallèles à passage de messages sur grille de calcul}.\\
			  &	Jury~: 
				\begin{small}
				\begin{itemize}
				\item Pascal Bouvry (PU, Université du Luxembourg), Examinateur,
				\item Christophe Cérin (PU, Université Paris 13), Rapporteur,
				\item Frédéric Desprez (DR, INRIA Grenoble Rhônes-Alpes), Rapporteur,
				\item Claude Godart, (PU, Université Henri Poincaré), Président,
				\item Jens Gustedt (DR, INRIA Nancy Grand-Est), Garant,
				\item Thierry Priol (DR, INRIA Rennes Bretagne-Atlantique), Rapporteur.
				\end{itemize}
				\end{small}\\[2mm]
	

	\textbf{1997} & \textbf{Doctorat en Sciences, mention Informatique}\\
			  & Université Louis Pasteur, Strasbourg.\\
      		  & Titre du mémoire: {\em Transformations de programmes \textsc{Pei} : applications au parallélisme de données}.\\
		        & Jury~:
				\begin{small}
				\begin{itemize}
					\item Luc Bougé (PU, \'{E}NS Lyon), Rapporteur,
			    		\item Christian Lengauer (PU, Université Passau, Allemagne), Examinateur,
					\item Catherine Mongenet (PU, Université Louis Pasteur), Rapporteur interne,
					\item Guy-René Perrin (PU, Université Louis Pasteur), Directeur de thèse,
					\item Patrice Quinton (PU, Université de Rennes 1), Rapporteur.
				\end{itemize}
				\end{small}\\[2mm]
	\textbf{1993} &  \textbf{Diplôme d'\'{E}tudes Supérieures Spécialisées} (DESS) en Informatique du parallélisme, 
				  Université de Franche-Comté, Besançon. Mention {\em très bien} (major).\\[2mm]
	\textbf{1991} &  \textbf{Bachelor of Sciences} (BSc) in European Informatics, Sheffield Hallam University.\\[2mm]
%	\textbf{1989} &  {\bf Diplôme Universitaire de Technologie (DUT)} en informatique, Institut Universitaire de 
%					Technologies de Nantes, Université de Nantes.\\[5mm]
%		%	& \rule{\linewidth}{.2mm} \\

\end{tabular}
\newpage


\subsection{Expérience professionnelle}

\textit{Seules les expériences liées à une activité d'enseignement et/ou de recherche sont décrites ici.}\\[2mm]

\noindent
\begin{tabular}{p{2cm}p{14.3cm}}
%		& \rule{\linewidth}{.2mm} \\
% 	depuis sept. 2009 &  \textbf{Maître de conférences en Informatique} à l'Université de Strasbourg (UdS):\\[-7mm]
%		&\parbox[t]{13.7cm}{%
%            \begin{itemize}
%				  \item[$\rhd$] Enseignant à l'\'Ecole de Management Strasbourg,
%				  \item[$\rhd$] Chercheur au Laboratoire des Sciences de l'Image, de l'Informatique et de la Télédétection (LSIIT, UMR 7005 CNRS-UdS),
%équipe \textit{Image et Calcul Parallèle Scientifique} (ICPS).
%				\end{itemize}
%				\vspace{3mm}
% 				}\\

 	 \textbf{2007--2009}  &  \textbf{Détaché Chargé de Recherche} au Laboratoire Lorrain de Recherche en Informatique et ses Applications
(LORIA, UMR 7503 CNRS-INPL-INRIA-UHP-Nancy 2), équipe projet INRIA \textsc{AlGorille}.\\[2mm]

	\textbf{1998--2007}    &  \textbf{Maître de conférences en Informatique} à l'Université Robert Schuman, Strasbourg:\\[-3mm]
				&\parbox[t]{13.7cm}{%
                          \begin{itemize}
				  \item[$\rhd$] Enseignant à l'IECS, Université Robert Schuman,
				  \item[$\rhd$] Chercheur au LSIIT, UMR 7005 CNRS-ULP, équipe ICPS.
				\end{itemize}
				\vspace{3mm}
 				}\\


	\textbf{1996--1998} &  \textbf{Attaché Temporaire d'Enseignement et de Recherche} au département d'informatique 
de l'IUT de l'université Robert Schuman.\\[2mm]
	\textbf{1994--1996} & Doctorant à l'Université Louis Pasteur (ULP), Strasbourg. \\
				  & Directeur de thèse: Guy-René Perrin.\\[2mm]
	\textbf{1993--1994} & Doctorant à l'Université de Franche-Comté, Besançon. \\
				  & Directeur de thèse: Guy-René Perrin.\\[2mm]
%				& \rule{\linewidth}{.2mm} \\[5mm]

\end{tabular}

\vspace{6mm}
\textbf{\underline{Position administrative actuelle (depuis septembre 2009)}}
\vspace{5mm}

\begin{itemize}
\item
 	\textbf{Maître de conférences en Informatique} à l'Université de Strasbourg (UdS):
	\begin{itemize}
		\item Enseignant à l'\'Ecole de Management Strasbourg,
		\item Chercheur au Laboratoire des Sciences de l'Image, de 
		      l'Informatique et de la Télédétection \\
			(LSIIT, UMR 7005 CNRS-UdS).
équipe \textit{Image et Calcul Parallèle Scientifique} (ICPS).
\\[1mm]
	\end{itemize}

\item Titulaire de la \textbf{Prime d'Excellence Scientifique} depuis septembre 2009.\\

\item \textbf{Qualifié aux fonctions de Professeur des Universités} dans la 
              section 27 en 2010.\\
	\indent {\small Numéro de qualification~:~PR-2010-27-10127207589}\\[2mm]
\end{itemize}


\vspace{-1mm}
\subsection{Responsabilités principales}

\begin{itemize}
\item
 	\textbf{Responsabilités pédagogiques}
	\begin{itemize}
		\item Directeur délégué aux Systèmes d'Information de 
		      l'\'Ecole de Management Strasbourg depuis 07/2011.
		\item Responsable de la filière \textit{Système d'Information} 
		      du master Management International de l'IECS de 2002 à 2007.
	\end{itemize}
 	\textbf{Responsabilités recherche}
	\begin{itemize}
		\item Responsable du thème de recherche \textit{Grilles} dans 
		      l'équipe ICPS du LSIIT de 2001 à 2007. 
			Reprise du thème à mon retour dans l'équipe en septembre 2009.
		\item Membre du conseil scientifique du département \emph{Expertise 
		      pour la recherche de l'UdS} depuis 2010.
	\end{itemize}
 	\textbf{Responsabilités administratives et collectives}
	\begin{itemize}
		\item Membre de 3 commissions de spécialistes entre 2004 et 2008.
		\item Membre du comité d'experts UdS (section 27) depuis 2009 et 
		      membre dans 2 comités de sélection externes.
	      \item Co-responsable des systèmes d'information de mon 
		      établissement depuis 2001.
	\end{itemize}
\end{itemize}




%---------------------------- R E C H E R C H E --------------------------------
\newpage
\section{Activités de recherches}

\paragraph{Résumé}
\textit{Mes thèmes de recherche concernent le \emph{parallélisme}, 
essentiellement sur des architectures de type cluster, \emph{grilles} ou 
\emph{cloud}. Ma thèse de doctorat avait pour objet l'écriture et la 
transformation de programmes parallèles à l'aide d'un langage formel. A la 
suite de cette approche formelle, je me suis investi sur un problème réel dans 
le domaine de la  géophysique, nécessitant la conception et le développement 
d'applications parallèles. Lorsqu'ont émergé les grilles, j'ai étudié certains 
problèmes nouveaux que posaient ces systèmes hétérogènes, et la façon dont des 
applications pouvaient y être déployées. Mon travail sur les grilles a concerné 
l'évaluation et l'amélioration des performances des programmes parallèles dans 
ce contexte, puis l'amélioration des intergiciels pour mieux prendre en charge 
les programmes parallèles. L'utilisation des principes des systèmes pair-à-pair 
pour la découverte et l'auto-organisation des ressources ainsi que des 
mécanismes de tolérance aux pannes par réplication des calculs ont été proposés.
Parallèlement aux évaluations sur des cas et des plate-formes réelles, je 
travaille à la simulation de programmes distribués en contribuant à l'outil 
\textsc{SimGrid}. Je m'intéresse particulièrement à l'extension du simulateur 
pour permettre la simulation de programmes MPI sans modification du code source.
Enfin, mes recherches actuelles sont tournées vers les problèmes d'allocation 
des ressources et d'ordonnancement des tâches, dans le but de proposer aux 
clients de bons compromis performance/prix sur des plates-formes virtualisées 
comme celles fournies par les clouds IaaS.}\\ 

Ces différents aspects sont détaillés ci-après, puis suivent la liste de mes 
publications, mon activité d'animation scientifique, mes participations à des 
encadrements, et mes perspectives de recherches.

\subsection{Détails des thèmes de recherche}


\subsubsection{Spécification de programmes data-parallèles}

Ma thèse de doctorat \cite{icps-1997-4} a porté sur la définition et 
l'utilisation d'un langage formel baptis\'e Pei~\footcite{Violard92}. Ce 
formalisme permet la description de programmes pour des ordinateurs parallèles, 
dans un modèle de programmation de type \emph{parallélisme de données}. Nous 
avons montré comment ce formalisme pouvait être utilisé pour raisonner sur les 
programmes et les transformer en nouveaux programmes sémantiquement 
équivalents ou raffinés \cite{icps-1994-46,icps-1995-1,icps-1997-3,icps-1996-2}.
Le deuxième volet du travail a consisté à proposer des méthodes pour traduire
ces énoncés formels vers des langages parallèles cibles comme 
\textit{High Performance Fortran} ainsi que les logiciels permettant les 
transformations, le contrôle de validité des programmes ainsi que les 
compilateurs pour la traduction%
\footnote{\url{http://icps.u-strasbg.fr/pei/PEI_SUMMARY/langage.htm}}.


\subsubsection{Parallélisation pour la géophysique}

%..............................................................................|80
Après cette expérience d'approche formelle d'un modèle de programmation pour 
le parallélisme, je me suis plongé dans un travail concret de parallélisation 
sur des codes scientifiques. Je me suis investi en particulier dans la 
conception et le développement d'outils logiciels pour la géophysique. L'enjeu 
est de pouvoir calculer une tomographie sismique globale en ondes de volumes 
permettant d'améliorer un modèle de vitesses des ondes sismiques, et de là 
en déduire des propriétés géologiques à l'intérieur de la Terre. L'ensemble 
de ces outils\footnote{\url{http://renass.u-strasbg.fr/ray2mesh}} conçus lors 
d'une thèse en collaboration avec l'Institut de Physique du Globe de Strasbourg 
(UMR CNRS-UdS 7516) permet de combiner et d'enchaîner des traitements sur des 
données géophysiques extrêmement volumineuses. Le travail s'est concrétisé par 
une tomographie utilisant la totalité des données (sismogrammes) acquises 
depuis 1965 par les réseaux de surveillances sismiques à travers le monde. Pour 
répondre à cet objectif les applications ont été conçues pour s'exécuter sur 
des architectures parallèles, et ont été testées sur des configurations très 
différentes allant de la machine parallèle à des réseaux de station de travail 
en passant par des grilles de calcul (cf. paragraphe suivant) à l'échelle 
nationale~\cite{icps-2005-146,icps-2007-184}.  


\subsubsection{Grilles de calcul}

%..............................................................................|80
L'exploitation de telles applications scientifiques pose des problèmes concrets 
quant au choix de la machine cible pour l'exécution. Un type d'architecture 
nouveau a émergé ces dernières années grâce aux technologies qui permettent
de fédérer efficacement des ressources de calcul provenant d'institutions 
différentes. Cette alternative a été popularisée~\footcite{Foster97,Foster98} 
sous l'appellation de \emph{grilles}. J'ai porté en 2001 un projet intitulé 
\textit{Transformations et Adaptations de programmes pour la Grille} (TAG)
accepté dans le cadre de l'Action Concertée Incitative (ACI) Grid du Ministère 
de la Recherche (2002--2005). Le projet visait l'étude du comportement 
d'applications scientifiques, essentiellement des programmes MPI~\footnote{%
Message Passing Interface définit une bibliothèque de fonctions permettant 
à des processus d'échanger des messages. C'est devenu le standard \textit{de 
facto} pour les programmes parallèles exécutés sur des architectures à mémoire 
distribuée.} sur les grilles. Deux catégories de problèmes ont été traitées 
dans ce projet. D'une part, déterminer quelles \emph{performances} on peut 
espérer d'une application exécutée sur des ressources hétérogènes distribuées 
à large échelle géographique et ce qu'on peut faire pour les améliorer.
D'autre part, travailler sur la couche \emph{intergicielle} pour masquer la 
complexité de ces systèmes répartis et améliorer la prise en charge des 
applications.


\paragraph{Performances} 

Pour améliorer les performances, une partie de mon travail a été l'étude de 
techniques d'équilibrage de charge statiques ou dynamiques.  En effet, les 
découpages des données et la distribution de la charge dans de nombreuses 
applications parallèles font l'hypothèse que les processeurs et les réseaux 
sont homogènes. Pour optimiser le temps de calcul d'un ensemble de tâches
indépendantes exécutées sur des processeurs hétérogène, il faut déterminer la 
meilleure quantité de données distribuées à chacune. Nous avons proposé des 
algorithmes pour déterminer statiquement de telles distributions optimales. Ces
algorithmes ont été testés sur de vraies applications, dont l'application de 
géophysique décrite précédemment~\cite{icps-2002-62,icps-2003-75,icps-2004-125}. 
Nous avons également comparé cette approche statique à l'approche dynamique 
dans laquelle le maître distribue un nouveau bloc de travail à un esclave dès 
que celui-ci le demande : l'équilibrage de charge se fait alors naturellement 
selon le rythme de calcul des esclaves. La difficulté dans cette approche est 
de déterminer la taille du bloc à envoyer: en dessous de la taille idéale, il y 
a des aller-retours inutiles, tandis qu'au dessus, les esclaves ne finissent 
pas simultanément. En revanche, cette stratégie possède l'avantage majeur
de s'adapter aux variations de charge.


\paragraph{Intergiciel}

Les travaux précédents se sont largement appuyés sur l'expérimentation réelle
sur des grilles construites avec le logiciel Globus. La prise en charge de nos 
applications par cet intergicel ne correspondant pas à nos attentes, nous avons 
proposé un nouveau type d'intergiciel spécialisé pour MPI. Parmi les 
lacunes observées pour l'exécution des programmes parallèles figurent l'absence 
de mécanisme de co-allocation de ressources sur différents sites, la détection 
de la disponibilité des ressources au moment précis de l'exécution, la gestion 
de binaires multiples pour chaque systèmes, la difficulté d'accéder aux fichiers
de données et programmes, et l'absence de détection des pannes et de tolérance 
aux pannes. Nous avons développé {\pmpi}\footnote{~\url{http://www.p2pmpi.org}},
pour pallier ces lacunes. {\pmpi} comprend à la fois la couche intergicielle et 
la bibliothèque de communication permettant de développer des programmes 
parallèles à passage de messages. Cette dernière est une implémentation de 
MPJ\footnote{Message Passing for Java: adaptation de MPI pour Java}qui intègre 
la notion de tolérance aux pannes. Les applications sont des programmes Java, 
beaucoup plus faciles à déployer dans un environnement hétérogène.
Pour la couche intergicielle, nous reprenons les principes des systèmes 
pair-à-pair: chaque machine démarrée avec {\pmpi} devient un pair susceptible 
de partager son CPU ou d'utiliser ceux des autres. Cette approche confère 
autonomie et robustesse aux applications. L'autonomie provient de la 
possibilité de découvrir dynamiquement un ensemble de pairs disponibles à ce 
moment précis pour exécuter un programme parallèle. On construit ainsi 
dynamiquement une ``plate-forme'' à chaque demande d'exécution. La façon de 
choisir les pairs les plus adaptés parmi ceux disponibles dépend de plusieurs 
critères, dont la latence réseau qui sépare la machine qui fait la requête des 
pairs candidats, la présence ou non des données nécessaires dans les caches des 
pairs distants, et le souhait de l'utilisateur de concentrer ou non les 
processus sur le minimum de machines. 

Dans de tels systèmes, les pannes sont fréquentes. Or, lors d'un calcul, la 
panne d'un des participants provoque l'arrêt de l'application. Pour diminuer le 
risque de panne, nous avons proposé une solution jamais expérimentée dans ce 
contexte qui est la réplication des calculs. L'utilisateur décide du taux de 
redondance de l'exécution de chaque processus, sur des machines différentes, et 
le système gère de manière transparente la cohérence de l'exécution. En cas de 
panne de l'un des processus, l'application peut poursuivre son exécution tant 
qu'il subsiste au moins une copie de ce processus de calcul.\\


La conception de {\pmpi} à été décrite dans~\cite{icps-2007-182,icps-2005-155}.
Nous avons aussi démontré sa capacité à prendre en charge l'exécution de 
programmes parallèles sur plusieurs centaines de processeurs~%
\cite{icps-2008-193}. La thèse de Choopan Rattanapoka~\footcite{icps-2008-208} 
présente l'ensemble des résultats. {\pmpi} est aussi un support pour l'étude de 
la tolérance aux pannes. Nous avons étudié le mécanisme de réplication des 
calculs que nous proposons~\cite{icps-2007-185} en montrant comment déterminer 
un taux optimal de réplication~\cite{icps-2009-217} puis en faisant une étude 
quantitative du coût de la réplication et des temps de reprise~%
\cite{icps-2009-214}.

Enfin, je me suis attaché à valider notre proposition sur de vraies 
applications.Deux collaborations ont abouties dans ce cadre. En 2007 et 2008 
nous avons aidé des collègues du LSIIT (Pierre Gançarski), dans le domaine de 
la fouille de données, à paralléliser leur méthode d'apprentissage non-%
supervisée pour le clustering~\cite{icps-2008-188}. En 2008 et 2009, nous avons 
collaboré avec des collègues de SUPELEC (Virginie Galtier et Stéphane Vialle) 
pour comparer les implantations parallèles de la méthode d'apprentissage 
Adaboost dans deux modèles de programmation différents (JavaSpace et MPJ)~%
\cite{icps-2009-219}.


\subsubsection{Simulation}
\label{sc:simulation}

Les expérimentations que nous avons menées dans notre travail sur les grilles 
ont été fastidieuses. L'apparition de l'outil Grid'5000 a considérablement
élargi les possibilités d'expérimentation dans un environnement réel, tout
en permettant un protocole expérimental plus rigoureux car l'utilisateur
peut sélectionner le matériel voulu, et installer le système d'exploitation de
son choix. Le caractère reproductible des expériences a donc été considérablement
amélioré avec Grid'5000, mais reste néanmoins imparfait, car le
réseau reliant les sites ainsi que les clusters sont régulièrement renouvelés.
Les expériences doivent donc se succéder sur une période relativement courte
pour obtenir une série de résultats comparables. Cependant, les problèmes 
techniques ou la difficulté d'obtenir l'accès à l'outil au moment souhaité 
peuvent fortement allonger la période d'expérimentation.
La simulation présente face à ce problème un grand intérêt. Elle peut permettre
de tester de nombreux scénarios, et de ne faire des expériences réelles que pour
les cas les plus intéressants.\\

\textsc{SimGrid}~\footcite{Casanova08} est un projet important dans le paysage 
de la recherche académique sur la simulation des systèmes distribués. Le 
logiciel permet de décrire l'ensemble des ordinateurs connectés et le réseau 
les interconnectant, ainsi que les opérations de calcul et de communication 
survenant au sein d'une application, et d'en faire une simulation à évènements 
discrets. Le déroulement d'une application est décrit à travers une 
\emph{interface} au simulateur, c'est-à-dire une API fournie pour décrire ces 
opérations de calcul et de communication.\\

Ce logiciel a été soutenu entre autres par le projets ANR USS-SimGrid%
\footnote{\url{http://uss-simgrid.gforge.inria.fr/}} 
\footnote{Labelisé projet \textit{phare} par l'ANR.}
(2009-2011), puis aujourd'hui par le projet SONGS%
\footnote{\url{http://infra-songs.gforge.inria.fr/}} 
(2012-2015). Je participe à ces projets, dont l'objectif est d'étendre les 
capacité de l'outil. Il n'existait pas par exemple, avant le projet Uss-Simgrid,  
d'interface permettant de simuler les programmes parallèles à passage de 
messages (ses deux principales API proposaient alors des communications point à 
point bloquantes). Nous avons proposé une interface pour simuler des programmes 
MPI. J'ai réactivé un effort fait dans ce sens à l'université d'Hawaï (Mark 
Stillwell et Henri Casanova) mais inachevé, baptisée SMPI. Ce travail, poursuivi 
par Pierre-Nicolas Clauss a abouti à une version désormais livrée avec SimGrid. 
La conception et l'évaluation de SMPI ont fait l'objet d'une publication à la 
conférence IPDPS~\cite{icps-2011-224}. Plusieurs perspectives sont ouvertes avec 
cet outil. On peut extrapoler l'exécution d'un programme sur une machine qui 
n'existe pas encore, dont on ne donne que la description. Ceci peut servir par 
exemple à des fins de dimensionnement d'un cluster. Cela peut être utile dans 
de nombreuses autres situations, comme l'enseignement, où une machine de bureau 
peut servir à simuler un clusterou un système distribué. On peut également 
imaginer dans le futur continuer des exécutions en simulation après avoir 
capturé la trace d'un programme réellement exécuté (avec des outils de 
profiling existants) et en injectant cette trace dans le simulateur.


%------------- P R O J E T    R E C H E R C H E S --------------------
\subsection{Programme de recherches au sein du LSIIT}

\textit{%
Mon projet pour la recherche s'inscrit naturellement dans la continuité et 
l'évolution de mon activité actuelle au Laboratoire des Sciences de l'Image,
de l'Informatique et de la Télé-détection (LSIIT), dans l'équipe Image et
Calcul Parallèle Scientifique (ICPS). Je souhaite développer mes activités 
de recherche dans ce laboratoire, et à partir de  2013 dans le laboratoire 
Icube qui succèdera à l'actuel LSIIT.}\\

Ma stratégie de développement comporte deux axes : un renforcement de 
l'activité du thème \textit{applications} de l'équipe, et une extension
du thème \textit{grilles} aux environnements de calcul basés sur 
l'externalisation des ressources, c'est-à-dire les \emph{clouds}. Ces deux
axes sont complémentaires.


\subsection*{Thème Applications}
Les activités de l'équipe ICPS sont structurées en trois thèmes: 
\begin{enumerate}
\item la \textit{compilation et l'optimisation de programmes}, qui 
cherche à exhiber du parallélisme qu'on peut mettre en {\oe}uvre à 
l'intérieur d'une machine (par exemple sur des multi-c{\oe}urs),
\item l'\textit{adaptation des programmes pour les grilles}, qui 
étudie la mise en {\oe}uvre d'applications quand le parallélisme mobilise 
plusieurs machines interconnectées, potentiellement hétérogènes (par 
exemple une fédération de clusters à l'échelle nationale ou internationale), 
\item les \textit{applications}, qui à travers l'étude de cas 
applicatifs concrets dans le domaine scientifique, examine 
l'apport du parallélisme pour améliorer les résultats scientifiques
envisageables par simulation  numérique.
\end{enumerate}

Ce dernier thème permet à l'ICPS d'établir des collaborations multi-%
disciplinaires et apporte des problématiques réelles aux deux autres 
thèmes. Ces problèmes sont variés dans leur forme mais sont tous
relatifs à la simulation numérique.
La simulation numérique est un besoin crucial dans de nombreuses 
disciplines scientifiques. L'augmentation théorique des capacités
de calcul des ordinateurs ne répond que très partiellement aux
exigences croissantes des utilisateurs. Ces derniers peuvent par
exemple vouloir ajouter une dimension au problème, accroître la 
résolution, ou simuler sur une période plus longue, etc. Répondre 
à ces exigences demande d'exploiter au mieux les dispositifs 
matériels à disposition dans le domaine du calcul intensif.
Cette tâche est difficile étant donné la diversité et la complexité
du matériel et des couches système. On est aujourd'hui réduit à examiner 
une application comme un cas particulier, pour lequel on ne sait souvent
pas prédire quelles configurations matérielles et logicielles vont 
permettrent les meilleures performances : on doit évaluer l'impact de
différent modèles de programmation, du nombre de c{\oe}urs par CPU,
du réseau d'interconnexion entre CPU et mémoire, évaluer l'opportunité
d'utiliser des GPU, faire de la vectorisation (SSE), etc.\\

%-----------------------------------------------------------------------------|80
Les exigences des applications de calcul intensif sont un moteur indéniable
pour la recherche en parallélisme. Les problèmes posés déclenchent une 
réflexion sur les modèles de programmation, sur les outils automatiques
d'analyse de code et les méthode de compilation, sur les intergiciels 
pour coordonner l'exécution distribuée des processus ou la gestion des pannes, 
sur l'architecture, sur les méthodes numériques. Les deux derniers domaines ne
 sont pas à proprement parler dans notre champ de compétence mais doivent
être maîtrisés, par exemple à travers des collaborations avec des 
spécialistes. Le travail que j'ai mené pour des applications en géophysique 
m'a ainsi beaucoup appris. Le code développé a souvent servi d'exemple dans
la communauté travaillant sur Grid'5000 pour évaluer des solutions proposées
pour des environnements de type grilles~\footcite{Cappello07}.  \\

Aujourd'hui, je compte poursuivre ce travail interdisciplinaire, à travers 
une collaboration avec Guillaume Latu, ancien membre de l'équipe ICPS, 
maintenant employé au CEA Cadarache. Nous co-encadrons depuis février 2011 un 
doctorant, Matthieu Kuhn, avec Nicolas Crouseille chargé de recherches en 
mathématiques appliquées dans l'équipe CalVi\footnote{L'équipe-projet CalVi
est bi-localisée entre Strasbourg et Nancy.  Nos collaborateurs directs font 
partie du laboratoire de mathématiques IRMA (UMR 7501 CNRS-UdS).}, pour 
paralléliser un code de physique de plasmas de bords. Ce travail renforce une 
collaboration de longue date (depuis 2002) entre l'ICPS et CalVi. Cette 
collaboration a été à l'origine de notre association à la proposition de LabEx 
avec l'IRMA. Ce LabEx a été obtenu en février 2012, et va donc permettre de 
renforcer de manière significative ces travaux interdisciplinaires, notamment 
à travers des thèses co-encadrées. Nous souhaitons également recruter un 
permanent sur ce thème. Un poste de CR2 CNRS ayant été fléché pour recruter
un informaticien dans un laboratoire de mathématiques, nous avons travaillé
en concertation avec l'IRMA pour porter un dossier de candidature calibré 
pour ces besoins à ce concours.\\  

%-----------------------------------------------------------------------------|80

Je compte également démarrer de nouvelles collaborations, comme celle avec 
l'institut pluridisciplinaire Hubert Curien (UMR 7178, CNRS-IN2P3-INC-INEE, UdS) 
dans le domaine de la protéomique (Alain Van Dorsselaer). Après une première 
analyse des besoins en protéomique, nous pensons devoir concentrer la plupart 
de nos efforts sur le couplage entre calculs et données, avec le plus souvent 
des calculs très nombreux mais indépendants, et utilisant de grandes bases de 
données dont le contenu est mis à jour presque quotidiennement. Les 
environnements de grille sont appropriés à ce type de travail, étant donné les
capacités de calcul qu'offrent l'organisation virtuelle BioMed (sous-ensemble
des ressources de la grille européenne EGI). Cette collaboration est importante
pour les protéomistes Strasbourgeois qui ont obtenu la responsabilité de
co-développer (avec deux autres centres) la plate-forme ProFI\footnote{\url{%
http://bit.ly/eXXCOR}}, sélectionnée dans l'appel à projets ``Infrastructures 
nationales de recherche en  biologie''. Trois étudiants sont en stage depuis
février 2012 pour apporter des améliorations à la plate-forme développée à
l'IPHC~\cite{iphc-2011} ou faire un état de l'art des technologies de cloud
et tester leur introduction dans cet environnement de production.
 



\subsection*{Thème Grilles et Clouds}

A mon retour au LSIIT en septembre 2009, j'ai infléchi l'orientation de 
mes recherches vers les problématiques de \emph{cloud} qui me semblent 
incontournables dans le domaine. En comparaison des grilles, par nature
hétérogènes, les clouds proposent des ressources de calcul provenant
de clusters hébergés dans des \textit{data centers}, dont on peut décider 
du type (CPU, quantité mémoire).  De plus, la virtualisation (cloud IaaS) 
permet de rendre également la configuration logicielle homogène, en 
transportant tout l'environnement logiciel dans la machine virtuelle exécutée 
chez le fournisseur de ressources. C'est donc un nouveau type d'architecture, 
entre le cluster homogène et la grille hétérogène, qui s'offre à nous. 
La disparition de l'hétérogénéité est un élément capital pour le succès
de ce type de solution. On peut ainsi simplifier considérablement les 
intergiciels comparativement à ceux utilisés traditionnellement dans les 
grilles. Cependant, les problématiques fondamentales des grilles, à
savoir l'allocation des ressources, l'ordonnancement des calculs et des
données, la tolérance aux pannes, demeurent mais doivent être revues 
dans ce contexte.
D'un point de vue économique, les fournisseurs de cloud proposent 
généralement une facturation proportionnelle au temps d'utilisation 
uniquement, ce qui rend cette solution très attractive car supprimant
totalement le coût de possession des matériels. Les économies d'échelle
réalisées grâce à la virtualisation et à la taille des \textit{data centers}
permettent par ailleurs aux fournisseurs de proposer des prix très 
compétitifs.  On assiste aujourd'hui à une multitude de projets visant 
à évaluer la viabilité de cette approche, en particulier dans le calcul 
scientifique. Par exemple, dès 2009, le département de l'énergie américain 
a (DOE) décidé de financer le projet Magellan%
\footnote{\url{http://www.nersc.gov/nusers/systems/magellan/}},
infrastructure de 5000 coeurs exploitée sous la forme d'un cloud
ouvert à 3000 scientifiques pour tester un large éventail 
d'applications scientifiques. Je crois qu'une expérimentation comparable
serait profitable pour le méso-centre de calcul de l'université de
Strasbourg, amené à jouer un rôle important après le succès de sa 
proposition d'Equipex. Nos relations privilégiées avec ce centre (Romaric 
David) encouragent un tel projet.\\

Faire émerger un thème de recherche nécessite cependant un investissement 
sur plusieurs années. Avec Julien Gossa, maître de conférences, nous avons 
initié de nombreuses démarches pour nous établir dans notre environnement 
local et national, afin d'être en position de travailler à des projets
de visibilité internationale. 

Au niveau local, la présence de ce thème dans nos enseignements a permis de 
recruter un étudiant de master en stage de recherche, puis en thèse en
octobre 2011 sur un financement DGA.

Au niveau national, nous avons contribué à la proposition de projet ANR 
baptisée SONGS, continuant le projet USS-SimGrid (voir section~%
\ref{sc:simulation}). Notre proposition a été acceptée et SONGS a démarré 
début 2012. Ce projet qui implique 19 chercheurs sur 7 sites est de type 
\emph{plateforme} au sens ANR, c'est-à-dire qu'il implique déjà une large 
communauté de chercheurs et d'utilisateurs très actifs, il est ouvert et 
son fonctionnement est durable. Le projet est aidé à hauteur de 1,8 M\euro{}.
L'objectif est d'étendre et d'améliorer les capacités de l'outil de 
simulation SimGrid en prenant en compte l'évolution des architectures 
distribuées. Parmi les domaines visés, on compte les plates-formes de 
\emph{volunteer computing}, les grilles de production, le HPC, et les 
clouds IaaS. Dans ce projet, je suis pour le LSIIT le responsable 
scientifique (porteur local) d'un work package sur la simulation des 
clouds.


\subsection*{Complémentarité des thèmes}

%------------------------------------------------------------------------------|80
La diversité des besoins en terme de calcul, face à la diversité croissante des
architectures de calcul parallèle ou distribué, m'incite à mener de front des 
états de l'art sur ces deux thèmes. Je crois que l'équipe peut contribuer sur 
deux aspects: la performance des applications et la gestion des ressources.\\

Nous avons parlé de la performance dans le thème applicatif, qui nécessite une 
coopération étroite entre les chercheurs du domaine et les chercheurs 
expérimentés en parallélisme, ces derniers pouvant apporter des méthodes 
applicables directement au problème. Parmi ces méthodes, du ressort de la 
recherche en informatique, on compte l'analyse (par exemple de trace, de 
complexité, la visualisation), la parallélisation et l'optimisation automatique, 
l'expérimentation et la simulation.\\


La gestion des ressources nécessite de construire des environnements logiciels 
de type intergiciel, capables de recenser, sélectionner, configurer des 
ressources adaptées pour une application à un moment donné. Nous avons évoqué 
l'évolution des équipements de calcul vers plus de complexité. Les machines 
individuelles voient leur puissance de calcul continuer de croître grâce à 
l'augmentation du nombre de c{\oe}urs, tandis que l'accès à des équipements 
de calcul de type cluster est facilité car ils sont de plus en plus nombreux~%
\footcite{Wu09} avec une  meilleure connectivité réseau. Certaines applications 
pourront être avantageusement prises en charge par une seule machine multi-%
c{\oe}urs, tandis que d'autres tireront plus de profit d'un petit laps de temps,  
éventuellement en mode \textit{best-effort}, accordé sur un cluster avec beaucoup 
de processeurs. Enfin, la nouvelle voie ouverte par les clouds IaaS permet 
d'envisager d'embarquer l'application  dans une image de machine virtuelle 
pour l'exécuter sur des ressources de calcul externalisées. \\

%------------------------------------------------------------------------------|80
Choisir la bonne plate-forme d'exécution est donc un problème à part entière.
Cette problématique est au centre de mon projet de recherches au LSIIT. De 
manière générale, je souhaite étudier les capacités des différents types de 
plates-formes, en intégrant, quand c'est possible, la dimension économique.

Optimiser l'exploitation des ressources revêt plusieurs formes. J'ai commencé à
travailler dans le cadre du projet ANR SPADES (2009-2011)%
\footnote{\url{http://graal.ens-lyon.fr/SPADES/SPADES}}  
sur un système permettant d'exploiter des tranches de temps réduites qui 
peuvent apparaître de manière fugace sur des clusters. L'exploitation de ces 
tranches se faisant en mode non-prioritaire (ou \textit{best-effort}) on 
pourrait ainsi augmenter le taux d'utilisation de ces clusters. Pour atteindre 
cet objectif, il faut concevoir des intergiciels extrêmement réactifs, capables 
de localiser rapidement sur un ensemble de clusters, les plages d'utilisation 
disponibles, ainsi que la compatibilité du système avec les besoins de 
l'application. Un deuxième travail démarré dans ce sens est l'étude de 
stratégies d'allocation puis d'ordonnancement \textit{online} de jobs 
séquentiels sur une ou plusieurs plates-formes de type cloud IaaS. 
L'utilisation des ressources dans ces centres étant facturée par tranche de 
temps (typiquement toute heure entamée est due), de nombreux jobs laissent des 
périodes de temps facturées mais non utilisées, qui peuvent donc être utilisées 
gratuitement pour d'autres jobs. Une stratégie d'allocation doit décider pour 
chaque job si une nouvelle machine virtuelle doit être démarrée pour ce 
traitement ou si l'on peut réutiliser une machine virtuelle déjà démarrée. On 
étudie dans ce contexte un problème bi-critère de performance incluant le temps 
d'attente des jobs et le coût de l'ensemble des jobs. Ces résultats ont été 
présentés à la dernière édition de la conférence 
\href{http://www.thecloudcomputing.org/2011/}{CLOUD 2011}~\cite{icps-2011-225}.
L'objectif à plus long terme est de construire un système de courtage permettant 
de sélectionner, éventuellement auprès de différents fournisseurs, des machines 
virtuelles dont les capacités sont les mieux adaptées au besoin.


\subsection*{Conclusion}

%------------------------------------------------------------------------------|80
L'équipe ICPS souhaite maintenir l'équilibre dans le développement de ses
différents thèmes. Je suis convaincu qu'il faut couvrir tout le spectre du
parallélisme pour avoir une bonne vision des solutions les plus appropriées
aux différents types de besoins. Cette strategie me semble particulièrement
importante dans un contexte où les technologies changent très vite. Certaines
applications qui nécessitaient un petit cluster il y a cinq ans peuvent 
utiliserune machine multi-c{\oe}urs aujourd'hui. Ce changement d'architecture 
cible peut induire un changement quant au modèle de programmation ainsi que 
des outils d'optimisation utilisables. C'est pourquoi il me semble important
d'avoir au sein de l'équipe l'ensemble des visions et des compétences du
parallélisme. J'ai par exemple entamé un travail~\cite{ketterlin11} avec 
Alain Ketterlin, sur l'abstraction des schémas de communications des 
programmes MPI. Ce travail basé sur la reconnaissance de répétitions 
linéaires s'appuye sur des résultats obtenus initialement pour l'analyse des 
accès mémoire, menés dans le thème compilation et optimisation de programmes.

En conclusion, je pense qu'un poste de professeur dans l'équipe ICPS amènerait
une accélération du développement du thèmes grilles et clouds, ce statut
ayant une plus grande reconnaissance au pan national ou international. 
Je souhaite également contribuer au développement du thème applications,
avec l'espoir de pouvoir recruter un spécialiste à l'interface dans le 
moyen terme. Je pense pouvoir faire  béneficier le laboratoire et l'équipe, de 
ma connaissance de l'environnement local, importante pour les collaborations en 
cours ou à venir, et de mon intégration dans la communauté de recherche des
grilles ou du HPC, avec en particulier le travail fait dans le projet SONGS
amené à avoir une grosse visibilité internationale.


%------------------------ P U B L I C A T I O N S -----------------------------|
\newpage

\subsection{Liste de publications}


\small
\bibliographystyle{plain}
\begin{thebibliography}{99}

\subsection*{Thèses}

\bibitem{hdr}
\textbf{Stéphane Genaud}.
\newblock 
{\em Exécutions de programmes parallèles à passage de messages sur grille de 
calcul}.
\newblock 
Habilitation à diriger des recherches de l'université Henri Poincaré, 
Nancy. Décembre 2009.
\newblock 
Rapporteurs : C. Cérin (Paris 13), F. Desprez (INRIA Rhônes-Alpes), 
T. Priol (INRIA Bretagne-Atlantique).\\[2mm]

\bibitem{icps-1997-4}
\textbf{Stéphane Genaud}.
\newblock {\em Transformations de programmes \textsc{Pei} : applications au
  parallélisme de données}.
\newblock Thèse de doctorat de l'université Louis Pasteur, Strasbourg, Janvier 1997.
\newblock Rapporteurs : Luc Bougé et Patrice Quinton. 

\subsection*{Chapitre de livre}

\bibitem{icps-book}
\textbf{Stéphane Genaud} et Choopan Rattanapoka.
\newblock 
\emph{A Peer-to-Peer Framework for Message Passing Parallel Programs.}
\newblock 
Parallel Programming and Applications in Grid, P2P and Network-based System,
in {\em Advances In Parallel Computing} Series. Editor G. R. Joubert.
IOS Press, juin 2009. 
 

\subsection*{Articles en revues internationales}

\setlength{\itemsep}{1.5mm}


\bibitem{icps-2009-217}
\newblock \textbf{Stéphane Genaud}, Emmanuel Jeannot et Choopan Rattanapoka.
\newblock Fault-Management in P2P-MPI.
\newblock {\em International Journal of Parallel Programming}, Springer, 
37(5):433--461, août 2009.


\bibitem{icps-2008-188}
\textbf{Stéphane Genaud}, Pierre Gançarski, Guillaume Latu, Alexandre Blansché, 
Choopan Rattanapoka et Damien Vouriot. \newblock Exploitation of a parallel 
clustering algorithm on commodity hardware with P2P-MPI.
\newblock 
{\em The Journal of SuperComputing}, Springer, 43(1):21--41, jan. 2008.


\bibitem{icps-2007-182}
\textbf{Stéphane Genaud} et Choopan Rattanapoka.
\newblock P2P-MPI: A Peer-to-Peer Framework for Robust Execution of Message 
Passing Parallel Programs on Grids.
\newblock {\em Journal of Grid Computing}, Springer, 5(1):27--42, mai 2007.


\bibitem{icps-2004-125}
\textbf{Stéphane Genaud}, Arnaud Giersch, et Frédéric Vivien.
\newblock Load-balancing scatter operations for Grid computing.
\newblock {\em Parallel Computing}, Elsevier, 30(8):923--946, août 2004.

\bibitem{icps-2004-107}
Marc Grunberg, \textbf{Stéphane Genaud} et Catherine Mongenet.
\newblock Seismic ray-tracing and Earth mesh modeling on various parallel
  architectures.
\newblock 
{\em The Journal of Supercomputing}, Kluwer, 29(1):27--44, juillet 2004.


\subsection*{Articles en revues nationales}
\bibitem{icps-2005-146}
\textbf{Stéphane Genaud} et Marc Grunberg. 
\newblock  Calcul de rais en tomographie sismique : exploitation sur la grille.
\newblock {\em Technique et Science Informatiques}, numéro spécial Renpar, 
Hermès-Lavoisier, 24(5), pages 591--608, décembre 2005.

\bibitem{icps-1996-2}
\textbf{Stéphane Genaud}.
\newblock Transformations d'énoncés \textsc{Pei}.
\newblock {\em Technique et Science Informatiques}, 15(5), pages 601--618, 
Hermès, avril 1996.

\bibitem{icps-1993-69}
\textbf{Stéphane Genaud} et Guy-René Perrin.
\newblock Une expérience d'implantation d'un algorithme systolique sur
  hypercube.
\newblock {\em La Lettre du Transputer et des calculateurs parallèles},
  (17), mars 1993.


\subsection*{Conférences internationales avec actes et comité de lecture}

\bibitem{icps-2011-225}
\newblock \textbf{Stéphane Genaud} et Julien Gossa,
\newblock Cost-wait Trade-offs in Client-side Resource Provisioning with 
Elastic Clouds.
\newblock {\em 4th IEEE International Conference on Cloud Computing (CLOUD 
2011)}, juillet 2011.
\newblock \small{\textit{(papiers acceptés/soumis{:}36/198, taux: 18\%)}}


\bibitem{icps-2011-224}
\newblock Pierre-Nicolas Clauss, Mark Stillwell, \textbf{Stéphane Genaud}, 
Fr\'ed\'eric Suter, Henri Casanova and  Martin Quinson.
\newblock Single Node On-Line Simulation of MPI Applications with SMPI.
\newblock {\em 25th IEEE International Parallel \& Distributed Processing 
Symposium (IPDPS 2011)}, mai 2011.
\newblock \small{\textit{(papiers acceptés/soumis{:}112/571, taux: 19\%)}}

\bibitem{icps-2009-219}
\newblock Virginie Galtier, \textbf{Stéphane Genaud} et Stéphane Vialle.
\newblock Implementation of the AdaBoost Algorithm for Large Scale Distributed 
Environments: Comparing JavaSpace and MPJ.
\newblock {\em International Conference on Parallel and Distributed Systems}, 
IEEE, déc. 2009.
\newblock \small{\textit{(papiers acceptés/soumis:91/305, taux: 29\%)}}


\bibitem{icps-2009-214}
\textbf{Stéphane Genaud} and Choopan Rattanapoka.
\newblock Evaluation of Replication and Fault Detection in P2P-MPI.
\newblock 
{\em 6th IEEE International Workshop on Grid Computing (HPGC), IPDPS 2009}, 
mai 2009.
\newblock \textit{(Papier invité)}.

\bibitem{icps-2008-193}
\textbf{Stéphane Genaud} and Choopan Rattanapoka. 
\newblock Large-Scale Experiment of Co-allocation Strategies for Peer-to-Peer 
Supercomputing in P2P-MPI,
\newblock 
{\em 5th IEEE International Workshop on Grid Computing (HPGC), IPDPS 2008}, 
avril 2008.

\bibitem{icps-2007-192}
Ludovic Hablot and Olivier Glück and Jean-Christophe Mignot and \textbf{Stéphane Genaud} and Pascale Vicat-Blanc Primet.
\newblock Comparison and tuning of MPI implementation in a grid context.
\newblock {\em Proceedings of 2007 IEEE International Conference on Cluster Computing (CLUSTER)}, 458--463, september 2007.
\newblock \small{\textit{(papiers acceptés/soumis:42/106, taux: 39\%)}}

\bibitem{icps-2007-185}
\newblock \textbf{Stéphane Genaud} et Choopan Rattanapoka.
\newblock Fault Management in {\pmpi}. 
\newblock International Conference on {\em Grid and Pervasive Computing, 
(GPC 2007)}, LNCS, Springer, mai 2007.
\newblock \small{\textit{(papiers acceptés/soumis:56/217, taux: 25\%)}}
%submitted papers: 217; accepted : 56 full papers, 12 oral-short papers;  acceptance rate: 25\%

\bibitem{icps-2007-184}
\newblock \textbf{Stéphane Genaud}, Marc Grunberg et Catherine Mongenet.
\newblock Experiments in running a scientific {MPI} Application on GRID'5000. 
\newblock distingué par le \textsc{Intel} \textit{best paper award}.
\newblock {\em 4th IEEE International Workshop on Grid Computing (HPGC), IPDPS 
2007}, mars 2007.


\bibitem{icps-2005-155}
\textbf{Stéphane Genaud} et Choopan Rattanapoka.
\newblock A Peer-to-peer Framework for Robust Execution of Message Passing 
Parallel Programs.
\newblock 
In {\em EuroPVM/MPI 2005}, LNCS 3666, Springer-Verlag, pages 276--284, 
septembre 2005.
\newblock \small{\textit{(papiers acceptés/soumis:61/126, taux: 48\%)}}


\bibitem{icps-2004-124}
Marc Grunberg, \textbf{Stéphane Genaud}, et Catherine Mongenet.
\newblock Parallel adaptive mesh coarsening for seismic tomography.
\newblock In {\em SBAC-PAD 2004, 16th Symposium on Computer Architecture and
  High Performance Computing}. IEEE Computer Society Press, octobre 2004.
\newblock \small{\textit{(papiers acceptés/soumis:32/93, taux: 34\%)}}

\bibitem{icps-2003-75}
\textbf{Stéphane Genaud}, Arnaud Giersch, et Frédéric Vivien.
\newblock Load-balancing scatter operations for Grid computing.
\newblock In {\em Proceedings of 12th Heterogeneous Computing Workshop 
(HCW), IPDPS 2003}. IEEE Computer Society Press, avril 2003.

\bibitem{icps-2002-62}
Romaric David, \textbf{Stéphane Genaud}, Arnaud Giersch, \'{E}ric Violard, et 
  Benjamin Schwarz.
\newblock Source-code transformations strategies to load-balance Grid
  applications.
\newblock In {\em International Conference on Grid Computing - GRID'2002}, 
LNCS 2536, pages 82--87. Springer-Verlag, novembre 2002.

\bibitem{icps-2002-20}
Marc Grunberg, \textbf{Stéphane Genaud}, et Catherine Mongenet.
\newblock Parallel seismic ray-tracing in a global {E}arth mesh.
\newblock In {\em Proceedings of the 2002 Parallel and Distributed Processing
  Techniques and Applications (PDPTA'02)}, pages 1151--1157, juin 2002.

\bibitem{icps-1997-3}
Eric Violard, \textbf{Stéphane Genaud} et Guy-René Perrin.
\newblock Refinement of data-parallel programs in pei.
\newblock In {\em IFIP Working Conference on Algorithmic Language and Calculi}, 
R.~Bird and L.~Meertens editors, Chapman \& Hall~Ed., février 1997.
\newblock 25 pages.

\bibitem{icps-1995-1}
\textbf{Stéphane Genaud}, Eric Violard, et Guy-René Perrin.
\newblock Transformation techniques in \textsc{Pei}.
\newblock In P.~Magnusson S.~Haridi, K.~Ali, editor, {\em Europar'95}, LNCS
  966, pages 131--142. Springer-Verlag, août 1995.
\newblock \small{\textit{(papiers acceptés/soumis:50/180, taux: 27\%)}}




\subsection*{Conférences nationales avec actes et comité de lecture}
\bibitem{icps-2003-111}
Marc Grunberg et \textbf{Stéphane Genaud}.
\newblock Calcul de rais en tomographie sismique : exploitation sur la grille.
\newblock In {\em Renpar2003}, pages 179--186. INRIA, octobre 2003.

\bibitem{icps-1995-6}
\textbf{Stéphane Genaud}.
\newblock Techniques de tranformations d'énoncés \textsc{Pei} pour la
  production de programmes data-parallèles.
\newblock In {\em RenPar 7}, mai 1995, Mons, Belgique.

\bibitem{icps-1994-46}
Guy-René Perrin, Eric Violard et \textbf{Stéphane Genaud}.
\newblock \textsc{Pei} : a theoretical framework for data-parallel programming.
\newblock In {\em Workshop on Data-Parallel Languages and Compilers}, Lille, 
mai 1994.
\vspace{3mm}


\subsection*{Autres communications}

\bibitem{iphc-2011}
Christine Carapito, Jérôme Pansanel, Patrick Guterl, Alexandre Burel, Fabrice 
Bertile, \textbf{Stéphane Genaud}, Alain Van Dorsselaer, Christelle Roy.
\newblock Une suite logicielle pour la protéomique interfacée sur une grille de 
calcul. Utilisation d'algorithmes libres pour l'identification MS/MS, le 
séquençage de novo et l'annotation fonctionnelle.
\newblock Rencontres Scientifiques France Grilles 2011, Lyon.


\bibitem{ketterlin11}
Alain Ketterlin, \textbf{Stéphane Genaud}, Matthieu Kuhn.
\newblock Loop-Nest Recognition for the Extraction of Communication Patterns 
and the Compression of Message-Passing Parallel Traces.
\newblock Research Report ICPS 11-01. Université de Strasbourg. déc. 2011.


\bibitem{cds-2005}
A. Schaaff, F. Bonnarel, J.-J. Claudon, R. David, \textbf{S. Genaud}, M. Louys, 
C. Pestel and C. Wolf.
\newblock Work around distributed image processing and workflow management, 
\newblock poster à ADASS 2005, Madrid.


\bibitem{icps-2003-113}
Marc Grunberg, \textbf{Stéphane Genaud}, et Michel Granet.
\newblock Geographical {ISC} data characterization with parallel ray-tracing.
\newblock In {\em Eos Trans. AGU, 84(46), Fall-Meeting Suppl., Abstract
  S31E-0793}, décembre 2003.

\bibitem{icps-2003-114}
\textbf{Stéphane Genaud}.
\newblock Applications parallèles sur la grille: mieux vaut il être rapide ou résistant ? 
\newblock {\em Actes de GridUSe-2004}, Workshop "What we have learned", conférence invitée.
\newblock Supélec Metz, juin 2004. 

\subsection*{En cours de soumission}

\bibitem{itpro-cs11}
\newblock \textbf{Stéphane Genaud}, Julien Gossa et Etienne Michon.
\newblock Provisioning Cloud resources on the client-side: a cost-performance trade-off approach.
IEEE IT Professional Magazine on Cloud Computing. Nov 2011

\bibitem{tomacs-11}
Olivier Beaumont, Laurent Bobelin, Henir Casanova, Pierre-Nicolas Clauss, 
Bruno Donassolo, Lionel Eyraud-Dubois, \textbf{Stéphane Genaud}, Sacha Hunold, 
Arnaud Legrand, Martin Quinson.
\newblock
Towards Scalable, Accurate, and Usable Simulations of Distributed Applications and Systems,
ACM Transactions on Modeling and Computer Simulation. Oct 2011.

\bibitem{michon2012}
Etienne Michon, Julien Gossa, \textbf{Stéphane Genaud}.
\newblock Free elasticity and free CPU power on IaaS Clouds: Promises and Study.
Europar 2012. Fev 2012.
\end{thebibliography}


%-----------------A N I M A T I O N -------------------
\subsection{Activités scientifiques}

\subsubsection{Niveau international}
Membre des comité de programmes des conférences internationales:\\[-3mm]
\begin{itemize}
\item[$\bullet$] 
14th IEEE International Conference on High Performance Computing and Communications (HPCC 2012), 
2012 (Liverpool, England),
\item[$\bullet$] 
13th IEEE International Conference on High Performance Computing and Communications (HPCC 2011), 
2011 (Banff, Canada),
\item[$\bullet$] 
IEEE/ACM International Conference on Grid Computing (GRID'10), 2010 (Bruxelles, Belgique),
\item[$\bullet$] 
20th IASTED International Conference on Parallel and Distributed Computing and Systems, 2010, (Marina Del Rey, USA),
\item[$\bullet$] 
IEEE/ACM International Conference on Grid Computing (GRID'08), 2008 (Tsukuba, Japon), 
\item [$\bullet$]
International Symposium on Grid and Distributed Computing, 2008 (Hainan Island, Chine),\\
\end{itemize}

Relecteur pour de nombreuses revues ou conférences internationales: IEEE Trans. on Distr. and Parallel Systems, 
J. of SuperComputing, J. of Grid Computing, IEEE Conference on Grid Computing, IEEE CCGrid conference, Europar,
IEEE IPDPS conference, \ldots.


\subsubsection{Niveau national}
%------------------------------------
\begin{itemize}

\item[$\bullet$] Obtention de la prime d'excellence scientifique (PES) à partir d'octobre 2009.\\

\item [$\bullet$]
membre du comité de rédaction de la revue Technique et Science Informatiques (2005--2009).\\
\end{itemize}

\underline{Projets en cours}\\
\begin{itemize}


\item[$\bullet$]
\textbf{Porteur local} pour le projet ANR SONGS (ANR 11 INFR 013-03)  (taux déclaré 40\%)
coordonné par Martin Quinson, LORIA, Nancy (2012-2015)  poursuivant le projet 
Uss-SimGrid (voir ci-dessous). Le projet vise à affiner les objets modélisés pour la 
simulation (processeurs multi-c{\oe}eurs, mémoire) ou en ajouter (disque, réseaux spécialisés
comme Infiniband) et à fournir des interfaces adaptées à la représentation de systèmes
complexes comme des machines HPC ou des Clouds. Je suis responsable du work package
sur les clouds.\\

\item[$\bullet$]
Participant au projet blanc ANR E2T2 (ANR 11 SIMI 9) (taux déclaré 15\%) coordonné par 
Peter Beyer, laboratoire PIIM, Université de Provence (2011-2014). L'objectif du projet 
est d'améliorer la modélisation physique des plasmas de bord dans un tokamak. Dans ce
projet, ma tâche est de co-encadrer un doctorant, Matthieu Kuhn avec Guillaume Latu et
Nicolas Crouseille (IRMA) pour paralléliser les codes développés par le CEA Cadarache 
(IRFM) et les physiciens du PIIM. \\

\item[$\bullet$]
\textbf{Co-animateur} d'une action d'animation scientifique 
dans le cadre de l'action de développement technologique (ADT) de Aladdin de l'INRIA, 
visant à pérenniser l'outil scientifique Grid5000. (07/2008--06/2012). 
Conjointement à l'ADT, l'animation scientifique est organisée autour de neuf actions d'animations baptisées \emph{défis}.
Je co-anime avec Nouredine Melab (LIFL, Université de Lille) le défi 
``{\em scalable application for large scale systems (algorithm, programming, execution models)}''.\\
\end{itemize}


\underline{Projets passés}\\
\begin{itemize}
\item[$\bullet$]
Participant (taux déclaré 20\%) au projet ANR USS-SimGrid (ANR 08 SEGI 022) coordonné par 
Martin Quinson, LORIA, Nancy (2009 -- 2011). Ce projet a été labelisé projet \textit{phare}
par l'ANR.
L'objectif général du projet était  d'élargir les capacités 
de l'environnement de simulation SimGrid pour satisfaire des besoins plus divers, comme la
simulation de systèmes pair-à-pair ou d'environnements de calcul intensif.
Mes tâches ont concerné l'enregistrement des traces d'exécutions (des programmes MPI en 
particulier) afin de les rejouer dans le simulateur. J'ai redémarré
le travail commencé à l'université de Hawaï sur l'interface SMPI, qui permet de simuler des 
programmes MPI sans modification des codes sources. Elle est maintenant fonctionnelle depuis
la version 3.5 de SimGrid.\\


\item[$\bullet$]
Participant (taux déclaré 20\%) au projet SPADES (ANR 08 SEGI 025) coordonné par Eddy Caron, 
LIP-ENS Lyon (2009 -- 2011). L'objectif était de concevoir et construire un intergiciel capable 
de gérer un environnement dans lequel la disponibilité des ressources change très rapidement. 
En particulier, cet intergiciel doit donner accès de manière fugace à des équipements de calculs 
très haute performance. Mes tâches ont concerné la conception et l'évaluation de l'ordonnanceur
travaillant en collaboration avec un système pair-à-pair utilisé pour recenser dynamiquement
les ressources disponibles.\\


\item[$\bullet$] Participant (10\%) au projet Masse de Données Astronomiques (ACI Masse de données) 
coordonné par Françoise Genova, observatoire de Strasbourg (2004 -- 2006). \\

\item[$\bullet$]
\textbf{Porteur d'un projet d'Action Concertée Incitative}. Projet TAG, 
pluridisciplinaire dans de l'ACI GRID (Globalisation des ressources informatiques et des données) 
du ministère de la recherche. Doté d'un budget de 182 K\euro{} et d'un poste d'ingénieur). 
(12/2001 -- 12/2003).\\
\end{itemize}


%\item Orateur dans diverses manifestations scientifiques sur ma thématique de recherche 
% * Journées \textit{Grilles} du CINES en mars 2003 à Montepllier, 
% * conférence et papier invité à GridUSe-2004 au workshop "What we have learned" 
%   de l'école thématique sur la \textit{globalisation des ressources informatiques et des données : Utilisation et Services},
%   juin 2004, Supélec Metz). 
% * Exposé Invité, Paristic 2006, 23 nov 2006, LORIA. Track 'Grid5000'. 
%   "Extensibilité d'un tracé de rais en tomographie sismique sur Grid5000."

% * Démonstrateur à SuperComputing'95 de l'environnement de développement Pei/VPei
% * Démonstrateur à SuperComputing'08 de l'environnement de développement P2P-MPI

\subsubsection{Niveau local}
\begin{itemize}


\item[$\bullet$]
\textbf{Responsable du thème} \emph{programmation parallèle sur les grilles} au sein l'équipe ICPS,
du laboratoire LSIIT. Ce thème a compté parmi ses membres~: Guillaume Latu (MC), 
Eric Violard (MC), Romaric David (IR), Benjamin Schwarz (IE en CDD), Arnaud Giersch (doctorant), 
Choopan Rattanapoka (doctorant) sur la période 2002 à 2007.\\

\item [$\bullet$]
Membre du conseil scientifique du département \emph{Expertise pour la recherche de l'UdS} (sept 2010--).
Le comité comprend 17 membres nommés, représentants les équipes scientifiques les plus impliquées
par rapport aux équipements de calcul de l'Université. Le rôle du comité est de piloter
l'investissement en matière de calcul, et de promouvoir les projets présentant le plus
d'intérêt scientifique par attribution de ressources.\\

\item[$\bullet$]
Correspondant depuis 2002 pour l'équipe ICPS auprès du groupe RGE (Réseau Grand Est),
action géographique regroupant 9 sites du GDR \textit{Architecture, Système et Réseaux} 
CNRS (GDR 725 ASR). RGE organise trois fois par an, une journée consacrée à des exposés 
scientifiques des équipes et à une conférence par un industriel invité.\\
%J'ai organisé la journée RGE du 01/06/2006 à Strasbourg.


\end{itemize}

\subsubsection{Jurys de thèse}

\begin{itemize}

\item[$\bullet$] 
Rapporteur de la thèse de Sébastien Miquée, Univ. Franche-Comté (soutenance 
jan. 2012), \textit{Exécution d'applications parallèles en environnements 
hétérogènes et volatils~: déploiement et virtualisation},
rapporteur C. Cérin (U. Paris 13), 
encadrants R. Couturier et D. Laiymani (U. Franche-Comté)\\

\item[$\bullet$] 
Rapporteur de la thèse de Fabrice Dupros, Univ. Bordeaux 1 (soutenance déc. 2010), 
\textit{Contribution à la modélisation numérique de la propagation des ondes 
sismiques sur architectures multic{\oe}urs et hiérarchiques},
rapporteur S. Lanteri (INRIA Sophia-Antipolis), 
encadrants D. Komatitsch (U. Pau) et J. Roman (Institut Polytechnique de 
Bordeaux).\\

\item[$\bullet$] 
Examinateur de la thèse d'Heithem Abbès (soutenance déc. 2009), 
\textit{Approches de décentralisation de la gestion des ressources dans les 
Grilles}, rapporteurs Mohamed Jmaiel (Université de Sfax) et Franck Capello 
(INRIA-U. Urbana-Champain), encadrants Christophe Cérin (U. Paris 13) et 
Mohamed Jemni (École Supérieure des Sciences et Techniques de Tunis).
\end{itemize}




\subsection{Encadrements}

\subsubsection{Thèses}
\begin{enumerate}

\item 10/2011-- : encadrement d'Etienne Michon. Taux d'encadrement: 50\%,
avec Julien Gossa. Financement DGA. La thèse porte sur les problématiques
d'allocation de ressources de cloud côté client.\\

\item 02/2011-- : encadrement de Matthieu Kuhn. Financement ANR E2T2. 
Taux d'encadrement prévisionnel: 20\%. Co-encadrants Guillaume Latu pour 
l'informatique, Nicolas Crouseille (HDR) pour les mathématiques appliquées. 
La thèse porte sur la parallélisation de modèles 
numériques pour la simulation de plasmas de bord.\\

\item 2004--2008 : encadrement de Choopan Rattanapoka. Taux d'encadrement: 100\%. 
Directeur de thèse: Catherine Mongenet. La thèse \underline{soutenue en avril 2008} 
est intitulée \textit{P2P-MPI: A Fault-tolerant Message Passing Interface Implementation 
for Grids} - rapporteurs : Franck Cappello (INRIA, Orsay) et Thilo Kielmann (Vrije 
Universiteit, Amsterdam). Choopan Rattanapoka a aujourd'hui un poste permanent 
d'assistant professor au Department of Eletronics Engineering Technology du King 
Mongkut's University of Technology, à Bangkok (Thailande).
Publications associées: 
\cite{icps-2005-155,icps-2007-182,icps-2007-185,
      icps-2008-188,icps-2008-193,icps-2009-214,
	icps-2009-217,icps-book}\\


\item 2001--2004 : co-encadrement d'Arnaud Giersch avec Frédéric Vivien. 
Taux de co-encadrement: $\sim$40\%. Directeur de thèse Guy-René Perrin (ULP).
La thèse \underline{soutenue en décembre 2004} est intitulée \textit{Ordonnancement 
sur plates-formes hétérogènes de tâches partageant des données} - rapporteurs : Denis 
Trystram (INPG, Grenoble) et Henri Casanova (UCSD, San Diego). Arnaud Giersch a 
aujourd'hui un poste de maître de conférences à l'IUT d'informatique de Belfort, 
université de Franche-Comté.
Publications associées: \cite{icps-2002-62,icps-2003-75,icps-2004-125}\\


\item 2000--2006 : co-encadrement de Marc Grunberg avec Catherine Mongenet 
(inscrit en thèse parallèlement à sa fonction d'ingénieur d'études au Réseau 
National de Surveillance Sismique). Taux de co-encadrement: $\sim$70\%.
Directeurs de thèse: Catherine Mongenet et Michel Granet (Physicien, ULP).
La thèse \underline{soutenue en septembre 2006} est intitulée \textit{Conception 
d'une méthode de maillage 3D parallèle pour la construction d'un modèle de Terre 
réaliste par la tomographie sismique} - rapporteurs : Thierry Priol (IRISA, Rennes) 
et Denis Trystram (INPG, Grenoble).
Marc Grunberg occupe toujours aujourd'hui un poste d'ingénieur d'études au Réseau 
National de Surveillance Sismique, \'{E}cole et Observatoire de Géophysique du Globe.
Publications associées: 
\cite{icps-2002-20,icps-2003-111,icps-2003-113,
      icps-2004-107,icps-2004-124,icps-2005-146,icps-2007-184}.\\



\end{enumerate}

\subsubsection{Stages de DEA/Master}
\begin{enumerate}

\item 2011 : Etienne Michon. Co-encadrement avec Julien Gossa. Mémoire 
intitulé  \textit{Allocation de ressources et ordonnancement côté client dans 
un environnement de Clouds}, soutenu 06/2011.
\item 2006 : Constantinos Makassikis. Co-encadrement avec Jean-Jacques Pansiot 
et Guillaume Latu. Mémoire intitulé \textit{Modèle de coût des communications 
TCP à un niveau applicatif}, soutenu 06/2006.
\item 2006 : Ghazi Bouabene. Mémoire intitulé  \textit{Sélection de pairs et 
allocation de tâches dans P2P-MPI}, soutenu 06/2006.
\item 2004 : Choopan Rattanapoka. Mémoire intitulé  \textit{P2P-MPI : A 
Peer-to-Peer Framework for Robust Execution of Message Passing Parallel 
Programs on Grids}, soutenu 07/2004.
\item 2002 : Dominique Stehly. Mémoire intitulé \textit{Ordonnancement 
d'applications parallèles sur la grille}, soutenu 07/2002.
\end{enumerate}

\subsubsection{Autres}
Encadrement internship INRIA
\smallskip
\begin{itemize}
\item[$\bullet$]  {\it Data Management in P2P-MPI}, Jagdish Achara, B-Tech de 
LNMIIT Jaipur, Inde. 3 mois, mai-août 2009. 
\item[$\bullet$]  {\it Optimisation de l'opération collective MPI all-to-all}, 
Antonio Grassi, Master de Facultad de Ingeniería, Universidad de la República, 
Montevideo, Uruguay. 4 mois, avril-août 2008, co-encadré avec Emmanuel Jeannot 
(AlGorille, LORIA).

\end{itemize}
~\\
Encadrement stage ENSIIE
\smallskip
\begin{itemize}
\item[$\bullet$]  {\it Experimentation de cluster virtualisé ave Nimbus}, 
Marien Ritzenthaler, ENSIIE 1A. 2,5 mois, juin-août 2010. 
\end{itemize}
~\\
%Encadrement apprentissage
%\begin{itemize}
%\item[$\bullet$]  
%{\it Conception et développement d'un outil de gestion de production}, Aymen Bouslama, Master 1 ILC, maître d'apprentissage David Damand, EM Strasbourg. 2010-2012.
%\end{itemize}

\label{sc:encadre-autres} 
Encadrements stages ou projets tutorés d'étudiants de l'UFR d'informatique, 
université de Strasbourg (150h, un mois plein). Parmi les plus récents:
\smallskip
\begin{itemize}
\item[$\bullet$]  {\it Interfaçage d'un batch scheduler un système de gestion 
de cloud IaaS}, Vincent Kerbache, TER, 2012, co-encadré avec J. Gossa. 
\item[$\bullet$]  {\it Mise en {\oe}uvre d'une technique de segmentation par 
ligne de partage des eaux dans un environnement distribué hétérogène}, 
Lionel Ketterer, projet tutoré, fév. 2007, co-encadré avec Sebastien 
Lefèvre (LSIIT).
\item[$\bullet$]  {\it Distribution de calculs de Pricing d'options au modele 
Europeen sur grille}, Nabil Michraf et Khalid Souissi, projet tutoré, fév. 2006, 
co-encadré avec Stéphane Vialle (Supelec).

\item[$\bullet$]  {\it Parallélisation de la méthode Adaboost}, Abdelaziz 
Gacemi, projet tutoré, fév. 2007.
\item[$\bullet$] {\it Réalisation d'un portail web pour P2P-MPI avec SOAP}, 
David Michea, projet tutoré, fév. 2005.

\item[$\bullet$] {\it Parallelisation de la méthode MACLAW}, stage master, 
avril-juillet 2006, Damien Vouriot, co-encadré avec Pierre Gancarski (LSIIT).
\item[$\bullet$] {\it Heursitiques d'ordonnancement basées sur des traces 
d'exécution pour pour programmes parallèles}, Ghazi Bouabene, stage master, 
juillet-septembre 2006.
\item[$\bullet$]  {\it Outil de visualisation du réseau P2P dans P2P-MPI}, 
Ghazi Bouabene, stage licence, juin-août 2005.
\end{itemize}





\subsection{Synthèse quantitative}

\begin{center}
	\textbf{--- Synthèse des activités de publication ---}\\[1mm]
\begin{tabular}{ll}
	\hline\\[-2mm]
	5	&	articles dans des revues internationales\\
	3	&	articles dans des revues nationales\\
	1	&	chapitre de livre dans une publication internationale\\
	15	& 	articles dans des conférences internationales à comité de lecture\\
	3	&	articles dans des conférences nationales à comité de lecture\\
	5	& 	participations à des comités de programmes de conférences internationales\\
	\hline\\
\end{tabular}


	\textbf{--- Synthèse des activités d'encadrement ---}\\[1mm]
\begin{tabular}{ll}
	\hline\\[-2mm]
	3	&	thèses soutenues encadrées ou co-encadrées\\
	1,5	& 	thèse en cours co-encadrée\\
	5	&	stages de DEA et master recherche encadrés ou co-encadrés\\
	7	& 	projets tutorés de niveau master encadrés ou co-encadrés\\
	1	& 	master en apprentissage encadré\\
	\hline\\
\end{tabular}

\end{center}



%-------------------------------- E N S G N T  ---------------------------------------------
\newpage

\section{Activités d'enseignement}
\label{sc:ensgnt-univ}

Je décris d'abord le contexte dans le lequel se sont déroulées mes activités
pédagogiques, puis je présente un tableau récapitulatif des cours assurés, suivi
d'une description des contenus de cours les plus récents. J'indique ensuite les
principales charges d'intérêt collectif exercées, et je termine par mes motivations
en matière de pédagogie.

\subsection{Contexte}

\noindent
J'ai été recruté en septembre 1998 sur un poste de maître de conférences 
en informatique (27$^e$ section) dans une composante de l'Université, l'IECS%
\footnote{Institut d'Enseignement Commercial Supérieur, devenue en 2007, après 
fusion avec l'IAE, l'\'Ecole de Management Strasbourg. \url{http://www.em-strasbourg.eu/}}
dédiée aux sciences de gestion (6$^e$ section). 
Ce poste préexistant marquait la volonté de la composante de développer les 
enseignements en systèmes d'information. \`A mon arrivée, j'ai pris en charge les cours
définis qui avaient trait à l'informatique. 
Puis en 2001, j'ai pris la responsabilité de la filière \textit{systèmes d'information} 
pour y développer et organiser les enseignements.
Mon rôle était de définir et de coordonner les enseignements de la filière, nécessitant
en plus des enseignants permanents, de recruter une dizaine de vacataires par an, 
professionnels de l'industrie dans leur très grande majorité. D'autre part, il m'incombait
le suivi des étudiants au travers de la définition de sujets de mémoires de fin d'études,
la validation de leur cursus à l'étranger, et l'information vis-à-vis des débouchés 
professionnels (réunions d'information avec des professionnels, animation d'échanges
avec des anciens de la filière).
Cette activité s'est arrêtée en 2007 lorsque je suis parti en détachement à l'INRIA.


Les principaux cours que j'ai enseignés dans cet établissement sont:
technologies des systèmes d'information, 
conception des systèmes d'information,
gestion de projet et outils pour la gestion de projet,
introduction à l'algorithmique et la programmation, 
architecture des applications web,
échange de données informatisées,
réseaux.
De retour de détachement, j'ai réintégré mon poste tout en effectuant plus d'un tiers
de mon service à l'extérieur de l'établissement (procédure facilitée par la fusion des 
universités strasbourgeoises), 
à l'UFR de mathématiques et informatique, et dans l'école d'ingénieur ENSIIE
\footnote{Ecole Nationale Supérieure d'Informatique pour l'Industrie et l'Entreprise, 
basée à Evry, ayant ouvert un campus à Strasbourg. \url{http://www.ensiie.fr/}}.


\subsection{Enseignements en informatique}

\noindent
\begin{itemize}
\item[$\bullet$] \textbf{IUT Informatique}
Avant mon recrutement comme maître de conférences, j'ai été pendant deux ans ATER temps plein
au département informatique de l'IUT de l'Université Robert Schuman de Strasbourg.
J'y ai effectué des enseignements en \textbf{algorithmique et programmation} et 
\textbf{système d'exploitation}. Durant cette période j'ai également continué 
d'assurer des cours du soir au CNAM 
(un tiers de l'unité de valeur \textbf{conception fondamentale des algorithmes} 
avec Franco Zaroli, de 1996 à 2000).\\

\item[$\bullet$] \textbf{UFR Mathématiques et Informatique}
Par ailleurs, j'ai gardé un contact permanent avec le département informatique 
de l'UFR de mathématiques-informatique de l'Université Louis Pasteur. 
J'ai assuré des vacations en licence (TD du cours \textbf{système d'exploitation}, 24h)
et un cours de \textbf{systèmes distribués} en DESS (14h par an, de 2001 à 2005).
J'ai poursuivi ce cours renommé \textbf{applications distribuées, parallélisme et grilles}
lors du passage aux masters (36h, effectué pour moitié avec Stéphane Vialle, SUPELEC).


Ma participation aux activités du département informatique se manifeste également par 
l'encadrement de certains Travaux d'Enseignement et Recherche ou d'encadrement 
de quelques projet 150h (voir liste page~\pageref{sc:encadre-autres}).
\end{itemize}




%--------------- T A B   E N S E I G N E M E N T S ------------------------------------
\subsection{Tableau récapitulatif}
\label{sc:tab-ens}

\begin{center}
{\footnotesize
\noindent \begin{tabular}{|c|c|p{5cm}|c|c|c|c|} \hline &&&&&\\[-8pt]

Année & Filière & Matière & cours & TD & TP &  occurrences \\[5pt] \hline
%&&&&&\\[-8pt]
 93-96 (3)
      & EOST 3 & progr. parallèle de méthode numériques pour la résolution de systèmes linéaires    &12&& &2    \\
      & CNAM cycle B & UV conception fondamentale des algorithmes & 12 & 12 & 	& 1 \\

\hline
96-98 (2)
      & IUT 1 + AS info 		& algorithmique 		&    & 24 & 64   &2\\
      & IUT 1 + AS info			& C et C++ 			&    & 24 & 64   &2\\
      & IUT 1 info                	& Système d'Exploitation, Unix 	& 6  & 16 & 	 &2\\
      & CNAM cycle B info		& UV conception fondamentale des algorithmes 		& 12 & 12 & 	 &2\\

 \hline
\multicolumn{7}{|c|}{\textit{Recrutement Maitre de conférences}}\\  \hline
 98-05 (7)
      & Licence info  & Système d'exploitation 	&  	& 24 & 	& 2 \\
	& DESS info	& Systèmes Distribués		& 14	&    &	& 4 \\
	& DEA info	& Modèles de programmation et grille (option) & 6	&    &	& 2 \\
	& CNAM cycle B 		& UV conception fondamentale des algorithmes 			& 12	& 12 &	& 2 \\
&&&&&&	\\
	& ENSPS 2	& Outils pour la gestion de projet		& 6	&    &   & 1 \\
	& IECS 2+3	& Gestion de projet				&  20	& 4 &   & 7 \\
	& IECS 1	& Introduction aux systèmes d'information	& 18	&    &  & 5 \\
	& IECS 2+3	& Introduction à la programmation		& 12	& 12   &  & 7 \\
	& IECS 2+3	& Technologies internet				& 24    &    & 24 & 7 \\

	& DESS ComElec	& Réseaux et échanges de données informatisés	& 12	&    &  & 4 \\
	
\hline
05-08 (3)
	& Master 2 info		& Applications distribuées, parallélisme et grilles	&  18	& & & 3 \\
	& Master Recherche info	& Modèles de programmation et grille (option) 		& 6	& & & 2 \\
\hline

\multicolumn{7}{|c|}{\textit{Détachement INRIA}}\\  \hline
09-11 (3)
	& Master 2 info		& Applications distribuées, parallélisme et grilles	&  45	&   &   & 3 \\
	& Master 2 info		& Fouille de données réparties			&  4	&   &   & 1 \\
	& ENSIIE 1		& Systèmes Informatiques				& 20	& 10.5 & 28 & 2 \\ 
&&&&&&\\
	& EM 2+3		& Outils pour la gestion de projet			&  40	& 8 &   & 2 \\
	& EM 2+3		& Conception et réalisation de systèmes d'information	&  12	&   &   & 1 \\
	& EM 2+3		& Technologies pour les applications web		&  48	&   & 24& 2 \\
	
\hline
\end{tabular}\\
}

\end{center}

%--- Etablisssemnts
\noindent
Les abrévitations sont les suivantes:\\

\begin{small}
\begin{tabular}{lp{14cm}}
AS		& Année Spéciale : cycle menant à l'obtention du DUT informatique 
		pour étudiants diplômés au moins d'un bac+2 non informatique. Aujourd'hui remplacé par la licence pro QCI.\\ 
ComElec	& Master Commerce Electronique: M2 en formation continue de l'IECS, \url{http://www.em-strasbourg.eu/docs/Master_CE.pdf}\\
EOST 		& \'Ecole et Observatoire des Sciences de la Terre : \'Ecole d'ingénieur dépendant de l'Université, \url{http://eost.u-strasbg.fr/ecoleing.php}\\
ENSIIE	& \'Ecole Nationale Supérieure d'Informatique pour l'Industrie et l'Entreprise, campus de Strasbourg. \url{http://www.ensiie.fr/}\\
ENSPS		& \'Ecole Nationale Supérieure de Physique de Strasbourg, \url{http://www-ensps.u-strasbg.fr/}\\
EM		& \'Ecole de Management Strasbourg. \\
\end{tabular}
\end{small}


\subsection{Contenus des enseignements}

\subsubsection*{Ecole de Management - Systèmes d'information}
\begin{itemize}
%
\item \textbf{Conception des systèmes d'information}: l'objectif du cours est de faire comprendre
aux étudiants en systèmes d'information les approches et techniques possibles ou employées 
pour analyser ou concevoir un système d'information. Dans l'application basée sur des études de cas, 
l'étudiant doit distinguer les cas d'utilisation, les flux d'information et les processus, et 
les données à mémoriser dans le système. Les technologies actuelles permettant de le faire sont 
aussi présentées.

\textit{Principes des méthodes de conception des projets logiciels 
(ex Merise), des cycles de développement, modélisation avec UML: modèle conceptuel des données
avec passage du diagramme de classes au modèle relationnel, représentations dynamiques d'un système. }\\

%
\item \textbf{Gestion de projet}: le cours présente la méthodologie usuelle de la planification
d'un projet et les outils ou méthodes qui sont disponibles indépendamment des spécificités
métiers. La mise en pratique se fait sur des études de cas avec un logiciel.

\textit{Démarche de gestion projet, principes de d'évaluation du projet, de découpages et 
d'estimation des tâches, planification PERT, PERT probabiliste, diagrammes Gantt. 
Application avec Microsoft Project et Primavera pour la gestion collaborative.}\\ 

%
\item \textbf{Introduction à l'algorithmique et à la programmation}: ce cours d'option
à destination d'étudiants en systèmes d'information donne les bases de l'algorithmique.
Il a pour objectif de montrer les problématiques techniques des équipes de réalisation
des systèmes d'information. La pratique se fait avec javascript et donne lieu à un projet.

\textit{Structures de contrôle et de
de données de la programmation impérative, notions de programmation orientée objet. 
Application avec JavaScript.}\\

%
\item \textbf{Architecture des applications web}: l'objectif est de faire comprendre
l'importance prise par ce type d'applications dans les systèmes d'information 
contemporains.  

\textit{Notion de client et serveur, 
serveur web et protocole http, serveur applicatif,
architecture $n$-tiers, principes des langages de script, structures de contrôles et de données de PHP, manipulation
des données avec SQL, interfaçage PHP/MySQL. Projet de mise en application durant lequel les étudiants
construisent un site web PHP/MySQL.}\\

%\item Echanges de données informatisés, réseaux.

\end{itemize}

\subsubsection*{ENSIIE}

\begin{itemize}
\item \textbf{Systèmes Informatiques}: l'objectif est de faire comprendre de nombreux principes
des systèmes informatiques en partant du problème de calculabilité jusqu'aux principes 
présidant les systèmes d'exploitation modernes, y compris leur composante réseau.
Cours conçu par Gérard Berthelot, ENSIIE Evry. TD assurés aussi par Alain Ketterlin (2009)
et Julien Gossa (2010).

\textit{Machines de Turing, automates, systèmes de numération, 
quantité d'information (Shannon), codages, structures d'un système d'exploitation,
gestion des processus et threads, gestion mémoire, gestion du disque, 
construction d'un exécutable, interpréteur de commandes, architectures réseaux, programmation socket TCP UDP}.
\end{itemize}


\subsubsection*{Master informatique}

\begin{itemize}
\item \textbf{Applications distribuées, parallélisme et grilles}: le cours
expose le paysage actuel des technologies dans le domaine de l'informatique distribuée ou
parallèle à travers quelques technologies clés et montre quelles technologies sont les mieux adaptées 
à tels ou tels besoins.
L'objectif est de donner du recul dans ce domaine aux étudiants, qui sont en passe de rejoindre
le monde professionnel.
Cours assuré pour moitié avec Stéphane Vialle.

\textit{Panorama des systèmes distribués, modèles de programmation RPC (illustration avec Java RMI),
web services (illustration avec SOAP), modèle de programmation pair-à-pair (illustration avec JXTA),
modèle à passage de messages (MPI), modèle pour le traitement de données massives (MapReduce), principes 
de mutualisation des ressources et intergiciels de grille (Globus), principe d'externalisation 
des traitements ou ressources (Software/Infrastructure/Platform as a Service)}.\\
 

\item \textbf{Modèles de programmation et grilles}: l'objectif de l'option est de présenter aux étudiants
se destinant à la recherche, un état de l'art des problématiques du développement d'applications
pour les \emph{grilles} ou les \emph{clouds}.

\textit{Catégories de grilles et usages, problématiques: modèles de programmation
disponibles, impact de l'hétérogénéité sur les performances, équilibrage de charge
applicatif, découverte des ressources, volatilités des ressources et tolérance aux pannes.}
\end{itemize}


\subsection{Principales charges d'intérêt collectif}

\begin{itemize}
\item[$\bullet$] Directeur délégué aux système d'information à l'EM Strasbourg (sept 2011--).\\
En charge d'un service de trois analystes-programmeurs. Mon rôle au sein de la direction
est de proposer des choix stratégiques pour les services à développer ou acquérir en matière 
de système d'information. Je planifie également la ré-organisation des services existants de 
l'établissment. Ces développements se font en concertation avec l'Université (Direction des 
Usages Numériques et Direction Informatique). Parmi les projets les plus importants~: 
implantation d'un CRM, gestion du processus de recrutement des vacataires, gestion des
étudiants à l'étranger.\\
		

\item[$\bullet$] Responsable de la filière d'enseignement \emph{système d'information} à l'IECS, Strasbourg (2001--2007).\\

\item[$\bullet$] 
Responsable de la mise en place de l'intranet de l'IECS (2000--2003). 
Participation au développement, essentiellement assuré par Christos Karacostas,
pour lequel j'étais l'encadrant de son mémoire de cycle C du CNAM (soutenu en juin 2001).\\

\item[$\bullet$] \textbf{Commissions de spécialistes} section CNU 27:
	\begin{itemize}
		\item titulaire à l'Université Louis Pasteur, Strasbourg, 2004--2008,
		\item suppléant à l'Université de Franche-Comté, Besançon, 2005--2008,
		\item suppléant à l'Université Henri Poincaré, Nancy, 2006--2008.\\
	\end{itemize}
	\item 2010-- : Membre du \textbf{comité d'experts} (9 membres) pour la section 
		27 de l'Université de Strasbourg.
	\item Membre des comités de sélection:
	\begin{itemize}
		\item poste MC 210 UdS Réseaux et Protocole, 2010,
		\item poste MC 1207 Université de Franche-Comté, IUT Belfort Montbéliard, 2010.\\
	\end{itemize}


\item[$\bullet$] 
Participation à l'animation de la communauté des enseignants-chercheurs en informatique, à travers 
l'action collective de l'association SPECIF (Société des Personnels Enseignants et Chercheurs 
en Informatique de France). Membre du conseil d'administration de SPECIF depuis 2010.
\end{itemize}
\vspace{1cm}



%-----------------------------------------------------------------------------|80
\subsection{Projet pédagogique}

Je connais le département informatique à travers les interventions que j'y mène
régulièrement (un cours annuel en master, interventions ponctuelles en license), 
depuis une dizaine d'années. D'autre part, beaucoup de mes collègues en recherche 
y sont permanents et ne manquent pas de me faire partager le quotidien du 
département. J'ai donc une connaissance empirique, mais bien sûr partielle, de 
la vie du département et de son évolution sur quelques années. Néanmoins, je 
garde une vision extérieure, façonnée en grande partie par mon expérience 
dans un établissement d'enseignement supérieur en sciences de gestion opérant 
dans un contexte très compétitif.

Je partage avec les collègues membres du département ce constat et ce regret: 
la licence et les masters devraient former plus d'étudiants, tout en gardant le 
même niveau d'exigence académique. Localement, on constate que les attentes de 
l'industrie concernant le nombre d'étudiants formés ne sont pas satisfaites. 
Un audit mené par la région sur les formations de type ingénieur avait conclu 
en particulier que le déficit le plus fort concernait les formations en 
informatique --- ce qui a ensuite débouché sur un appel et l'implantation
d'un campus à Strasbourg de L'ENSIIE. Un autre indicateur est le nombre de
demandes d'apprentis de la part des entreprises, qui est supérieur à ce que
le master en alternance peut accueillir dans sa configuration actuelle.

Face à cette demande, paradoxalement, les étudiants se sont moins orientés 
vers la licence informatique ces dernières années  Le problème de la baisse des 
effectifs est posé à toutes les filières scientifiques. Cependant, les 
formations en informatique de l'UFR ont deux atouts considérables. D'une part, 
le marché de l'emploi très actif dans le secteur des technologies de la 
communication peut permettre de développer davantage les relations avec les 
entreprises et accroître l'attractivité auprès des étudiants. D'autre part, 
la réputation de l'Université de Strasbourg permet d'envisager une 
plus large publicité des formations, en France mais aussi à l'étranger.

Je détaille ci-dessous ma perception de ces deux aspects. Je montre que le
levier de la relation avec les entreprises est déjà partiellement actionné.
Par conséquent, mon projet vise à actionner le second levier à travers l'%
\textbf{internationalisation des formations du département}.


%------------------------------------------------------------------------------|80

\subsubsection{Relations avec le monde professionnel}

Des actions sur le moyen terme pourraient jouer en faveur d'une communication 
plus développée avec le monde professionnel. En particulier, l'outil de CRM%
\footnote{\textit{Customer Relationship Management}, logiciel dédié à la 
gestion relation client.}
dans le nouveau système d'information de l'université (à la rentrée 2013) 
devrait permettre de gagner en efficacité dans la gestion de contacts, des 
stages et des diplômés. La fidélisation des diplômés est un effort qui peut 
être payant et ouvrir la porte des entreprises pour des stages, des contrats 
de collaborations, thèses CIFRE, etc. Une autre piste est la création de 
formation en co-tutelle avec des établissements ayant de fortes relations avec
le tissu socio-économique comme l'EM Strasbourg.

Cependant, ce rapprochement avec les entreprises a déjà commencé à s'opérer 
avec le passage du master Ingénierie du Logiciel et des Connaissances (ILC) en 
alternance à partir de 2007. Ce passage à l'alternance est de nature à créer 
desrelations durables avec les entreprises. Le recrutement en 2011 d'une 
chargée de communication et des relations entreprise atteste de l'importance de 
cette activité. De même, le succès de cette formation auprès des étudiants ne 
se dément pas après plusieurs années. La formation est même passée en 
alternance sur les deux années du master en 2010. J'ai le sentiment que cette 
attractivité de l'alternance bénéficiera sur le long terme à l'ensemble des 
formations, car elle renforce la réputation du département en matière de liens 
avec le monde professionnel, et donc d'opportunités de carrière pour les 
étudiants. Cette perception, si elle se répand, pourrait convaincre plus 
d'étudiants de s'inscrire en licence informatique. En revanche, il existe un 
risque dans le contexte actuel, que l'alternance déséquilibre l'orientation des 
étudiants en asséchant les effectifs des autres filières. Il faut donc 
renforcer et mettre davantage en avant d'autres perspectives pour les 
formations classiques, notamment celles plus orientées vers la recherche. 
Une piste est de construire un ou des masters recrutant à l'international.


\subsubsection{Internationalisation}

%------------------------------------------------------------------------------|80
Les étudiants étrangers représentent 20\%\footnote{Dont 47\% en doctorat, 31\% 
en Master et 15\% en licence. Chiffres 2010.} des effectifs de l'Université de 
Strasbourg. Ces chiffres montrent l'attractivité de l'Université.  Néanmoins, 
les formations en informatique et le département informatique en particulier, 
sont très loin de ces niveaux.

J'ai été récemment confronté à la demande d'un étudiant indien que j'avais 
encadré en stage INRIA. Celui-ci, en passe de terminer son bachelor en Inde, 
souhaitait faire acte de candidature pour un master en informatique à \hbox{l'
Université} de Strasbourg. Cet étudiant ne maîtrisant pas le français, j'ai fait 
une enquête rapide auprès de mes collègues du département informatique, et je 
n'ai malheureusement pu que le décourager de faire cette démarche. Cet étudiant 
s'est donc tourné vers des programmes proposant des enseignements en anglais 
proposés aux Pays-bas et en Allemagne. Les demandes d'étudiants étrangers 
possédant déjà une formation de premier cycle de bon niveau sont croissantes. 
Si des formations de type master international pouvaient les accueillir, ils 
formeraient un vivier de ressources intéressant, notamment pour la formation 
doctorale ou les équipes de recherche. 

Le projet que j'aimerais promouvoir si j'étais recruté à l'UFR serait la 
construction d'un programme de master international. Mon établissement de 
rattachement actuel propose un cursus en formation initiale largement ouvert 
à l'international (plus de 150 universités partenaires). Les étudiants doivent 
faire au moins une de leurs trois années dans une Université à l'étranger.
Réciproquement, l'établissement reçoit de nombeux étudiants étrangers en échange, 
qui représentent jusqu'à 50\% des étudiants présents. Les étudiants sont ainsi 
confrontés à des cultures et des méthodes de travail différentes, impliquant 
des remises en question particulièrement enrichissantes.

En discutant avec mes collègues, j'ai noté que la plupart d'entres eux 
se disaient prêts à faire des enseignements en anglais. Dans le domaine de 
l'informatique, la technicité de la discipline rend le fond plus important que 
la forme et je crois qu'il n'est pas nécessaire de maîtriser la langue anglaise 
comme une langue maternelle pour dispenser un bon enseignement.

Trois solutions au mois sont possibles pour un tel projet: i) intervenir 
dans une formation délocalisée, ii) constuire un dossier Erasmus Mundus, 
iii) créer ou transformer un master existant avec des enseignements en 
anglais. La première solution est déjà expérimentée par des Universités 
comme Bordeaux 1 et Paris 6 depuis 2006 dans un master implanté sur site 
à Hô-Chi-Min-Ville au Vietnam, via le le Pôle Universitaire Français.
Le retour d'expérience semble mitigé : les difficultés logistiques et
financières sont conséquentes alors que les étudiants diplômés sont peu
nombreux à poursuivre en France en doctorat ou dans l'industrie.
La deuxième solution d'un master Erasmus Mundus est la plus prestigieuse 
mais aussi la plus longue à construire. Un master Erasmus Mundus est une
formation conjointe à plusieurs établissements européens, ouverte à des 
candidats du monde entier, qui doivent passer leur scolarité sur au moins
deux des sites partenaires. Elle assure un fort taux de selectivité par
un recrutement au niveau mondial. Cependant, c'est aussi la solution la
plus incertaine en raison de la forte sélectivité, et la plus longue car 
un des critères d'éligibilité est la pré-existence de collaborations
entre les établissements partenaires. La troisième solution semble plus 
simple à mettre en place. C'est ce qu'à fait Grenoble avec la déclinaison 
de son master 2 recherche en deux parcours%
\footnote{\url{http://ufrima.imag.fr/spip.php?rubrique94}} enseigné en 
français, l'autre en anglais. Le parcours en anglais, 
\href{http://mosig.imag.fr/}{MoSIG} est perçu, y compris par les 
étudiants français, comme la formation de prestige. Cette solution est 
probablement transposable à Strasbourg, sur n'importe lequel des masters 
orienté recherche, ou, dans une prochaine maquette, sur un nouveau master
recherche plus généraliste qui emprunterait des cours en anglais à 
différents masters.



\subsection*{Conclusion}

Je serais donc heureux de pouvoir initier ou participer au montage de tels
programmes internationaux, en faisant bénéficier le département de mon 
expérience de l'EM Strasbourg, qui a developpé un réseau extrêmement large
d'Universités partenaires à l'étranger. Le paysage de l'offre dans 
l'enseignement supérieur est actuellement en pleine restructuration, en 
France mais aussi chez nos voisins, avec la formation de réseaux d'excellence 
dont l'Université de Strasbourg a vocation à faire partie. C'est donc une 
période d'opportunités qu'il ne faut pas manquer de saisir.



% Pour faire apparaître une section biblio classique
%\bibliography{generalbiblio}

%--------------------- A N N E X E S ------------------------

%PIECES PR
%― la déclaration de candidature imprimée depuis GALAXIE, datée, avec la signature du candidat ;
%― une copie d'une pièce d'identité avec photographie ;
%― une pièce attestant de la possession de l'un des titres mentionnés à l'article 9 ci-dessus ;
%― un curriculum vitae donnant une présentation analytique de leurs travaux, ouvrages, articles, réalisations et activités en précisant ceux qui sont joints ;
%― un exemplaire d'au moins un des travaux, ouvrages, articles et réalisations parmi ceux mentionnés dans le curriculum vitae ;
%― une copie du rapport de soutenance du diplôme détenu, le cas échéant.

\newpage
\section{Documents annexes}

Le reste du document fournit:\\
\begin{itemize}
\item la copie d'une pièce d'identité,
\item la déclaration de candidature imprimée depuis GALAXIE, datée et signée,
\item l'attestation de réussite au diplôme d'HDR,
\item le rapport de soutenance d'HDR,
\item les trois rapports de pré-soutenance,
\item La recommandation de Pierre Tellier, directeur des études de l'antenne de Strasbourg de l'ENSIIE,
\item La recommandation de Monique Rice, directrice des études à l'IECS.
\item La recommandation de Babak Mehmanpazir, directeur de l'IECS de 2005 à 2007.
\item La recommandation de Jens Gustedt, chef de l'équipe projet INRIA AlGorille. 
\end{itemize}

\includepdf[pages=1-2]{fiche_candidature.pdf}
\includepdf[pages=1]{genaud_id.pdf}
\includegraphics[height=.98\textheight]{attestation_hdr_steph.jpg}
\includepdf[pages=1]{rapport_soutenance.pdf}
\includepdf[pages=1-5]{rapport_Cerin.pdf}
\includepdf[pages=1-2]{rapport_Desprez.pdf}
\includepdf[pages=1-2]{rapport_Priol.pdf}
\includepdf[pages=1]{reco_tellier.pdf}
\includepdf[pages=1]{reco_rice.pdf}
\includepdf[pages=1]{reco_babak.pdf}
\includepdf[pages=1-2]{reco_jens.pdf}

%----------------------------- A R T I C L E S -------------------------------------------------------
\section{Copies d'articles}


\vspace{1cm}
Je joint ci-après six copies d'article. Ils sont classés dans un ordre chronologique inverse:

\vspace{1cm}

\begin{enumerate}
\item Article sur la simulation de programmes MPI dans le simulateur \textsc{SimGrid} (SMPI)\\
Single Node On-Line Simulation of MPI Applications with SMPI.\\
Pierre-Nicolas Clauss, Mark Stillwell, \textbf{Stéphane Genaud}, Fr\'ed\'eric Suter,\\
{\em 25th IEEE International Parallel \& Distributed Processing Symposium (IPDPS 2011)}, 
IEEE Computer Society Press, mai 2011.\\


\item Article sur la tolérance aux pannes et la détection de pannes en P2P-MPI.\\ 
Fault-Management in P2P-MPI.\\
Stéphane Genaud, Emmanuel Jeannot et Choopan Rattanapoka.\\
{\em International Journal of Parallel Programming}, Springer, 37(5):433--461, août 2009.\\


\item Article sur la parallélisation d'une méthode de clustering avec P2P-MPI.\\
Exploitation of a parallel clustering algorithm on commodity hardware with P2P-MPI.\\
Stéphane Genaud, Pierre Gançarski, Guillaume Latu, Alexandre Blansché, Choopan Rattanapoka et Damien Vouriot.\\
{\em The Journal of SuperComputing}, Springer, 43(1):21--41, jan. 2008.\\


\item Article sur la conception de P2P-MPI\\
P2P-MPI: A Peer-to-Peer Framework for Robust Execution of Message Passing Parallel Programs on Grids.
Stéphane Genaud et Choopan Rattanapoka.\\
{\em Journal of Grid Computing}, Springer, 5(1):27--42, mai 2007.\\


\item Article sur l'équilibrage de charge.\\
Load-balancing scatter operations for Grid computing.\\
Stéphane Genaud, Arnaud Giersch, et Frédéric Vivien.\\
{\em Parallel Computing}, Elsevier, 30(8):923--946, août 2004.\\


\item Article sur la parallélisation du tracé de rai de l'application de géophysique.\\
Seismic ray-tracing and Earth mesh modeling on various parallel architectures.\\
Marc Grunberg, \textbf{Stéphane Genaud} et Catherine Mongenet.\\
{\em The Journal of Supercomputing}, Kluwer, 29(1):27--44, juillet 2004.\\
\end{enumerate}

\end{document}
\includepdf[pages=1-18]{../mypapers/Seismic-raytracing_JSC04.pdf}
%\includepdf[pages=1-8]{../mypapers/Mesh_coarsening_SBAC-PAD-04.pdf}
\includepdf[pages=1-24]{../mypapers/Load-bal-scatter_ParCo04.pdf}
\includepdf[pages=1-16]{../mypapers/P2P-MPI_joGC2007.pdf}
\includepdf[pages=1-21]{../mypapers/Parallel-clustering-P2PMPI_JSC08.pdf}
\includepdf[pages=1-28]{../mypapers/P2P-MPI-fault_management-IJPP.pdf}
\includepdf[pages=1-12]{../mypapers/IPDPS-11.pdf}






\end{document}
