
% PROjet d'enseignement

% Détails responsabilités filière

%   20% 

% Motivations pour l'enseignement

% 3 lettres reco
% copie des contrats

\documentclass[11pt]{article}
%\usepackage{makeidx}
%\usepackage{graphicx}
\usepackage[utf8]{inputenc}
\usepackage[OT1]{fontenc}
\usepackage[francais]{babel}
%\usepackage{a4wide}
\usepackage{fancybox}
%\usepackage{hyperref}
\usepackage{comment}
\usepackage{geometry}
\usepackage{eurosym}
\usepackage{amssymb}
\usepackage[pdftex]{graphicx}
\usepackage{pdfpages}

\usepackage{pgf}
\usepackage{pdfswitch}
\hypersetup{urlcolor=black,linkcolor=black,citecolor=black} % Keep printer friendly
%\geometry{hmargin=3.5cm, bottom=22mm,vmargin=2.5cm }
\geometry{top=23mm,bottom=27mm,left=17mm,right=27mm,headsep=0pt}
\setlength{\parindent}{1cm}


\usepackage[twoside,citeonce(page)]{footbib}
\usepackage{todonotes}
\usepackage{sectsty}

\footbibliography{generalbiblio}
\footbibliographystyle{plain}

\newcommand {\burl}[1]{\url{http://icps.u-strasbg.fr/~genaud/courses/#1}}

\newcommand {\rubrique}[1]{
\begin{flushleft}
{\large\bf #1}
\rule{\linewidth}{1mm}
\end{flushleft}
}
\newcommand{\pmpi}{\mbox{\textsc{P2P-MPI}}}


\makeatletter
\def\thebibliography#1{\subsection*{\rule{\linewidth}{1pt}\@mkboth
%\def\thebibliography#1{\subsection*{\rubrique{\large\bf Références biliographiques}\@mkboth
  {PUBLICATIONS}{PUBLICATIONS}}\list
      {[\arabic{enumi}]}{\settowidth\labelwidth{[#1]}\leftmargin\labelwidth
      \advance\leftmargin\labelsep
      \usecounter{enumi}}
      \def\newblock{\hskip .11em plus .33em minus .07em}
      \sloppy\clubpenalty4000\widowpenalty4000
      \sfcode`\.=1000\relax}
    \makeatother


\begin{document}
%------------------------- Page de Garde -------------------
%\setlength{\parskip}{0mm}
\thispagestyle{empty}
\vspace*{\stretch{1}}



\noindent
\rule{\linewidth}{1mm}
\begin{center}
\Large{\textbf{Rapport d'activités de recherches et administratives}}\\[3mm]
\Large{2009--2012}\\[3mm]
\Large{Stéphane \textsc{Genaud}}\\[1cm]

\rule{\linewidth}{1mm}
\vspace*{\stretch{2}}\\
\vspace{3cm}
%\DeclareGraphicsExtensions{.jpg}
%\includegraphics[width=5cm]{inria.jpg}
\end{center}
\begin{center}
Date: \today\\
\end{center}

\newpage

\setlength{\parindent}{5mm} %% restore a decnt value
\setlength{\parindent}{0mm}
%\setlength{\itemsep}{1mm}


%---------------------- P O S I T I O N   A D M I N I S T R A T I V E ------------------------------
\vspace{8mm}
\textbf{\underline{Position administrative actuelle (depuis novembre 2012)}}
\vspace{5mm}

\begin{itemize}
\item Fonction :\textbf{Directeur de l'antenne de Strasbourg de l'ENSIIE}.\\[1mm]
       
\item Position administrative: 
   \begin{itemize}
	\item[$\rhd$] Maître de conférences mis à disposition ide l'ENSIIE par l'Université de Strasbourg (UdS),
	\item[$\rhd$] Chercheur au Laboratoire des sciences de l'ingénieur, de l'informatique et de l'imagerie
	  l'Informatique et de la Télédétection (Icube, UMR 7357 CNRS-UdS),
          membre de l'équipe \textit{Image et Calcul Parallèle Scientifique} (ICPS).\\
   \end{itemize}

\item Titulaire de la \textbf{Prime d'Excellence Scientifique} depuis septembre 2009.\\

\item \textbf{Qualifié aux fonctions de Professeur des Universités} dans la 
              section 27 en 2010.\\
	\indent {\small Numéro de qualification~:~PR-2010-27-10127207589}\\[2mm]
\end{itemize}



%------------------ R E S P O N S A B I L I T E S ---------------------------

\section{Responsabilités principales (2009-2012)}
\vspace{3mm}
\medskip

\begin{itemize}
\item
 	\textbf{Responsabilités administratives}\\
	\begin{itemize}
		\item \textbf{Directeur de l'antenne de Strasbourg de l'ENSIIE}, depuis 03/2013.
		\item Directeur délégué aux Systèmes d'Information de 
		      l'\'Ecole de Management Strasbourg 07/2011-09/2012.\\
	\end{itemize}


 	\textbf{Responsabilités recherche}
	\begin{itemize}
		\item Responsable du thème de recherche \textit{Grilles et Clouds} dans 
		      l'équipe ICPS au sein du laboratoire Icube. Création du thème en 2002. 
	              Reprise du thème à mon retour de détachement INRIA en septembre 2009.
                      L'effectif actuel (au 14/5/2013) du thème est : 1 Directeur de recherche 
                      INRIA, 2 Maitres de conférences, 1 PostDoc, 1 doctorant, 2 Masters recherche.
	\end{itemize}
\end{itemize}


%---------------------------- R E C H E R C H E --------------------------------
\newpage
\section{Activités de recherches}

Sur la période 2009-2012, mes thèmes de recherche ont abordés trois domaines.

\paragraph{Grilles informatiques}
Lors de  l'année 2009, j'ai  finalisé une période  de recherche sur  les grilles
débutées en 2002.  Dans la dernière  phase de cette période, nous avons finalisé
le développement du logiciel \href{http://www.p2pmpi.org}{\pmpi}, logiciel libre
qui   a   servi   de   prototype    de   recherche.    La   thèse   de   Choopan
Rattanapoka~\footcite{icps-2008-208} achevée  en avril 2008  présente l'ensemble
des résultats.  Les derniers résultats, publiés  en journal~\cite{icps-2009-217}
et  en  conférence  invitée\cite{icps-2009-214}~, concernent  la  tolérance  aux
pannes par  réplication des processus  de calcul. Enfin, une  collaboration avec
des  collègues  de   Supelec  a  permis  de  comparer  {\pmpi}   à  la  solution
JavaSpace~\cite{icps-2009-217}.

\paragraph{Simulation de systèmes distribués}
Ma   participation  aux   projets  ANR   destinés  à   développer  le   logiciel
\textsc{SimGrid} (ANR Uss-SimGrid 2009-2012,  puis ANR SONGS 2012-2015) est
allé croissante. Après avoir été  participant simple en contribuant à développer
la  simulation  de  programmes   MPI  avec  SMPI~\cite{icps-2011-224},  je  suis
responsable d'un work package destiné à  la simulation de clouds dans le nouveau
projet ANR. Ce  projet finance un postdoc,  Marc Frincu qui a  débuté en octobre
2012  et   permis  d'étudier   différentes  stratégies  d'allocation   pour  des
workflows~\cite{FrincuGG13}.


\paragraph{Clouds IaaS côté client}
Parallèlement à la  simulation des systèmes de clouds, je  travaille depuis 2010
sur des  problèmes d'allocation  des ressources  et d'ordonnancement  des tâches
dans des clouds IaaS. L'objectif est de construire des stratégies qui permettent
d'aider le client de ces clouds à  faire ses choix dans les différents compromis
possibles  entre  le coût  des  ressources  louées  et la  performance  attendue
\cite{icps-2011-225,michon2012}. Ce travail fait l'objet de la thèse d'Etienne 
Michon depuis octobre 2011.



%------------------------ P U B L I C A T I O N S -----------------------------|
\subsection{Liste de publications 2009-2013}


\small
\bibliographystyle{plain}
\begin{thebibliography}{99}

\subsection*{Thèses}

\bibitem{hdr}
\textbf{Stéphane Genaud}.
\newblock 
{\em Exécutions de programmes parallèles à passage de messages sur grille de 
calcul}.
\newblock 
Habilitation à diriger des recherches de l'université Henri Poincaré, 
Nancy. Décembre 2009.
\newblock 
Rapporteurs : C. Cérin (Paris 13), F. Desprez (INRIA Rhônes-Alpes), 
T. Priol (INRIA Bretagne-Atlantique).\\[2mm]

\subsection*{Chapitre de livre}

\bibitem{icps-book}
\textbf{Stéphane Genaud} et Choopan Rattanapoka.
\newblock 
\emph{A Peer-to-Peer Framework for Message Passing Parallel Programs.}
\newblock 
Parallel Programming and Applications in Grid, P2P and Network-based System,
in {\em Advances In Parallel Computing} Series. Editor G. R. Joubert.
IOS Press, juin 2009. 
 

\subsection*{Articles en revues internationales}

\setlength{\itemsep}{1.5mm}


\bibitem{icps-2009-217}
\newblock \textbf{Stéphane Genaud}, Emmanuel Jeannot et Choopan Rattanapoka.
\newblock Fault-Management in P2P-MPI.
\newblock {\em International Journal of Parallel Programming}, Springer, 
37(5):433--461, août 2009.


\subsection*{Conférences internationales avec actes et comité de lecture}

\bibitem{FrincuGG13}
Marc Frincu, \textbf{Stéphane Genaud} et Julien Gossa.
\newblock Comparing Provisioning and Scheduling Strategies for Workflows on Clouds
\newblock {\em 2nd IEEE International Workshop on Workflow Models, Systems, Services
and Applications in the Cloud (CloudFlow), IPDPS 2013}, mai 2013.

\bibitem{michon2012}
Etienne Michon, Julien Gossa, \textbf{Stéphane Genaud}.
\newblock Free elasticity and free CPU power for scientific workloads on IaaS Clouds
\newblock {\em 18th IEEE International Conference on Parallel and Distributed Systems}, 
IEEE, déc. 2012.
\newblock \small{\textit{(papiers acceptés/soumis:87/294, taux: 29\%)}}


\bibitem{icps-2011-225}
\newblock \textbf{Stéphane Genaud} et Julien Gossa,
\newblock Cost-wait Trade-offs in Client-side Resource Provisioning with 
Elastic Clouds.
\newblock {\em 4th IEEE International Conference on Cloud Computing (CLOUD 
2011)}, juillet 2011.
\newblock \small{\textit{(papiers acceptés/soumis{:}36/198, taux: 18\%)}}


\bibitem{icps-2011-224}
\newblock Pierre-Nicolas Clauss, Mark Stillwell, \textbf{Stéphane Genaud}, 
Fr\'ed\'eric Suter, Henri Casanova and  Martin Quinson.
\newblock Single Node On-Line Simulation of MPI Applications with SMPI.
\newblock {\em 25th IEEE International Parallel \& Distributed Processing 
Symposium (IPDPS 2011)}, mai 2011.
\newblock \small{\textit{(papiers acceptés/soumis{:}112/571, taux: 19\%)}}

\bibitem{icps-2009-219}
\newblock Virginie Galtier, \textbf{Stéphane Genaud} et Stéphane Vialle.
\newblock Implementation of the AdaBoost Algorithm for Large Scale Distributed 
Environments: Comparing JavaSpace and MPJ.
\newblock {\em 15th IEEE International Conference on Parallel and Distributed Systems}, 
IEEE, déc. 2009.
\newblock \small{\textit{(papiers acceptés/soumis:91/305, taux: 29\%)}}


\bibitem{icps-2009-214}
\textbf{Stéphane Genaud} and Choopan Rattanapoka.
\newblock Evaluation of Replication and Fault Detection in P2P-MPI.
\newblock 
{\em 6th IEEE International Workshop on Grid Computing (HPGC), IPDPS 2009}, 
mai 2009.
\newblock \textit{(Papier invité)}.

\subsection*{Autres communications}

\bibitem{iphc-2011}
Christine Carapito, Jérôme Pansanel, Patrick Guterl, Alexandre Burel, Fabrice 
Bertile, \textbf{Stéphane Genaud}, Alain Van Dorsselaer, Christelle Roy.
\newblock Une suite logicielle pour la protéomique interfacée sur une grille de 
calcul. Utilisation d'algorithmes libres pour l'identification MS/MS, le 
séquençage de novo et l'annotation fonctionnelle.
\newblock Rencontres Scientifiques France Grilles 2011, Lyon.

\bibitem{ketterlin11}
Alain Ketterlin, \textbf{Stéphane Genaud}, Matthieu Kuhn.
\newblock Loop-Nest Recognition for the Extraction of Communication Patterns 
and the Compression of Message-Passing Parallel Traces.
\newblock Research Report ICPS 11-01. Université de Strasbourg. déc. 2011.


\subsection*{En cours de soumission}

\bibitem{sc13}
Paul Bédaride, Augustin Degomme, Stéphane Genaud, Arnaud Legrand, George S. Markomanolis,
Martin Quinson, Frédéric Suter, Mark Stillwell, Brice Videau
\newblock
Improving Simulations of MPI Applications Using A Hybrid Network Model with Topology and Contention Support,
International Conference for High Performance Computing, Networking, Storage, and Analysis (SuperComputing 13).
\end{thebibliography}


%-----------------A N I M A T I O N -------------------
\subsection{Activités scientifiques}

\subsubsection{Niveau international}
Membre des comité de programmes des conférences internationales:\\[-3mm]
\begin{itemize}
\item[$\bullet$]
15th IEEE International Conference on Computational Science and Engineering (CSE 2012) 
Cluster, Grid, Cloud and P2P Computing track. Paphos, Cyprus, October 3-5, 2012. 
http://www.cse2012.cs.ucy.ac.cy/

\item[$\bullet$] 
14th IEEE International Conference on High Performance Computing and Communications (HPCC 2012), 
2012 (Liverpool, England),
\item[$\bullet$] 
13th IEEE International Conference on High Performance Computing and Communications (HPCC 2011), 
2011 (Banff, Canada),
\item[$\bullet$] 
IEEE/ACM International Conference on Grid Computing (GRID'10), 2010 (Bruxelles, Belgique),
\end{itemize}

\subsubsection{Niveau national}
%------------------------------------
\begin{itemize}

\item[$\bullet$] Obtention de la prime d'excellence scientifique (PES) à partir d'octobre 2009.\\

\item [$\bullet$]
expert pour l'ANR, appel à projets JCJC SIMI 2 Science informatique et applications (2013).\\
\end{itemize}

\underline{Projets en cours}\\
\begin{itemize}

\item[$\bullet$]
\textbf{Porteur local} pour le projet ANR SONGS (ANR 11 INFR 013-03)  (taux déclaré 40\%)
coordonné par Martin Quinson, LORIA, Nancy (2012-2015)  poursuivant le projet 
Uss-SimGrid (voir ci-dessous). Le projet vise à affiner les objets modélisés pour la 
simulation (processeurs multi-c{\oe}urs, mémoire) ou en ajouter (disque, réseaux spécialisés
comme Infiniband) et à fournir des interfaces adaptées à la représentation de systèmes
complexes comme des machines HPC ou des Clouds. Je suis responsable du work package
sur les clouds.\\

\item[$\bullet$]
Participant au projet blanc ANR E2T2 (ANR 11 SIMI 9) (taux déclaré 15\%) coordonné par 
Peter Beyer, laboratoire PIIM, Université de Provence (2011-2014). L'objectif du projet 
est d'améliorer la modélisation physique des plasmas de bord dans un tokamak. Dans ce
projet, ma tâche est de co-encadrer un doctorant, Matthieu Kuhn avec Guillaume Latu et
Nicolas Crouseille (IRMA) pour paralléliser les codes développés par le CEA Cadarache 
(IRFM) et les physiciens du PIIM. \\

\textbf{Co-animateur} d'une action d'animation scientifique 
dans le cadre de l'action de développement technologique (ADT) de Aladdin de l'INRIA, 
visant à pérenniser l'outil scientifique Grid5000. (07/2008--06/2012). 
Conjointement à l'ADT, l'animation scientifique est organisée autour de neuf actions d'animations baptisées \emph{défis}.
Je co-anime avec Nouredine Melab (LIFL, Université de Lille) le défi 
``{\em scalable application for large scale systems (algorithm, programming, execution models)}''.\\
\end{itemize}


\underline{Projets passés}\\
\begin{itemize}
\item[$\bullet$]
Participant (taux déclaré 20\%) au projet ANR USS-SimGrid (ANR 08 SEGI 022) coordonné par 
Martin Quinson, LORIA, Nancy (2009 -- 2011). Ce projet a été labelisé projet \textit{phare}
par l'ANR.
L'objectif général du projet était  d'élargir les capacités 
de l'environnement de simulation SimGrid pour satisfaire des besoins plus divers, comme la
simulation de systèmes pair-à-pair ou d'environnements de calcul intensif.
Mes tâches ont concerné l'enregistrement des traces d'exécutions (des programmes MPI en 
particulier) afin de les rejouer dans le simulateur. J'ai redémarré
le travail commencé à l'université de Hawaï sur l'interface SMPI, qui permet de simuler des 
programmes MPI sans modification des codes sources. Elle est maintenant fonctionnelle depuis
la version 3.5 de SimGrid.\\


\item[$\bullet$]
Participant (taux déclaré 20\%) au projet SPADES (ANR 08 SEGI 025) coordonné par Eddy Caron, 
LIP-ENS Lyon (2009 -- 2011). L'objectif était de concevoir et construire un intergiciel capable 
de gérer un environnement dans lequel la disponibilité des ressources change très rapidement. 
En particulier, cet intergiciel doit donner accès de manière fugace à des équipements de calculs 
très haute performance. Mes tâches ont concerné la conception et l'évaluation de l'ordonnanceur
travaillant en collaboration avec un système pair-à-pair utilisé pour recenser dynamiquement
les ressources disponibles.\\
\end{itemize}

\subsubsection{Niveau local}
\begin{itemize}


\item[$\bullet$]  \textbf{Fondateur  et  Responsable   du  thème  de  recherche}
  \emph{grilles   et   clouds}   au   sein   l'équipe   ICPS,   du   laboratoire
  Icube. L'effectif  actuel (14/5/2013)  du thème est:  1 Direteur  de recherche
  INRIA, 2 Maîtres de conférences, 1 PostDoc, 1 doctorant, 2 Masters recherche.
 
\item [$\bullet$]
Membre du conseil scientifique du département \emph{Expertise pour la recherche de l'UdS} (sept 2010--).
Le comité comprend 17 membres nommés, représentants d'équipes scientifiques, dont le rôle
est de piloter l'investissement en matière de calcul, et de promouvoir les projets 
d'intérêt scientifique par attribution de ressources.\\
\end{itemize}

\subsubsection{Jurys de thèse}

\begin{itemize}

\item[$\bullet$] 
Rapporteur de la thèse d'Adrian Muresan, \'Ecole Normale Supérieure de Lyon
(soutenance déc. 2012), \textit{Scheduling and deployment of large-scale applications on 
Cloud platforms},
rapporteur J. F. Méhaut (U. de Grenoble), 
encadrants F. Desprez (INRIA Rhône-Alpes) et E. Caron (ENS Lyon)\\
\item[$\bullet$] 
Rapporteur de la thèse de Sébastien Miquée, Univ. Franche-Comté (soutenance 
jan. 2012), \textit{Exécution d'applications parallèles en environnements 
hétérogènes et volatils~: déploiement et virtualisation},
rapporteur C. Cérin (U. Paris 13), 
encadrants R. Couturier et D. Laiymani (U. Franche-Comté)\\
\item[$\bullet$] 
Rapporteur de la thèse de Fabrice Dupros, Univ. Bordeaux 1 (soutenance déc. 2010), 
\textit{Contribution à la modélisation numérique de la propagation des ondes 
sismiques sur architectures multic{\oe}urs et hiérarchiques},
rapporteur S. Lanteri (INRIA Sophia-Antipolis), 
encadrants D. Komatitsch (U. Pau) et J. Roman (Institut Polytechnique de 
Bordeaux).\\

\item[$\bullet$] 
Examinateur de la thèse d'Heithem Abbès (soutenance déc. 2009), 
\textit{Approches de décentralisation de la gestion des ressources dans les 
Grilles}, rapporteurs Mohamed Jmaiel (Université de Sfax) et Franck Capello 
(INRIA-U. Urbana-Champain), encadrants Christophe Cérin (U. Paris 13) et 
Mohamed Jemni (École Supérieure des Sciences et Techniques de Tunis).
\end{itemize}




\subsection{Encadrements}

\subsubsection{PostDoc}
\begin{enumerate}
\item 10/2012--09/2013 : Postdoc financé par l'ANR SONGS. Le travail 
doit contribuer à la modélisation de stratégies d'allocation de ressources 
de cloud et à en permettre la simulation avec \textsc{SimGrid}.\\
\end{enumerate}

\subsubsection{Thèses}
\begin{enumerate}
\item 10/2011-- : encadrement d'Etienne Michon. Taux d'encadrement: 50\%,
avec Julien Gossa. Financement DGA. La thèse porte sur les problématiques
d'allocation de ressources de cloud côté client.\\

\item 02/2011-- : encadrement de Matthieu Kuhn. Financement ANR E2T2. 
Taux d'encadrement prévisionnel: 20\%. Co-encadrants Guillaume Latu pour 
l'informatique, Nicolas Crouseille (HDR) pour les mathématiques appliquées. 
La thèse porte sur la parallélisation de modèles 
numériques pour la simulation de plasmas de bord.\\
\end{enumerate}

\subsubsection{Stages de DEA/Master}
\begin{enumerate}
\item 2013 : Loic Huertas. Co-encadrement avec Marc Frincu.
Mémoire intitulé \textit{Stratégies d'allocation et d'ordonnancement de workflows sur Clouds},
soutenance prévue juin 2013.

\item 2013 : Vincent Kherbache. Co-encadrement avec Julien Gossa.
Mémoire intitulé \textit{Courtier de clouds IaaS côté client},
soutenance prévue juin 2013.

\item 2011 : Etienne Michon. Co-encadrement avec Julien Gossa. Mémoire 
intitulé  \textit{Allocation de ressources et ordonnancement côté client dans 
un environnement de Clouds}, soutenu 06/2011.
\end{enumerate}

\subsubsection{Autres}
Encadrement internship INRIA
\smallskip
\begin{itemize}
\item[$\bullet$]  {\it Data Management in P2P-MPI}, Jagdish Achara, B-Tech de 
LNMIIT Jaipur, Inde. 3 mois, mai-août 2009. 
\end{itemize}
~\\
Encadrement stage ENSIIE
\smallskip
\begin{itemize}
\item[$\bullet$]  {\it Experimentation de cluster virtualisé avec Nimbus}, 
Marien Ritzenthaler, ENSIIE 1A. 2,5 mois, juin-août 2010. 
\end{itemize}




%-------------------------------- CH A R G E S  ---------------------------------------------
\section{Principales charges d'intérêt collectif}

\begin{itemize}
\item[$\bullet$] Directeur délégué aux système d'information à l'EM Strasbourg (sept 2011--).\\
En charge d'un service de trois analystes-programmeurs. Mon rôle au sein de la direction
est de proposer des choix stratégiques pour les services à développer ou acquérir en matière 
de système d'information. Je planifie également la ré-organisation des services existants de 
l'établissment. Ces développements se font en concertation avec l'Université (Direction des 
Usages Numériques et Direction Informatique). Parmi les projets les plus importants~: 
implantation d'un CRM, gestion du processus de recrutement des vacataires, gestion des
étudiants à l'étranger.\\
		
\item[$\bullet$] Membre du \textbf{comité d'experts} (9 membres) pour la section 
		CNU 27 de l'Université de Strasbourg depuis 2010 (membre de la commission
                de spécialiste aupravant).
	\item Membre des comités de sélection:
	\begin{itemize}
		\item poste MC 210 UdS Réseaux et Protocole, 2010,
		\item poste MC 1207 Université de Franche-Comté, IUT Belfort Montbéliard, 2010.\\
	\end{itemize}


\item[$\bullet$] 
Membre du conseil d'administration de l'association SPECIF (Société des Personnels Enseignants et Chercheurs 
en Informatique de France) de 2010 à 2012. SPECIF est devenue la SIF (Société Informatique de France) depuis le 31/mai/2012.\\

\item[$\bullet$] 
Membre élu, collège A, du conseil scientifique de l'ENSIIE depuis juin 2012. 
\vspace{1cm}
\end{itemize}


% Pour faire apparaître une section biblio classique
%\bibliography{generalbiblio}


\end{document}

