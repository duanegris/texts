% Created 2013-07-16 Mar 12:01
\documentclass[11pt]{article}
\usepackage[utf8]{inputenc}
\usepackage[T1]{fontenc}
\usepackage{fixltx2e}
\usepackage{graphicx}
\usepackage{longtable}
\usepackage{float}
\usepackage{wrapfig}
\usepackage{soul}
\usepackage{textcomp}
\usepackage{marvosym}
\usepackage{wasysym}
\usepackage{latexsym}
\usepackage{amssymb}
\usepackage{hyperref}
\tolerance=1000
\usepackage{color}
\usepackage{listings}
\providecommand{\alert}[1]{\textbf{#1}}

\title{Viste au KIT, le 11/07/2013}
\author{Stephane Genaud}
\date{\today}
\hypersetup{
  pdfkeywords={},
  pdfsubject={},
  pdfcreator={Emacs Org-mode version 7.9.3f}}

\begin{document}

\maketitle

\setcounter{tocdepth}{3}
\tableofcontents
\vspace*{1cm}


\section{Participants :}
\label{sec-1}

\begin{itemize}
\item ENSIIE : Pierre Dossantos-Uzarralde, Stéphane Genaud, Martin Weil
\item KIT : Isabelle Hornik (Institut Franco-Allemand), Maren Daniell (ERASMUS Incoming coordinator), a PhD student qui représente Sebastian Kobbe.
\item contacts: Maren.Daniell@kit.edu, hornik@kit.edu
\end{itemize}
\section{Agenda de la rencontre}
\label{sec-2}
\subsection{Présentation de KIT par I. Hornik}
\label{sec-2-1}
\subsubsection{KIT est une Université Technique (l'une des 9 grandes en Allemagne TU9,  voir annexe)}
\label{sec-2-1-1}
\subsubsection{Politique de partenariat académique}
\label{sec-2-1-2}


Le DeFI (institut franco-allemand au sein du KIT) gère des double-diplômes
\begin{itemize}
\item avec 7 unviersités/écoles en France : X, Arts-et-métiers PartisTech, INSA Strasbourg et Lyon, Ensas Strasbourg, Phelma-Grenoble INP, UJF Grenoble.
\item En informatique avec INSA Lyon et UJF-ENSIMAG.
\item En Math avec ENSIMAG.
\item Echange avec X : de l'ordre de 2 étudiants par an.
\end{itemize}
  
\subsubsection{Exemple de Double-diplôme avec INSA Strasbourg :}
\label{sec-2-1-3}

\begin{itemize}
\item S1-S4 : classe prepas dans pays origine
\item S5-S7 : groupe franco-allemand en France
\item S8 stage
\item S9-11 : @KIT
\end{itemize}
(donc 1 semestre de plus qu'un parcours normal Master)
\subsubsection{Processus de mise ne place d'un double-diplôme : il faut une base scientifique, c-a-d}
\label{sec-2-1-4}

    trouver un professeur correspondant au KIT pour espérer monter un projet échange
    ou de double-diplôme.
\subsection{Présentation de la faculté Informatics}
\label{sec-2-2}


2600 students in Informatics, 690 in Information Management (sur les 5 années)

9 Institutes
\begin{itemize}
\item Theoretical Informatics (ITI)  (8 people)
\item Cryptographie und Security (IKS) (3 profs)
\item Centre for Applied Science of Law  (ZAR) (3 profs)
\item Program Structures and Data Organization (9 people) (IPD) -- multicore
\item Operating and Dialog Systems (3 profs) (IBDS) -- micro-kernel, virtualization, energy saving
\item Telematics (6 profs + 2 ) (TM)  -- Networks, sensors, mobility
\item Technical Informatics (ITEC) - architecture
\item Process Control, Automation and Robotics (IPR)
\item Anthropomatics (IFA)
\end{itemize}
\subsection{Présentation ENSIIE avec focus Strasbourg par S. Genaud}
\label{sec-2-3}

\begin{itemize}
\item présentation des spécificités du système français et ce qui distingue les écoles d'ingénieur
\item Isabelle Hornik connait très bien le système français, demande sur quel concours nous recrutons
\item présentation détaillée des matières enseignées.
\item Nous insistons sur l'importance des séjours à l'étranger et l'intérêt de faire des stages à l'étranger.
\end{itemize}
\subsection{Actions prévues}
\label{sec-2-4}


\begin{itemize}
\item 2 étudiants ENSIIE-Evry partent en échange Erasmus au KIT à partir de sept 2013
\item Martin Weil va faire une visite sur place de type tutorat pour avoir un retour sur d'éventuelles autres points de collaborations
\item Maren Daniell et Isabelle Hornik seront invitées à la journée d'information des RI se tenant en avril prochain.
\end{itemize}
\subsection{Discussion informelle}
\label{sec-2-5}


\begin{itemize}
\item il n'y a pas de sélectivé à l'entrée de l'Université en général (il peut y en avoir pour des raisons logisitiques)
  Le taux d'échec après la première année en informatique est de l'ordre de 50\%. Réputée difficile.
\item les étudiants de l'Université ne sont pas obligé de faire de stage en entreprise. Le parcours peut rester complètement théorique en dehors des projets dont l'importance est de plus en plus prise en compte.
\item En réponse à notre besoin de stages à l'étranger, 
  quelques entreprises intéressantes sur Karlsruhe ou la région : Michelin, Kärcher, L'Oreal, SAP.
\end{itemize}
\section{Annexe: Liste des TU9:}
\label{sec-3}



\begin{itemize}
\item RWTH Aachen (TU9, Exzellenzuniversität)
\item Technische Universität Berlin (TU9)
\item Technische Universität Braunschweig (TU9 )
\item Technische Universität Darmstadt (TU9)
\item Technische Universität Dresden (TU9, Exzellenzuniversität)
\item Leibniz Universität Hannover (TU9)
\item Karlsruher Institut für Technologie (TU9)
\item Technische Universität München (TU9, Exzellenzuniversität)
\item Universität Stuttgart (TU9 )
\end{itemize}

\end{document}
