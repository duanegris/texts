%\subsubsection{Thèses}
\begin{enumerate}

\item 10/2015--06/2019 : encadrement de Luke Bertot. Taux d'encadrement: 50\%,
  avec Julien Gossa. Financement ED. Thèse soutenue en juin 2019. \textit{Improving the simulation of IaaS Clouds}.\\

\item 10/2011--06/2015 : encadrement d'Etienne Michon. Taux d'encadrement: 50\%,
	  avec Julien Gossa. Financement DGA. Thèse soutenue juin 2015. \emph{Allocation dynamique sur cloud IaaS : allocation dynamique d’infrastructure de SI sur plateforme de cloud avec maîtrise du compromis coûts/performances}. Publications associées: \cite{michon2012,MichonGGFB13,icps-2016-fgcs}.\\

\item 02/2011--09/2014 : encadrement de Matthieu Kuhn. Financement ANR E2T2. 
Taux d'encadrement: 20\%. Co-encadrants Guillaume Latu pour 
l'informatique, Nicolas Crouseille (HDR) pour les mathématiques appliquées. Thèse soutenue sept. 2014. \textit{Calcul parallèle et méthodes numériques pour la simulation de plasmas de bords}.
Publications associées: \cite{icps-europar-2015,KuhnLGC13,ketterlin11}\\

\item 2004--2008 : encadrement de Choopan Rattanapoka. Taux d'encadrement: 100\%. 
Directeur de thèse: Catherine Mongenet. Thèse soutenue en avril 2008.
\textit{P2P-MPI: A Fault-tolerant Message Passing Interface Implementation 
for Grids} - rapporteurs : Franck Cappello (INRIA, Orsay) et Thilo Kielmann (Vrije 
Universiteit, Amsterdam). Choopan Rattanapoka a aujourd'hui un poste permanent 
d'assistant professor au Department of Eletronics Engineering Technology du King 
Mongkut's University of Technology, à Bangkok (Thailande).
Publications associées: 
\cite{icps-2005-155,icps-2007-182,icps-2007-185,
      icps-2008-188,icps-2008-193,icps-2009-214,
	icps-2009-217,icps-book}\\


\item 2001--2004 : co-encadrement d'Arnaud Giersch avec Frédéric Vivien. 
Taux de co-encadrement: $\sim$40\%. Directeur de thèse Guy-René Perrin.
Thèse soutenue en décembre 2004.\textit{Ordonnancement 
sur plates-formes hétérogènes de tâches partageant des données} - rapporteurs : Denis 
Trystram (INPG, Grenoble) et Henri Casanova (UCSD, San Diego). Arnaud Giersch a 
aujourd'hui un poste de maître de conférences à l'IUT d'informatique de Belfort, 
université de Franche-Comté.
Publications associées: \cite{icps-2002-62,icps-2003-75,icps-2004-125}\\


\item 2000--2006 : co-encadrement de Marc Grunberg avec Catherine Mongenet 
(inscrit en thèse parallèlement à sa fonction d'ingénieur d'études au Réseau 
National de Surveillance Sismique). Taux de co-encadrement: $\sim$70\%.
Directeurs de thèse: Catherine Mongenet et Michel Granet (Physicien, ULP).
Thèse soutenue en septembre 2006. \textit{Conception 
d'une méthode de maillage 3D parallèle pour la construction d'un modèle de Terre 
réaliste par la tomographie sismique} - rapporteurs : Thierry Priol (IRISA, Rennes) 
et Denis Trystram (INPG, Grenoble).
Marc Grunberg occupe toujours aujourd'hui un poste d'ingénieur d'études au Réseau 
National de Surveillance Sismique, \'{E}cole et Observatoire de Géophysique du Globe.
Publications associées: 
\cite{icps-2002-20,icps-2003-111,icps-2003-113,
      icps-2004-107,icps-2004-124,icps-2005-146,icps-2007-184}.\\



\end{enumerate}
