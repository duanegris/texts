\setlength{\tabcolsep}{5pt}
\noindent
\begin{tabular}{lp{13.7cm}}
\textsc{NOM}		&{\bf Stéphane \textsc{Genaud}}, \\
			& \texttt{\url{http://icube-icps.unistra.fr/index.php/Stéphane\_Genaud}}\\
			& Né le 10 mars 1969. Français. Marié, 2 enfants. \\
			& Pôle API, Blvd S. Brant,\\
			& F-67400 Illkirch\\
			& Tél. : 03 68 85 45 42\\ 
			& \texttt{genaud@unistra.fr}\\[5mm]
\hline
\textsc{Responsabilités}& 2013--2017: Direction de l'antenne de l'ENSIIE à Strasbourg.\\[2mm]
			& 2002--2007: Responsable du thème de recherche \textit{Grilles et Clouds}, équipe ICPS, laboratoire Icube.\\[2mm]
			& 2001-2007 : Responsable de la filière \textit{Système d'Information} du master Management International de l'IECS.\\[2mm]
\hline
\textsc{Emplois} 	& depuis 01/2014: Professeur des Universités à l'ENSIIE. Laboratoire Icube.\\
			& 09/2012 -- 12/2013: Maître de conférences Unistra mise à disposition de l'ENSIIE.\\[2mm]
			& 2009--2012 : Maître de conférences Unistra. Laboratoire LSIIT.\\[2mm]
 	                & 2007--2009: Détaché Chargé de Recherches à INRIA
                          Nancy-Grand Est, équipe-projet \textsc{AlGorille}.\\[2mm]
			& 1998--2007: Maître de conférences Université Robert Schuman de Strasbourg. Laboratoire LSIIT\\[2mm]
			& 1996--1998: ATER à l'IUT Robert Schuman, département informatique, Strasbourg.\\[2mm]
%			& 1993--1996: Doctorant à l'Université de Franche-Comté puis Louis Pasteur. Directeur de thèse: Guy-René Perrin.\\[2mm]
\hline
\textsc{Formation}	& 2009:  Habilitation à Diriger les Recherches en Informatique, Université Henri Poincaré Nancy.
      			  Titre: {\em Exécutions de programmes parallèles à passage de messages sur grille de calcul.}
   				 Rapporteurs: 	Christophe Cérin (Univ. Paris Nord), 
				   		   	Frédéric Desprez (INRIA),
							Thierry Priol (INRIA). 
							Examinateur: Pascal Bouvry (Univ. Luxembourg)
							Garant: Jens Gustedt (INRIA).\\[5mm]
%
			& 1997:  Doctorat en Sciences, mention Informatique, Université Louis Pasteur, Strasbourg.
      			  Titre: {\em Transformations de programmes \textsc{Pei} : applications au parall\'{e}lisme de donn\'{e}es}.
		              Rapporteurs: Luc Bougé (\'{E}cole Normale Supérieure Lyon) et Patrice Quinton (Université de Rennes). Examinateur : Christian Lengauer (Université Passau, Allemagne).\\[2mm]
			& 1993: Diplôme d'\'{E}tudes supérieures Spécialisées (DESS) en Informatique du parallélisme, 
				Université de Franche-Comté, Besançon. mention {\em très bien}.\\[2mm]
			& 1991: Bachelor of Sciences (BSc) in European Informatics, Sheffield Hallam University.\\[5mm]
%                 & 1989  & Diplôme Universitaire de Technologie (DUT) en informatique, Institut Universitaire de Technologies de Nantes, Université de Nantes.\\
%\end{tabular}

\end{tabular}

%------------- P H O T O ---------------------------
%\pgfdeclareimage[interpolate=true,width=2.3cm]{idphoto}{stephane_genaud_gray_picture}
%\pgfputat{\pgfxy(14.5,19)}{\pgfuseimage{idphoto}}
%          pgfxy(width,height) from


