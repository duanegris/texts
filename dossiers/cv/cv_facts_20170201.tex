

\setlength{\tabcolsep}{5pt}
\noindent
\begin{tabular}{lp{13.7cm}}
\textsc{NOM}		&{\bf Stéphane \textsc{Genaud}}, \\
			& \texttt{\url{http://icube-icps.unistra.fr/index.php/Stéphane\_Genaud}}\\
			& Né le 10 mars 1969. Français. Marié, 2 enfants. \\
			& Pôle API, Blvd S. Brant,\\
			& F-67400 Illkirch\\
			& Tél. : 03 68 85 45 42\\ 
			& \texttt{genaud@unistra.fr}\\[5mm]
\hline
\textsc{Responsabilités}& 2013--2017: Direction de l'antenne de l'ENSIIE à Strasbourg.\\[2mm]
			& 2002--2007: Responsable du thème de recherche \textit{Grilles et Clouds}, équipe ICPS, laboratoire Icube.\\[2mm]
			& 2001-2007 : Responsable de la filière \textit{Système d'Information} du master Management International de l'IECS.\\[2mm]
\hline
\textsc{Emplois} 	& depuis 01/2014: Professeur des Universités à l'ENSIIE. Laboratoire Icube.\\
			& 09/2012 -- 12/2013: Maître de conférences Unistra mise à disposition de l'ENSIIE.\\[2mm]
			& 2009--2012 : Maître de conférences Unistra. Laboratoire LSIIT.\\[2mm]
 	                & 2007--2009: Détaché Chargé de Recherches à INRIA
                          Nancy-Grand Est, équipe-projet \textsc{AlGorille}.\\[2mm]
			& 1998--2007: Maître de conférences Université Robert Schuman de Strasbourg. Laboratoire LSIIT\\[2mm]
			& 1996--1998: ATER à l'IUT Robert Schuman, département informatique, Strasbourg.\\[2mm]
%			& 1993--1996: Doctorant à l'Université de Franche-Comté puis Louis Pasteur. Directeur de thèse: Guy-René Perrin.\\[2mm]
\hline
\textsc{Formation}	& 2009:  Habilitation à Diriger les Recherches en Informatique, Université Henri Poincaré Nancy.
      			  Titre: {\em Exécutions de programmes parallèles à passage de messages sur grille de calcul.}
   				 Rapporteurs: 	Christophe Cérin (Univ. Paris Nord), 
				   		   	Frédéric Desprez (INRIA),
							Thierry Priol (INRIA). 
							Examinateur: Pascal Bouvry (Univ. Luxembourg)
							Garant: Jens Gustedt (INRIA).\\[5mm]
%
			& 1997:  Doctorat en Sciences, mention Informatique, Université Louis Pasteur, Strasbourg.
      			  Titre: {\em Transformations de programmes \textsc{Pei} : applications au parall\'{e}lisme de donn\'{e}es}.
		              Rapporteurs: Luc Bougé (\'{E}cole Normale Supérieure Lyon) et Patrice Quinton (Université de Rennes). Examinateur : Christian Lengauer (Université Passau, Allemagne).\\[2mm]
			& 1993: Diplôme d'\'{E}tudes supérieures Spécialisées (DESS) en Informatique du parallélisme, 
				Université de Franche-Comté, Besançon. mention {\em très bien}.\\[2mm]
			& 1991: Bachelor of Sciences (BSc) in European Informatics, Sheffield Hallam University.\\[5mm]
%                 & 1989  & Diplôme Universitaire de Technologie (DUT) en informatique, Institut Universitaire de Technologies de Nantes, Université de Nantes.\\
%\end{tabular}

\end{tabular}

%------------- P H O T O ---------------------------
%\pgfdeclareimage[interpolate=true,width=2.3cm]{idphoto}{stephane_genaud_gray_picture}
%\pgfputat{\pgfxy(14.5,19)}{\pgfuseimage{idphoto}}
%          pgfxy(width,height) from






\section{Synopsis}

%-------------------------------------------------------------------------------
% section expliquant le cheminement de carrière.
%-------------------------------------------------------------------------------
%\section{Synopsis}

J'ai mené une carrière de maître  de conférences à l'Université de Strasbourg de
1998 à 2013, avec une interruption de 2 ans en détachement à l'INRIA Nancy-Grand
Est (2007-2009).  J'ai été recruté sur  un poste de professeur des universités à
l'ENSIIE  en 2014  avec comme  affectation le  site Strasbourgeois  de l'ENSIIE.
J'ai dirigé l'antenne Strasbourgeoise de l'ENSIIE de 2014 jusqu'à sa fermeture à
l'été 2017. Par  la suite, il a  été convenu par l'Université  de Strasbourg que
mon expérience pouvait  contribuer au renforcement des  efforts de développement
de  l'enseignement en  informatique  à Strasbourg.   Pour  cela, mon  employeur,
L'ENSIIE, et  l'Université de Strasbourg ont  signé un contrat annuel  de mise à
disposition,  renouvellé pour  la troisième  fois en  cette année  universitaire
2019-2020.   Le contrat  mentionne en  substance que  mon emploi  à l'Unistra  a
vocation à être pérennisée.\\

Mon  activité de  recherche  s'est  déroulée de  manière  continue durant  cette
période au  sein de l'équipe ICPS  du laboratoire d'Icube, y  compris pendant la
période  ou  j'avais  la  charge  administrative de  l'antenne  de  l'ENSIIE.  À
l'intérieur de mon équipe, j'étais en  charge du thème de recherche \emph{Cloud}
dont la problématique est la gestion des ressources de calcul de type cloud pour
les applications scientifiques. Le dernier  doctorant travaillant sur le thème à
soutenu sa thèse  en juin 2019. 

L'autre thème  de l'ICPS (qui constitue  la plus grosse partie  de l'équipe) est
également  une  équipe-projet INRIA  nommée  Camus.   Son objectif  initial  est
l'optimisation  des programmes  pour  les architectures  multi-core.  Depuis  le
printemps  2019,  nous  élaborons  un  projet  scientifique  pour  une  nouvelle
équipe-projet INRIA.  Cette redéfinition du  projet scientifique est normale car
Camus  arrive à  sa  durée limite  d'existence et  des  nouveaux chercheurs  ont
rejoint  l'équipe (un  directeur  de  recherches et  deux  chargés de  recherche
récemment).   Mon objectif  est  de  m'inscrire dans  ce  nouveau  projet et  je
participe activement à l'élaboration de la nouvelle proposition.





\section{Enseignements}

Je suis mis à  disposition de l'Université de Strasbourg par l'ENSIIE. Je suis 
rattaché principalement à l'UFR de Mathématiques et Informatique mais j'interviens
dans deux composantes :
\begin{itemize}
\item UFR Mathématiques et Informatique, pour laquelle je réalise environ
  deux-tiers de mon service,
\item Télécom Physique Strasbourg, pour un tiers de mon service.
\end{itemize}

\bigskip
Je lise dans  la table ci-dessous les enseignements effectués  dans le cadre de
cette mise à disposition pour l'année universitaire 2019-2020. J'ai bien entendu
assuré de  nombreux autres cours par  le passé.  Les cours  et volumes effectués
les années 2017-2018 et 2018-2019 sont très similaires à cette liste.

\begin{table}[hbt]
\label{tb:ens}
\begin{center}
\begin{tabular}{ccr}						
Niveau  & Intitulé Cours & heures (eqTD) \\
\hline
\multicolumn{3}{c}{\textit{UFR Mathématiques Informatique}}\\
\hline
Licence 1 & Algo et Programmation & 	71.67 \\
Licence 3 & Programmation Parallèle &	5.33 \\
Master 2 SIRIS & Cloud \& Virtualisation &	20.00 \\
Master 2 SDSC & Traitement de données réparties	&	15.83 \\
Master 2 SIL  & Traitement  large échelle & 15.83 \\
\hline
sous-total UFR &                 & 128.67 \\
& & \\ 
\hline		
\multicolumn{3}{c}{\textit{Télécom Physique Strasbourg}}\\
\hline
dipl. IR & Initiation Système Exploitation &        19.25 \\
dipl. IR &  Programmation Orientée Objet (C++,Java) &	42.00 \\
dipl. Gén. &Programmation parallèle	       &	12.00 \\
\hline
sous-total TPS	       &       & 73.25 \\
\end{tabular}	
\caption{Service prévisionnel à l'Unistra 2019-2020} 
\end{center}
\end{table}
Les éléments saillants de ces interventions sont:
\begin{itemize}
\item je contribue  au nouveau diplôme \emph{informatique et  réseaux} (dipl. IR
  dans le tableau) depuis sa création par Télécom Physique Strasbourg.
\item je me suis impliqué dans des cours de Master de l'offre de formation renouvellée
qui traitent de sujets ``en vogue'' dans le monde industriel : j'ai pris la
responsabilité des cours \emph{clouds et virtualisation} (cloud) et \emph{traitements
  large échelle} (big data).
\end{itemize}\\

\medskip
D'autre   part,  j'ai   contribué   à  la   construction   du  nouveau   diplôme
\emph{Informatique et  Réseaux} de  Télécom Physique  Strasbourg, dont  une part
importante des enseignements  s'appuient sur les Masters  en informatique opérés
par l'UFR de Mathématique et Informatique.  J'ai animé en 2017-2018 un groupe de
travail qui a  produit une convention de coopération pédagogique  entre les deux
composantes pour la mise en place de ce diplôme d'ingénieur.



%------------------- avant MAD -----------------------------
\begin{comment}

\section{Activités d'enseignement}

\label{sc:ensgnt-univ}

\begin{itemize}

\item[$\diamond$] \`A la rentrée 2009, maître de conférences en informatique rattaché à l'EM%
\footnote{Ecole de Management, établissement créé fin 2007 par la fusion des établissement IECS et IAE de l'Université R. Schuman}, 
Université de Strasbourg. Service d'enseignement partagé entre les établissements:
\begin{itemize}
	\item EM Strasbourg
	\item ENSIIE\footnote{Ecole Nationale Supérieure d'Informatique pour l'Industrie et l'Entreprise}, antenne de Strasbourg. Responsable du cours \textit{Systèmes Informatiques}, première année.
      \item UFR mathématiques-informatique. \textit{Systèmes distribués, calcul parallèles, grilles}, Master ILC M2. \\[2mm]
\end{itemize}

\item[$\diamond$] Maitre de conférences en informatique à l'IECS, université Robert Schuman, Strasbourg (1998-2007).\\
	Principaux cours assurés en Master International à la Gestion:
\begin{itemize}
	\item Technologies des systèmes d'information
	\item Algorithmique - Programmation (application avec Java)
	\item Architecture des applications web (architecture, JavaScript, PHP, base de données)
	\item Outils pour la gestion de projet
	\item Réseaux \\[2mm]
\end{itemize}

\item [$\diamond$]
Vacataire 
\begin{itemize}
\item DESS puis Master informatique pro Université Louis Pasteur (ULP). Cours de systèmes distribués. (2001-2007) 
\item Master recherche informatique ULP. Cours d'options {\em Grilles informatiques}. (2002-2003)
\item licence informatique ULP. TD du cours système d'exploitation. (1999-2001). 
\item CNAM de Strasbourg, filière informatique. 
cours de programmation parallèlele dans l'unité de valeur {\em conception et développement du logiciel} 
du cycle B (1/3 de l'UV). (1996-2000).
Cours magistral et TD (TD assurés avec Franco Zaroli).
\item école d'ingénieur \textit{ENSPS}. Outils de gestion de projets. (2000)
\item école d'ingénieur \emph{Ecole et Observatoire de Physique du Globe} (1997-1999).
Calcul parallèle. Généralités et application à des méthodes de résolution directe de systèmes d'équations 
linéaires en data-paralléle. 
\item DESS informatique du parallélisme (1993).\\[2mm]
\end{itemize}

\item [$\diamond$]
ATER plein-temps à l'IUT d'informatique de Strasbourg. (1996-1998). 
Principaux cours assurés: l'algorithmique, avec comme support la programmation en langage C et C++, 
et responsable de la première partie du cours système d'exploitation. \\[2mm]


\end{itemize}





\section{Responsabilités administratives}

%------------------------------------

\begin{itemize}
\item[$\bullet$] \textbf{Comissions Spécialistes / Comité d'experts}:
\begin{itemize}
\item 2004--2008: Membre de plusieurs commissions de spécialistes (section CNU 27).
	\begin{itemize}
		\item titulaire à l'Université Louis Pasteur (Strasbourg) entre 2004 et 2008,
		\item suppléant à l'Université de Franche-Comté (Besançon) entre 2005 et 2008,
		\item suppléant à l'Université Henri Poincaré (Nancy) entre 2006 et 2008.\\
	\end{itemize}
\item 2010-- \'Elu membre du comité d'experts (9 membres) pour la section 27 Université de Strasbourg en 2010.
Membre des comités de sélection:
	\begin{itemize}
		\item poste MC 210 UdS Réseaux et Protocole, 2010,
		\item poste MC 1207 Université de Franche-Comté, IUT Belfort Montbéliard, 2010.\\
	\end{itemize}
\end{itemize}



\item[$\bullet$] \textbf{Pédagogique}:
Responsable de la filière d'enseignement \emph{système d'information} à l'IECS, Strasbourg. (2001--2007)
\end{itemize}
\vspace{1cm}




\section{Animation scientifique}

\begin{itemize}
\item[$\bullet$] Obtention de la prime d'excellence scientifique (PES) à partir d'octobre 2009.
\end{itemize}

\subsection{Responsabilités éditoriales}
%------------------------------------
\begin{itemize}
\item[$\bullet$] 
Membre des comité de programmes des conférences internationales:
12th International Conference on Algorithms and Architectures for Parallel Processing (ICA3PP-12), 2012 (Fukuoka, Japan),
13th IEEE International Conference on High Performance Computing and Communications, 2011 (Banff, Canada),
IEEE/ACM International Conference on Grid Computing, 2008 (Tsukuba, Japon), et 2010 (Bruxelles, Belgique),
21th IASTED International Conference on Parallel and Distributed Computing and Systems, 2010 (Marina del Rey, USA),
International Symposium on Grid and Distributed Computing, 2008 (Hainan Island, Chine). 
\item [$\bullet$]
Membre du comité de rédaction de la revue Technique et Science Informatiques (2005--2009).
\item [$\bullet$]
Membre du comité scientifique du département \emph{Expertise pour la recherche de l'UdS} (sept 2010--).
Le comité comprend 17 membres nommés, représentants les équipes scientifiques les plus impliquées 
par rapport aux équipements de calcul de l'Université. Le rôle du comité est de piloter
l'investissement en matière de calcul, et de promouvoir les projets présentant le plus 
d'intérêt scientifique par attribution de ressources.
\end{itemize}


%-----------------A N I M A T I O N -------------------
\subsection{Activités scientifiques}

\subsubsection{Niveau international}
Membre des comité de programmes des conférences internationales:\\[-3mm]
\begin{itemize}
\item[$\bullet$]
15th IEEE International Conference on Computational Science and Engineering (CSE 2012) 
Cluster, Grid, Cloud and P2P Computing track. Paphos, Cyprus, October 3-5, 2012. 
http://www.cse2012.cs.ucy.ac.cy/

\item[$\bullet$] 
14th IEEE International Conference on High Performance Computing and Communications (HPCC 2012), 
2012 (Liverpool, England),
\item[$\bullet$] 
13th IEEE International Conference on High Performance Computing and Communications (HPCC 2011), 
2011 (Banff, Canada),
\item[$\bullet$] 
IEEE/ACM International Conference on Grid Computing (GRID'10), 2010 (Bruxelles, Belgique),
\item[$\bullet$] 
20th IASTED International Conference on Parallel and Distributed Computing and Systems, 2010, (Marina Del Rey, USA),
\item[$\bullet$] 
IEEE/ACM International Conference on Grid Computing (GRID'08), 2008 (Tsukuba, Japon), 
\item [$\bullet$]
International Symposium on Grid and Distributed Computing, 2008 (Hainan Island, Chine),\\
\end{itemize}

Relecteur pour de nombreuses revues ou conférences internationales: IEEE Trans. on Distr. and Parallel Systems, 
J. of SuperComputing, J. of Grid Computing, IEEE Conference on Grid Computing, IEEE CCGrid conference, Europar,
IEEE IPDPS conference, \ldots.


\subsubsection{Niveau national}
%------------------------------------
\begin{itemize}

\item[$\bullet$] Prime d'excellence scientifique (PES) de 2009 à 2012.\\


\item[$\bullet$]
\textbf{Porteur local} pour le projet ANR SONGS (ANR 11 INFR 013-03)  (taux déclaré 40\%)
coordonné par Martin Quinson, LORIA, Nancy (2012-2015)  poursuivant le projet 
Uss-SimGrid (voir ci-dessous). Le projet vise à affiner les objets modélisés pour la 
simulation (processeurs multi-c{\oe}urs, mémoire) ou en ajouter (disque, réseaux spécialisés
comme Infiniband) et à fournir des interfaces adaptées à la représentation de systèmes
complexes comme des machines HPC ou des Clouds. Je suis responsable du work package
sur les clouds.\\


\item[$\bullet$]
Participant au projet blanc ANR E2T2 (ANR 11 SIMI 9) (taux déclaré 15\%) coordonné par 
Peter Beyer, laboratoire PIIM, Université de Provence (2011-2014). L'objectif du projet 
est d'améliorer la modélisation physique des plasmas de bord dans un tokamak. Dans ce
projet, ma tâche est de co-encadrer un doctorant, Matthieu Kuhn avec Guillaume Latu et
Nicolas Crouseille (IRMA) pour paralléliser les codes développés par le CEA Cadarache 
(IRFM) et les physiciens du PIIM. \\


\item[$\bullet$]
\textbf{Co-animateur} d'une action d'animation scientifique 
dans le cadre de l'action de développement technologique (ADT) de Aladdin de l'INRIA, 
visant à pérenniser l'outil scientifique Grid5000. (07/2008--06/2012).\\


\item[$\bullet$]
Participant (taux déclaré 20\%) au projet ANR USS-SimGrid (ANR 08 SEGI 022) coordonné par 
Martin Quinson, LORIA, Nancy (2009 -- 2011). Ce projet a été labelisé projet \textit{phare}
par l'ANR.
L'objectif général du projet était  d'élargir les capacités 
de l'environnement de simulation SimGrid pour satisfaire des besoins plus divers, comme la
simulation de systèmes pair-à-pair ou d'environnements de calcul intensif.\\


\item[$\bullet$]
Participant (taux déclaré 20\%) au projet SPADES (ANR 08 SEGI 025) coordonné par Eddy Caron, 
LIP-ENS Lyon (2009 -- 2011). L'objectif était de concevoir et construire un intergiciel capable 
de gérer un environnement dans lequel la disponibilité des ressources change très rapidement. 
En particulier, cet intergiciel doit donner accès de manière fugace à des équipements de calculs 
très haute performance. Mes tâches ont concerné la conception et l'évaluation de l'ordonnanceur
travaillant en collaboration avec un système pair-à-pair utilisé pour recenser dynamiquement
les ressources disponibles.\\


\item[$\bullet$] Participant (10\%) au projet Masse de Données Astronomiques (ACI Masse de données) 
coordonné par Françoise Genova, observatoire de Strasbourg (2004 -- 2006). \\

\item[$\bullet$]
\textbf{Porteur d'un projet d'Action Concertée Incitative}. Projet TAG, 
pluridisciplinaire dans de l'ACI GRID (Globalisation des ressources informatiques et des données) 
du ministère de la recherche. Doté d'un budget de 182 K\euro{} et d'un poste d'ingénieur). 
(12/2001 -- 12/2003).\\

\end{itemize}

\end{itemize}
\end{comment}





\newpage
\section{Recherche}

La  thématique  générale  de  recherche  de  l'équipe  ICPS  étudie  les  moyens
d'améliorer la vitesse  d'exécution des calculs. Ceci  couvre plusieurs domaines
comme les modèles  de programmation, les outils  automatiques de parallélisation
et d'optimisation de code, ou la  gestion des ressources de calcul sous-jacentes.
Les architectures cibles représentent également un spectre large et en évolution
constante  :  elles  vont  du  multi-core   aux  cluster  HPC,  en  passant  par
l'utilisation de GPU, ou le recours à des ressources distribuées distantes comme
les grilles  ou les  clouds. Mon thème  de recherche est  actuellement lié  à ce
dernier type d'architecture.


\subsection{Thème principal}
Mon thème prinicpal de recherche a concerné, à partir de 2006, les architectures
distribuées émergentes que sont les grilles  de calcul puis le cloud computing à
partir  de 2010.   La ligne  directrice  a été  l'étude de  l'adéquation de  ces
architectures à  des calculs  parallèles, en  particulier pour  des applications
scientifiques.\\

Nous avons proposé des mécanismes d'équilibrage de charge prenant
en   compte  les   particularités   de  ces   architectures  caractérisées   par
l'hétérogénéité, notamment les  latences réseaux. Puis de  manière plus gloable,
nous avons proposé un framework (P2P-MPI%
\footnote{http://www.p2p-mpi.org}) destiné à rassembler dynamiquement des
ressources de calcul pour les besoins  d'une exécution. Ce projet de recherche a
abordé  les problématiques  de la  découverte des  ressources disponibles  (sous
forme de réseaux pair-à-pair) et de la tolérance aux pannes par des approches de
réplication des calculs.\\

Avec  le développement  des  data-centers  offrant des  services  de type  cloud
computing, nous avons ensuite étudié la gestion des ressources de calcul pour un
client de cloud  IaaS. Cette étude a débouché sur  le développement du framework
\emph{Schlouder}\footnote{http://schlouder.gforge.inria.fr}   qui    permet   de
soumettre l'exécution d'applications parallèles  ayant une structure de workflow
ou sac de tâches. Ces recherches abordent les problèmes d'ordonnancement de type
\emph{on-line} dans un contexte nouveau  car les services cloud introduisent une
dimension  économique   (le  coût  de  location   des  ressources).   L'objectif
d'ordonnancement  qui était  précédemment  la minimisation  du makespan  devient
bi-objectif : il faut  décider non seulement du lieu et  moment où s'exécute une
tâche de calcul, mais également du  nombre de ressources louées, nombre qui peut
évoluer au  cours de  l'exécution. De fait,  les ordonnancements  deviennent des
compromis  entre le  coût  et  le makespan,  parmi  lesquels l'utilisateur  doit
choisir.

La comparaison des  meilleures solutions (telles que  perçues par l'utilisateur)
étant complexe, nous  avons développé la posssibilité de  simuler les exécutions
produites  par  les ordonnancements  de  Schlouder.   Ainsi, l'utilisateur  peut
obtenir  quasi-instantanément une  prédiction du  temps et  coût de  l'exécution
ordonnancée avec telle  ou telle méthode. Ce travail constitue  un pan entier de
recherche sur  la précision des  simulateurs de systèmes distribués.   Nous nous
appuyons sur SimGrid\footnote{http://simgrid.org}, dont  nous avons contribué au
développement au cours de deux projets ANR afin que SimGrid puisse modéliser les
clouds IaaS. De larges campagnes expérimentales comparant des exécutions réelles
aux modèles utilisés en simulation ont permis de mieux comprendre les facteurs
influents dans la précision des prédictions.

Enfin,  nous avons  abordé le  problème de  la variabilité  inhérente à  ce type
d'architecture.  Pour  cela, nous  avons proposé  d'enrichir les  prédictions en
donnant des  encadrements probabilistes du  makespan et  du cout plutôt  que des
valeurs  scalaires,  en  l'aide  d'une   méthode  basée  sur  la  simulation  de
Monte-Carlo.


\subsection{Thème secondaire}

Je  participe au  thème tranvserse  de l'équipe  qui concerne  les applications.
Cette  activité s'appuie  sur des  collaborations amenant  des applications  sur
lesquelles nous testons nos  propositions de recherche. Plusieurs collaborations
ont fait  l'objet de travaux  longs de  plusieurs années :  tomographie sismique
avec  l'IPGS\footnote{Institut de  Physique du  Globe de  Strasbourg (UMR7516)},
analyse  protéomique  avec  l'IPHC\footnote{Institut  Pluridisciplinaire  Hubert
  Curien   (UMR7178),},   simulation  pour   la   physique   des  plasmas   avec
l'IRMA\footnote{Institut de Recherche Mathématique Avancée, UMR7501} et d'autres
partenaires à travers un projet ANR.



\section{Conclusion}

Mes perspectives professionnelles sont celles d'une intégration à l'université
de Strasbourg, portées par le projet d'une intégration en recherche à une équipe
INRIA, et à une prise de responsabilités pour le développement de l'enseignement
en informatique.\\

Du point de vue de la recherche, j'ai eu une activité continue au sein de l'ICPS
d'Icube dont je  suis directeur-adjoint. Des perspectives  de recherche émergent
suite  à l'arrivée  de  nouveaux  personnels. Avec  Bérenger  Bramas, chargé  de
recherches  INRIA recruté  en  2018,  nous avons  proposé  des  sujets de  thèse
(candidats non retenus pour financement) sur la parallélisation à base de tâches
de programmes  C++. Cet axe  fera partie de  la proposition d'un  nouveau projet
INRIA.\\

Du  point  de vue  de  l'enseignement  et  des responsabilités  collectives,  je
souhaite faire profiter  les composantes de mon expérience  de directeur d'école
d'ingénieur.   Dans  ce  cadre,  j'ai  établi des  relations  proches  avec  les
collectivités territoriales, qui peuvent aider les institutions académiques dans
la perspective  d'apporter au tissu  économique local  les ressources dont  il a
besoin (spécialistes formés et innovation).  J'ai également établi de nombreuses
relations  avec   les  entreprises   locales  (contrat  de   parainage,  stages,
intervenants d'entreprise, organisation  de forum étudiants-entreprises). Enfin,
j'ai   travaillé  à   intégrer   l'antenne  de   l'ENSIIE   à  Strasbourg   dans
l'environnement académique strasbourgeois en favorisant les passerelles pour les
étudiants à travers  des conventions de coopération  spécifiques aux composantes
(Télécom Physique  Strasbourg, les  département informatique et  mathématique de
l'UFR de Mathématique et Informatique, et  la faculté de sciences économiques et
gestion).

\section{Annexes}
\subsection{Publications}

%------------------------ P U B L I C A T I O N S -----------------------------|

\small
\bibliographystyle{plain}
\begin{thebibliography}{99}

\subsection*{Thèses}

\bibitem{hdr}
\textbf{Stéphane Genaud}.
\newblock 
{\em Exécutions de programmes parallèles à passage de messages sur grille de 
calcul}.
\newblock 
Habilitation à diriger des recherches de l'université Henri Poincaré, 
Nancy. Décembre 2009.
\newblock 
Rapporteurs : C. Cérin (Paris 13), F. Desprez (INRIA Rhônes-Alpes), 
T. Priol (INRIA Bretagne-Atlantique).\\[2mm]

\bibitem{icps-1997-4}
\textbf{Stéphane Genaud}.
\newblock {\em Transformations de programmes \textsc{Pei} : applications au
  parallélisme de données}.
\newblock Thèse de doctorat de l'université Louis Pasteur, Strasbourg, Janvier 1997.
\newblock Rapporteurs : Luc Bougé et Patrice Quinton. 

\subsection*{Chapitre de livre}

\bibitem{icps-book}
\textbf{Stéphane Genaud} et Choopan Rattanapoka.
\newblock 
\emph{A Peer-to-Peer Framework for Message Passing Parallel Programs.}
\newblock 
Parallel Programming and Applications in Grid, P2P and Network-based System,
in {\em Advances In Parallel Computing} Series. Editor G. R. Joubert.
IOS Press, juin 2009. 
 

\subsection*{Articles en revues internationales}

\setlength{\itemsep}{1.5mm}

\bibitem{icps-2016-fgcs}
\newblock Etienne Michon, Julien Gossa, \textbf{Stéphane Genaud}, Léo Unbekandt, et Vincent Kherbache.
\newblock Schlouder: a broker for IaaS clouds
\newblock {\em Future Generation Computer Systems}, Elsevier, Future Generation Comp. Syst. 69: 11-23 (2017), sept 2016.


\bibitem{icps-2015-ijcc}
\newblock Marc Frincû, \textbf{Stéphane Genaud} et Julien Gossa.
\newblock Client-Side Resource Management on the Cloud: Survey and Future Directions.
\newblock {\em International Journal of Cloud Computing}, Inderscience, 4(3), octobre 2015.


\bibitem{icps-2014-fgg-computer}
\newblock Marc Frincû, \textbf{Stéphane Genaud} et Julien Gossa.
\newblock On the efficiency of several VM provisioning strategies for workflows
with multi-threaded tasks on clouds.
\newblock {\em Computing}, Springer, published online, juin 2014.

%@article{
%year={2014},
%issn={0010-485X},
%journal={Computing},
%doi={10.1007/s00607-014-0410-0},
%title={On the efficiency of several VM provisioning strategies for workflows with multi-threaded tasks on clouds},
%url={http://dx.doi.org/10.1007/s00607-014-0410-0},
%publisher={Springer Vienna},
%keywords={Workflow scheduling; Virtual machine provisioning; Cloud computing; Cost and makespan modeling; 68M14; 68M20},
%author={Frincu, MarcE. and Genaud, Stéphane and Gossa, Julien},
%pages={1-28},
%language={English}
%}


\bibitem{icps-2009-217}
\newblock \textbf{Stéphane Genaud}, Emmanuel Jeannot et Choopan Rattanapoka.
\newblock Fault-Management in P2P-MPI.
\newblock {\em International Journal of Parallel Programming}, Springer, 
37(5):433--461, août 2009.


\bibitem{icps-2008-188}
\textbf{Stéphane Genaud}, Pierre Gançarski, Guillaume Latu, Alexandre Blansché, 
Choopan Rattanapoka et Damien Vouriot. \newblock Exploitation of a parallel 
clustering algorithm on commodity hardware with P2P-MPI.
\newblock 
{\em The Journal of SuperComputing}, Springer, 43(1):21--41, jan. 2008.


\bibitem{icps-2007-182}
\textbf{Stéphane Genaud} et Choopan Rattanapoka.
\newblock P2P-MPI: A Peer-to-Peer Framework for Robust Execution of Message 
Passing Parallel Programs on Grids.
\newblock {\em Journal of Grid Computing}, Springer, 5(1):27--42, mai 2007.


\bibitem{icps-2004-125}
\textbf{Stéphane Genaud}, Arnaud Giersch, et Frédéric Vivien.
\newblock Load-balancing scatter operations for Grid computing.
\newblock {\em Parallel Computing}, Elsevier, 30(8):923--946, août 2004.

\bibitem{icps-2004-107}
Marc Grunberg, \textbf{Stéphane Genaud} et Catherine Mongenet.
\newblock Seismic ray-tracing and Earth mesh modeling on various parallel
  architectures.
\newblock 
{\em The Journal of Supercomputing}, Kluwer, 29(1):27--44, juillet 2004.


\subsection*{Articles en revues nationales}
\bibitem{icps-2005-146}
\textbf{Stéphane Genaud} et Marc Grunberg. 
\newblock  Calcul de rais en tomographie sismique : exploitation sur la grille.
\newblock {\em Technique et Science Informatiques}, numéro spécial Renpar, 
Hermès-Lavoisier, 24(5), pages 591--608, décembre 2005.

\bibitem{icps-1996-2}
\textbf{Stéphane Genaud}.
\newblock Transformations d'énoncés \textsc{Pei}.
\newblock {\em Technique et Science Informatiques}, 15(5), pages 601--618, 
Hermès, avril 1996.


\subsection*{Conférences internationales avec actes et comité de lecture}

\bibitem{icps-europar-2018}
Luke Bertot, Stéphane Genaud, Julien Gossa.
\newblock Improving Cloud Simulation Using the Monte-Carlo Method. Euro-Par 2018.
\newblock Euro-Par, Turin, Italy, Springer, LNCS, sept. 2018 

\bibitem{icps-ccgrid-2018}
Luke Bertot, Stéphane Genaud, Julien Gossa:
\newblock An Overview of Cloud Simulation Enhancement Using the Monte-Carlo Method. 
\newblock CCGrid, Washington, mai 2018.



\bibitem{icps-europar-2015}
Matthieu Kuhn, Guillaume Latu, Nicolas Crouseilles et \textbf{Stéphane Genaud}
\newblock Parallelization of an advection-diffusion problem arising in edge plasma physics using hybrid MPI/OpenMP programming.
\newblock Euro-Par, Vienne, Austria, Springer, LNCS, janvier 2015.

\bibitem{smpi13}  Paul  Bédaride,  Augustin Degomme,  \textbf{Stéphane  Genaud},
  Arnaud  Legrand,  George  S.  Markomanolis, Martin  Quinson,  Mark  Stillwell,
  Frederic Suter  and Brice Videau.   
\newblock Toward Better Simulation  of MPI
  Applications on  Ethernet/TCP Networks.  
\newblock 4th  International Workshop
  on  Performance  Modeling, Benchmarking  and  Simulation  of high  performance
  computing  systems,  part  of  SuperComputing   13.  Denver,  USA,  nov  2013.
\newblock \small{\textit{(taux: 30\%)}}

\bibitem{MichonGGFB13}
Etienne Michon, Julien Gossa, \textbf{Stéphane Genaud}, Marc Frincu and Alexandre Burel.
\newblock Porting Grid applications to the Cloud with Schlouder.
\newblock 5th IEEE International Conference on Cloud Computing Technology and Science (CloudCom 2013). Bristol, UK, dec 2013.
\newblock \small{\textit{(papiers acceptés/soumis{:}337/60(+34), taux: 17.8\%)}}

\bibitem{KuhnLGC13}
Matthieu Kuhn, Guillaume Latu, Stéphane Genaud and Nicolas Crouseilles.
\newblock Optimization and parallelization of Emedge3D on shared memory architecture.
\newblock HPCSP 2013: Workshop on HPC for Scientific Problems, Timisoara, Romania, sept 2013.


\bibitem{FrincuGG13}
Marc Frincu, \textbf{Stéphane Genaud} et Julien Gossa.
\newblock Comparing Provisioning and Scheduling Strategies for Workflows on Clouds
\newblock {\em 2nd IEEE International Workshop on Workflow Models, Systems, Services
and Applications in the Cloud (CloudFlow), IPDPS 2013}, mai 2013.

\bibitem{michon2012}
Etienne Michon, Julien Gossa, \textbf{Stéphane Genaud}.
\newblock Free elasticity and free CPU power for scientific workloads on IaaS Clouds
\newblock {\em 18th IEEE International Conference on Parallel and Distributed Systems}, 
IEEE, déc. 2012.
\newblock \small{\textit{(papiers acceptés/soumis:87/294, taux: 29\%)}}


\bibitem{icps-2011-225}
\newblock \textbf{Stéphane Genaud} et Julien Gossa,
\newblock Cost-wait Trade-offs in Client-side Resource Provisioning with 
Elastic Clouds.
\newblock {\em 4th IEEE International Conference on Cloud Computing (CLOUD 
2011)}, juillet 2011.
\newblock \small{\textit{(papiers acceptés/soumis{:}36/198, taux: 18\%)}}


\bibitem{icps-2011-224}
\newblock Pierre-Nicolas Clauss, Mark Stillwell, \textbf{Stéphane Genaud}, 
Fr\'ed\'eric Suter, Henri Casanova and  Martin Quinson.
\newblock Single Node On-Line Simulation of MPI Applications with SMPI.
\newblock {\em 25th IEEE International Parallel \& Distributed Processing 
Symposium (IPDPS 2011)}, mai 2011.
\newblock \small{\textit{(papiers acceptés/soumis{:}112/571, taux: 19\%)}}

\bibitem{icps-2009-219}
\newblock Virginie Galtier, \textbf{Stéphane Genaud} et Stéphane Vialle.
\newblock Implementation of the AdaBoost Algorithm for Large Scale Distributed 
Environments: Comparing JavaSpace and MPJ.
\newblock {\em 15th IEEE International Conference on Parallel and Distributed Systems}, 
IEEE, déc. 2009.
\newblock \small{\textit{(papiers acceptés/soumis:91/305, taux: 29\%)}}


\bibitem{icps-2009-214}
\textbf{Stéphane Genaud} and Choopan Rattanapoka.
\newblock Evaluation of Replication and Fault Detection in P2P-MPI.
\newblock 
{\em 6th IEEE International Workshop on Grid Computing (HPGC), IPDPS 2009}, 
mai 2009.
\newblock \textit{(Papier invité)}.

\bibitem{icps-2008-193}
\textbf{Stéphane Genaud} and Choopan Rattanapoka. 
\newblock Large-Scale Experiment of Co-allocation Strategies for Peer-to-Peer 
Supercomputing in P2P-MPI,
\newblock 
{\em 5th IEEE International Workshop on Grid Computing (HPGC), IPDPS 2008}, 
avril 2008.

\bibitem{icps-2007-192}
Ludovic Hablot and Olivier Glück and Jean-Christophe Mignot and \textbf{Stéphane Genaud} and Pascale Vicat-Blanc Primet.
\newblock Comparison and tuning of MPI implementation in a grid context.
\newblock {\em Proceedings of 2007 IEEE International Conference on Cluster Computing (CLUSTER)}, 458--463, september 2007.
\newblock \small{\textit{(papiers acceptés/soumis:42/106, taux: 39\%)}}

\bibitem{icps-2007-185}
\newblock \textbf{Stéphane Genaud} et Choopan Rattanapoka.
\newblock Fault Management in {\pmpi}. 
\newblock International Conference on {\em Grid and Pervasive Computing, 
(GPC 2007)}, LNCS, Springer, mai 2007.
\newblock \small{\textit{(papiers acceptés/soumis:56/217, taux: 25\%)}}
%submitted papers: 217; accepted : 56 full papers, 12 oral-short papers;  acceptance rate: 25\%

\bibitem{icps-2007-184}
\newblock \textbf{Stéphane Genaud}, Marc Grunberg et Catherine Mongenet.
\newblock Experiments in running a scientific {MPI} Application on GRID'5000. 
\newblock distingué par le \textsc{Intel} \textit{best paper award}.
\newblock {\em 4th IEEE International Workshop on Grid Computing (HPGC), IPDPS 
2007}, mars 2007.


\bibitem{icps-2005-155}
\textbf{Stéphane Genaud} et Choopan Rattanapoka.
\newblock A Peer-to-peer Framework for Robust Execution of Message Passing 
Parallel Programs.
\newblock 
In {\em EuroPVM/MPI 2005}, LNCS 3666, Springer-Verlag, pages 276--284, 
septembre 2005.
\newblock \small{\textit{(papiers acceptés/soumis:61/126, taux: 48\%)}}


\bibitem{icps-2004-124}
Marc Grunberg, \textbf{Stéphane Genaud}, et Catherine Mongenet.
\newblock Parallel adaptive mesh coarsening for seismic tomography.
\newblock In {\em SBAC-PAD 2004, 16th Symposium on Computer Architecture and
  High Performance Computing}. IEEE Computer Society Press, octobre 2004.
\newblock \small{\textit{(papiers acceptés/soumis:32/93, taux: 34\%)}}

\bibitem{icps-2003-75}
\textbf{Stéphane Genaud}, Arnaud Giersch, et Frédéric Vivien.
\newblock Load-balancing scatter operations for Grid computing.
\newblock In {\em Proceedings of 12th Heterogeneous Computing Workshop 
(HCW), IPDPS 2003}. IEEE Computer Society Press, avril 2003.

\bibitem{icps-2002-62}
Romaric David, \textbf{Stéphane Genaud}, Arnaud Giersch, \'{E}ric Violard, et 
  Benjamin Schwarz.
\newblock Source-code transformations strategies to load-balance Grid
  applications.
\newblock In {\em International Conference on Grid Computing - GRID'2002}, 
LNCS 2536, pages 82--87. Springer-Verlag, novembre 2002.

\bibitem{icps-2002-20}
Marc Grunberg, \textbf{Stéphane Genaud}, et Catherine Mongenet.
\newblock Parallel seismic ray-tracing in a global {E}arth mesh.
\newblock In {\em Proceedings of the 2002 Parallel and Distributed Processing
  Techniques and Applications (PDPTA'02)}, pages 1151--1157, juin 2002.

\bibitem{icps-1997-3}
Eric Violard, \textbf{Stéphane Genaud} et Guy-René Perrin.
\newblock Refinement of data-parallel programs in pei.
\newblock In {\em IFIP Working Conference on Algorithmic Language and Calculi}, 
R.~Bird and L.~Meertens editors, Chapman \& Hall~Ed., février 1997.
\newblock 25 pages.

\bibitem{icps-1995-1}
\textbf{Stéphane Genaud}, Eric Violard, et Guy-René Perrin.
\newblock Transformation techniques in \textsc{Pei}.
\newblock In P.~Magnusson S.~Haridi, K.~Ali, editor, {\em Europar'95}, LNCS
  966, pages 131--142. Springer-Verlag, août 1995.
\newblock \small{\textit{(papiers acceptés/soumis:50/180, taux: 27\%)}}




\subsection*{Conférences nationales avec actes et comité de lecture}
\bibitem{icps-2003-111}
Marc Grunberg et \textbf{Stéphane Genaud}.
\newblock Calcul de rais en tomographie sismique : exploitation sur la grille.
\newblock In {\em Renpar2003}, pages 179--186. INRIA, octobre 2003.

\bibitem{icps-1995-6}
\textbf{Stéphane Genaud}.
\newblock Techniques de tranformations d'énoncés \textsc{Pei} pour la
  production de programmes data-parallèles.
\newblock In {\em RenPar 7}, mai 1995, Mons, Belgique.

\bibitem{icps-1994-46}
Guy-René Perrin, Eric Violard et \textbf{Stéphane Genaud}.
\newblock \textsc{Pei} : a theoretical framework for data-parallel programming.
\newblock In {\em Workshop on Data-Parallel Languages and Compilers}, Lille, 
mai 1994.
\vspace{3mm}


\subsection*{Autres communications}

\bibitem{iphc-2011}
Christine Carapito, Jérôme Pansanel, Patrick Guterl, Alexandre Burel, Fabrice 
Bertile, \textbf{Stéphane Genaud}, Alain Van Dorsselaer, Christelle Roy.
\newblock Une suite logicielle pour la protéomique interfacée sur une grille de 
calcul. Utilisation d'algorithmes libres pour l'identification MS/MS, le 
séquençage de novo et l'annotation fonctionnelle.
\newblock Rencontres Scientifiques France Grilles 2011, Lyon.


\bibitem{ketterlin11}
Alain Ketterlin, \textbf{Stéphane Genaud}, Matthieu Kuhn.
\newblock Loop-Nest Recognition for the Extraction of Communication Patterns 
and the Compression of Message-Passing Parallel Traces.
\newblock Research Report ICPS 11-01. Université de Strasbourg. déc. 2011.


\bibitem{cds-2005}
A. Schaaff, F. Bonnarel, J.-J. Claudon, R. David, \textbf{S. Genaud}, M. Louys, 
C. Pestel and C. Wolf.
\newblock Work around distributed image processing and workflow management, 
\newblock ADASS 2005, Madrid.


\bibitem{icps-2003-113}
Marc Grunberg, \textbf{Stéphane Genaud}, et Michel Granet.
\newblock Geographical {ISC} data characterization with parallel ray-tracing.
\newblock In {\em Eos Trans. AGU, 84(46), Fall-Meeting Suppl., Abstract
  S31E-0793}, décembre 2003.

\end{thebibliography}


\subsection{Encadrements}
%\subsubsection{Thèses}
\begin{enumerate}

\item 10/2015--06/2019 : encadrement de Luke Bertot. Taux d'encadrement: 50\%,
  avec Julien Gossa. Financement ED. Thèse soutenue en juin 2019. \textit{Improving the simulation of IaaS Clouds}.\\

\item 10/2011--06/2015 : encadrement d'Etienne Michon. Taux d'encadrement: 50\%,
	  avec Julien Gossa. Financement DGA. Thèse soutenue juin 2015. \emph{Allocation dynamique sur cloud IaaS : allocation dynamique d’infrastructure de SI sur plateforme de cloud avec maîtrise du compromis coûts/performances}. Publications associées: \cite{michon2012,MichonGGFB13,icps-2016-fgcs}.\\

\item 02/2011--09/2014 : encadrement de Matthieu Kuhn. Financement ANR E2T2. 
Taux d'encadrement: 20\%. Co-encadrants Guillaume Latu pour 
l'informatique, Nicolas Crouseille (HDR) pour les mathématiques appliquées. Thèse soutenue sept. 2014. \textit{Calcul parallèle et méthodes numériques pour la simulation de plasmas de bords}.
Publications associées: \cite{icps-europar-2015,KuhnLGC13,ketterlin11}\\

\item 2004--2008 : encadrement de Choopan Rattanapoka. Taux d'encadrement: 100\%. 
Directeur de thèse: Catherine Mongenet. Thèse soutenue en avril 2008.
\textit{P2P-MPI: A Fault-tolerant Message Passing Interface Implementation 
for Grids} - rapporteurs : Franck Cappello (INRIA, Orsay) et Thilo Kielmann (Vrije 
Universiteit, Amsterdam). Choopan Rattanapoka a aujourd'hui un poste permanent 
d'assistant professor au Department of Eletronics Engineering Technology du King 
Mongkut's University of Technology, à Bangkok (Thailande).
Publications associées: 
\cite{icps-2005-155,icps-2007-182,icps-2007-185,
      icps-2008-188,icps-2008-193,icps-2009-214,
	icps-2009-217,icps-book}\\


\item 2001--2004 : co-encadrement d'Arnaud Giersch avec Frédéric Vivien. 
Taux de co-encadrement: $\sim$40\%. Directeur de thèse Guy-René Perrin.
Thèse soutenue en décembre 2004.\textit{Ordonnancement 
sur plates-formes hétérogènes de tâches partageant des données} - rapporteurs : Denis 
Trystram (INPG, Grenoble) et Henri Casanova (UCSD, San Diego). Arnaud Giersch a 
aujourd'hui un poste de maître de conférences à l'IUT d'informatique de Belfort, 
université de Franche-Comté.
Publications associées: \cite{icps-2002-62,icps-2003-75,icps-2004-125}\\


\item 2000--2006 : co-encadrement de Marc Grunberg avec Catherine Mongenet 
(inscrit en thèse parallèlement à sa fonction d'ingénieur d'études au Réseau 
National de Surveillance Sismique). Taux de co-encadrement: $\sim$70\%.
Directeurs de thèse: Catherine Mongenet et Michel Granet (Physicien, ULP).
Thèse soutenue en septembre 2006. \textit{Conception 
d'une méthode de maillage 3D parallèle pour la construction d'un modèle de Terre 
réaliste par la tomographie sismique} - rapporteurs : Thierry Priol (IRISA, Rennes) 
et Denis Trystram (INPG, Grenoble).
Marc Grunberg occupe toujours aujourd'hui un poste d'ingénieur d'études au Réseau 
National de Surveillance Sismique, \'{E}cole et Observatoire de Géophysique du Globe.
Publications associées: 
\cite{icps-2002-20,icps-2003-111,icps-2003-113,
      icps-2004-107,icps-2004-124,icps-2005-146,icps-2007-184}.\\



\end{enumerate}


%\subsubsection{Jurys de thèse}

\begin{itemize}

\item[$\bullet$] 
Examinateur de la thèse d'Anchen Chai, Université de Lyon (INSA),
(soutenance jan. 2019), \textit{Simulation Réaliste de l'exécution des
  applications déployées sur des systèmes distribués avec un focus sur
  l'amélioration de la gestion des fichiers},
rapporteur Radu Prodan (U. Klagenfurt), V. Breton (CNRS) 
encadrants F.Suter (CNRS) et S. Pop (CNRS)\\


\item[$\bullet$] 
Examinateur de la thèse de Ye Xia, Université Grenoble,
(soutenance dec. 2018), \textit{Combining Heuristics for Optimizing and Scaling
  the Placement of IoT Applications in the Fog},
rapporteur G. Pierre (U. Rennes 1), P. Sens (CNRS(U. Paris Sorbonne) 
encadrants T. Coupaye (Orange), X. Etchevers (Orange), F. Desprez (INRIA)\\

\item[$\bullet$] 
Rapporteur de la thèse de Jonathan Pastor, \'Ecole des Mines de Nantes
(soutenance oct. 2016), \textit{Contributions à la mise en place d'une
  infrastructure de Cloud Computing à large échelle},
rapporteur P. Sens (U. Paris Sorbonne), 
encadrants F. Desprez (INRIA Rhône-Alpes) et A. Lèbre (EMN)\\

\item[$\bullet$]
Examinateur de la thèse de Guillaume Laville, Université de Franche-Comté,
(soutenance juillet 2014), \textit{Exécution efficace de systèmes multi-agents
  sur GPU}
rapporteurs M. Krajecki (Univ. Reims), C. Cambier (Univ. Paris 6),
encadrants L. Philippe (Univ. Franche-Comté)\\

\item[$\bullet$]
Examinateur de la thèse d'Imen Chakroun, Université de Lille 1,
(soutenance juin 2013), \textit{Parallel heterogeneous Branch and Bound algorithms for multi-core and multi-GPU environments},
rapporteurs P. Manneback (Univ. de Mons), C. Roucairol (UVSQ),
encadrants N. Melad (Univ. Lille 1)\\

\item[$\bullet$] 
Rapporteur de la thèse d'Adrian Muresan, \'Ecole Normale Supérieure de Lyon
(soutenance déc. 2012), \textit{Scheduling and deployment of large-scale applications on 
Cloud platforms},
rapporteur J. F. Méhaut (U. de Grenoble), 
encadrants F. Desprez (INRIA Rhône-Alpes) et E. Caron (ENS Lyon)\\
\item[$\bullet$] 
Rapporteur de la thèse de Sébastien Miquée, Univ. Franche-Comté (soutenance 
jan. 2012), \textit{Exécution d'applications parallèles en environnements 
hétérogènes et volatils~: déploiement et virtualisation},
rapporteur C. Cérin (U. Paris 13), 
encadrants R. Couturier et D. Laiymani (U. Franche-Comté)\\
\item[$\bullet$] 
Rapporteur de la thèse de Fabrice Dupros, Univ. Bordeaux 1 (soutenance déc. 2010), 
\textit{Contribution à la modélisation numérique de la propagation des ondes 
sismiques sur architectures multic{\oe}urs et hiérarchiques},
rapporteur S. Lanteri (INRIA Sophia-Antipolis), 
encadrants D. Komatitsch (U. Pau) et J. Roman (Institut Polytechnique de 
Bordeaux).\\

\item[$\bullet$] 
Examinateur de la thèse d'Heithem Abbès (soutenance déc. 2009), 
\textit{Approches de décentralisation de la gestion des ressources dans les 
Grilles}, rapporteurs Mohamed Jmaiel (Université de Sfax) et Franck Capello 
(INRIA-U. Urbana-Champain), encadrants Christophe Cérin (U. Paris 13) et 
Mohamed Jemni (École Supérieure des Sciences et Techniques de Tunis).
\end{itemize}


