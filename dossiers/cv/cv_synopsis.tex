
%-------------------------------------------------------------------------------
% section expliquant le cheminement de carrière.
%-------------------------------------------------------------------------------
%\section{Synopsis}

J'ai mené une carrière de maître  de conférences à l'Université de Strasbourg de
1998 à 2013, avec une interruption de 2 ans en détachement à l'INRIA Nancy-Grand
Est (2007-2009).  J'ai été recruté sur  un poste de professeur des universités à
l'ENSIIE  en 2014  avec comme  affectation le  site Strasbourgeois  de l'ENSIIE.
J'ai dirigé l'antenne Strasbourgeoise de l'ENSIIE de 2014 jusqu'à sa fermeture à
l'été 2017. Par  la suite, il a  été convenu par l'Université  de Strasbourg que
mon expérience pouvait  contribuer au renforcement des  efforts de développement
de  l'enseignement en  informatique  à Strasbourg.   Pour  cela, mon  employeur,
L'ENSIIE, et  l'Université de Strasbourg ont  signé un contrat annuel  de mise à
disposition,  renouvellé pour  la troisième  fois en  cette année  universitaire
2019-2020.   Le contrat  mentionne en  substance que  mon emploi  à l'Unistra  a
vocation à être pérennisée.\\

Mon  activité de  recherche  s'est  déroulée de  manière  continue durant  cette
période au  sein de l'équipe ICPS  du laboratoire d'Icube, y  compris pendant la
période  ou  j'avais  la  charge  administrative de  l'antenne  de  l'ENSIIE.  À
l'intérieur de mon équipe, j'étais en  charge du thème de recherche \emph{Cloud}
dont la problématique est la gestion des ressources de calcul de type cloud pour
les applications scientifiques. Le dernier  doctorant travaillant sur le thème à
soutenu sa thèse  en juin 2019. 

L'autre thème  de l'ICPS (qui constitue  la plus grosse partie  de l'équipe) est
également  une  équipe-projet INRIA  nommée  Camus.   Son objectif  initial  est
l'optimisation  des programmes  pour  les architectures  multi-core.  Depuis  le
printemps  2019,  nous  élaborons  un  projet  scientifique  pour  une  nouvelle
équipe-projet INRIA.  Cette redéfinition du  projet scientifique est normale car
Camus  arrive à  sa  durée limite  d'existence et  des  nouveaux chercheurs  ont
rejoint  l'équipe (un  directeur  de  recherches et  deux  chargés de  recherche
récemment).   Mon objectif  est  de  m'inscrire dans  ce  nouveau  projet et  je
participe activement à l'élaboration de la nouvelle proposition.

