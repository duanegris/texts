
\setlength{\tabcolsep}{5pt}
\noindent
\begin{tabular}{lp{13.7cm}}
\textsc{NOM}		&{\bf Stéphane \textsc{Genaud}}, \texttt{{http://}www.loria.fr/\~{}genaudst}\\
				& Né le 10 mars 1969. Français. Marié, 2 enfants. \\
				  & 615, rue du Jardin Botanique,\\
				  & F-54600 Villers-lès-Nancy\\
				  & Tél. : 03 54 95 84 87\\ 
				 & \texttt{stephane.genaud@loria.fr}\\[5mm]

\textsc{Formation}	& 1997:  Doctorat en Sciences, mention Informatique, Université Louis Pasteur, Strasbourg.
      			  Titre: {\em Transformations de programmes \textsc{Pei} : applications au parall\'{e}lisme de donn\'{e}es}.
		              Rapporteurs: Luc Bougé (\'{E}cole Normale Supérieure Lyon) et Patrice Quinton (Université de Rennes). Examinateur : Christian Lengauer (Université Passau, Allemagne).\\[2mm]
			      & 1993: Diplôme d'\'{E}tudes supérieures Spécialisées (DESS) en Informatique du parallélisme, 
				  Université de Franche-Comté, Besançon. mention {\em très bien}.\\[2mm]
				& 1991: Bachelor of Sciences (BSc) in European Informatics, Sheffield Hallam University.\\[5mm]
%{\bf 1989}  & {\bf Diplôme Universitaire de Technologie (DUT)} en informatique, Institut Universitaire de Technologies de Nantes, Université de Nantes.\\
%\end{tabular}


\textsc{Emplois} 		& depuis sept. 2009: Maître de conférences à l'Université de Strasbourg. Membre du LSIIT (Laboratoire des Sciences de l'Image, de l'Informatique et de la Télédétection) unité mixte de recherche de l'Université de Strasbourg et du CNRS (UMR 7005),
équipe ICPS (Image et Caclul Parallèle Scientifique).\\[2mm]
 	                  & 2007--2009: Détaché Chargé de Recherches au LORIA, équipe projet 
				  INRIA \textsc{AlGorille}.\\[2mm]

				& 1998--2007: Maître de conférences à l'IECS, Université Robert Schuman de Strasbourg (aujourd'hui \'{E}cole de Management de Strasbourg).\\[2mm]
				& 1998--2007: Membre du LSIIT, UMR de l'Université Louis Pasteur et du CNRS (UMR 7005), 
équipe ICPS (Image et Caclul Parallèle Scientifique).\\[2mm]
				& 1996--1998: Attaché Temporaire d'Enseignement et de Recherches (ATER) au département d'informatique 
de l'IUT de Strasbourg Sud.\\[2mm]
				& 1994--1996: Doctorant à l'Université Louis Pasteur (Strasbourg). Directeur de thèse: Guy-René Perrin.\\[2mm]
				& 1993--1994: Doctorant à l'Université de Franche-Comté (Besançon). Directeur de thèse: Guy-René Perrin.\\[5mm]


\textsc{Responsabilités} & 2002--2007: Responsable du thème de recherche \textit{Grilles} dans l'équipe ICPS du LSIIT.\\[2mm]
				 & 2001-2007 : Responsable de la filière \textit{Système d'Information} du master Management International de l'IECS.\\[2mm]
\end{tabular}
%------------- P H O T O ---------------------------
\pgfdeclareimage[interpolate=true,width=2.3cm]{idphoto}{stephane_genaud_gray_picture}
\pgfputat{\pgfxy(14.5,19)}{\pgfuseimage{idphoto}}
%          pgfxy(width,height) from      



%---------------------------------------------------------------------------------------------
\newpage
\small
\bibliographystyle{plain}
\begin{thebibliography}{99}

\subsection*{Th\`{e}se}

\bibitem{icps-1997-4}
St\'{e}phane Genaud.
\newblock {\em Transformations de programmes \textsc{Pei} : applications au
  parall\'{e}lisme de donn\'{e}es}.
\newblock Thèse de doctorat de l'universit\'{e} Louis Pasteur, Strasbourg, Janvier 1997.
\newblock Rapporteurs : Luc Bougé et Patrice Quinton. Examinateur : Christian Lengauer.

\subsection*{Chapitre de livre}

\bibitem{icps-book}
St\'{e}phane Genaud et Choopan Rattanapoka.
\newblock \emph{A Peer-to-Peer Framework for Message Passing Parallel Programs.}
\newblock Parallel Programming and Applications in Grid, P2P and Network-based System,
in {\em Advances In Parallel Computing} Series. Editor Prof. Dr. Gerhard R. Joubert.
IOS Press, 2009. (à paraître)
 

\subsection*{Articles en revues internationales}

\setlength{\itemsep}{1.5mm}


\bibitem{icps-2009-217}
\newblock Stéphane Genaud, Emmanuel Jeannot and Choopan Rattanapoka.
\newblock Fault-Management in P2P-MPI.
\newblock {\em International Journal of Parallel Programming}, Springer, 37(5):433--461, août 2009.


\bibitem{icps-2008-188}
Stéphane Genaud, Pierre Gançarski, Guillaume Latu, Alexandre Blansché, Choopan Rattanapoka et Damien Vouriot.
\newblock Exploitation of a parallel clustering algorithm on commodity hardware with P2P-MPI.
\newblock {\em The Journal of SuperComputing}, Springer, 43(1):21--41, january 2008.


\bibitem{icps-2007-182}
Stéphane Genaud et Choopan Rattanapoka.
\newblock P2P-MPI: A Peer-to-Peer Framework for Robust Execution of Message Passing Parallel Programs on Grids.
\newblock {\em Journal of Grid Computing}, Springer, 5(1):27--42, may 2007.


\bibitem{icps-2004-125}
St\'{e}phane Genaud, Arnaud Giersch, et Fr\'{e}d\'{e}ric Vivien.
\newblock Load-balancing scatter operations for Grid computing.
\newblock {\em Parallel Computing}, Elsevier, 30(8):923--946, août 2004.

\bibitem{icps-2004-107}
Marc Grunberg, St\'{e}phane Genaud, et Catherine Mongenet.
\newblock Seismic ray-tracing and Earth mesh modeling on various parallel
  architectures.
\newblock {\em The Journal of Supercomputing}, Kluwer, 29(1):27--44, juillet 2004.


\subsection*{Articles en revues nationales}
\bibitem{icps-2005-146}
St\'{e}phane Genaud et Marc Grunberg. 
\newblock  Calcul de rais en tomographie sismique : exploitation sur la grille.
\newblock {\em Technique et Science Informatiques}, numéro spécial Renpar, 
Herm\`{e}s-Lavoisier, 24(5), pages 591--608, décembre 2005.

\bibitem{icps-1996-2}
St\'{e}phane Genaud.
\newblock Transformations d'\'{e}nonc\'{e}s \textsc{Pei}.
\newblock {\em Technique et Science Informatiques}, 15(5), pages 601--618, Herm\`{e}s, avril 1996.

\bibitem{icps-1993-69}
St\'{e}phane Genaud et Guy-Ren\'{e} Perrin.
\newblock Une exp\'{e}rience d'implantation d'un algorithme systolique sur
  hypercube.
\newblock {\em La Lettre du Transputer et des calculateurs parall\`{e}les},
  (17), mars 1993.


\subsection*{Conf\'{e}rences internationales avec actes et comité de lecture}

\bibitem{icps-2009-219}
\newblock Virginie Galtier, Stéphane Genaud et Stéphane Vialle.
\newblock Implementation of the AdaBoost Algorithm for Large Scale Distributed Environments:%
Comparing JavaSpace and MPJ.
\newblock {\em International Conference on Parallel and Distributed Systems}, IEEE, décembre 2009.


\bibitem{icps-2009-214}
Stéphane Genaud and Choopan Rattanapoka.
\newblock Evaluation of Replication and Fault Detection in P2P-MPI.
\newblock \textit{Invited Speaker}.
\newblock {\em 6th IEEE International Workshop on Grid Computing (HPGC), IPDPS 2009}, mai 2009.


\bibitem{icps-2008-193}
Stéphane Genaud and Choopan Rattanapoka. 
\newblock Large-Scale Experiment of Co-allocation Strategies for Peer-to-Peer Supercomputing in P2P-MPI,
\newblock {\em 5th IEEE International Workshop on Grid Computing (HPGC), IPDPS 2008}, avril 2008.

\bibitem{icps-2007-192}
Ludovic Hablot and Olivier Glück and Jean-Christophe Mignot and Stéphane Genaud and Pascale Vicat-Blanc Primet.
\newblock Comparison and tuning of MPI implementation in a grid context.
\newblock {\em Proceedings of 2007 IEEE International Conference on Cluster Computing (CLUSTER)}, 458--463, september 2007.


\bibitem{icps-2007-185}
\newblock Stéphane Genaud et Choopan Rattanapoka.
\newblock Fault management in {\pmpi}. 
\newblock International Conference on {\em Grid and Pervasive Computing, GPC 2007}, LNCS, Springer, mai 2007.


\bibitem{icps-2007-184}
\newblock Stéphane Genaud, Marc Grunberg et Catherine Mongenet.
\newblock Experiments in running a scientific MPI application on Grid'5000. 
\newblock Distingué par le \textsc{Intel} \textit{Best Paper Award}.
\newblock {\em 4th IEEE International Workshop on Grid Computing (HPGC), IPDPS 2007}, mars 2007.


\bibitem{icps-2005-155}
Stéphane Genaud et Choopan Rattanapoka.
\newblock A Peer-to-Peer Framework for Robust Execution of Message Passing Parallel Programs.
\newblock In {\em EuroPVM/MPI 2005}, LNCS 3666, Springer-Verlag, pages 276--284, septembre 2005.


\bibitem{icps-2004-124}
Marc Grunberg, St\'{e}phane Genaud, et Catherine Mongenet.
\newblock Parallel adaptive mesh coarsening for seismic tomography.
\newblock In {\em SBAC-PAD 2004, 16th Symposium on Computer Architecture and
  High Performance Computing}. IEEE Computer Society Press, octobre 2004.

\bibitem{icps-2003-75}
St\'{e}phane Genaud, Arnaud Giersch, et Fr\'{e}d\'{e}ric Vivien.
\newblock Load-balancing scatter operations for Grid computing.
\newblock In {\em Proceedings of 12th Heterogeneous Computing Workshop 
(HCW), IPDPS 2003}. IEEE Computer Society Press, avril 2003.

\bibitem{icps-2002-62}
Romaric David, Stéphane Genaud, Arnaud Giersch, \'{E}ric Violard, et 
  Benjamin Schwarz.
\newblock Source-code transformations strategies to load-balance Grid
  applications.
\newblock In {\em International Conference on Grid Computing - GRID'2002}, LNCS 2536,
  pages 82--87. Springer-Verlag, novembre 2002.

\bibitem{icps-2002-20}
Marc Grunberg, St\'{e}phane Genaud, et Catherine Mongenet.
\newblock Parallel seismic ray-tracing in a global {E}arth mesh.
\newblock In {\em Proceedings of the 2002 Parallel and Distributed Processing
  Techniques and Applications (PDPTA'02)}, pages 1151--1157, juin 2002.

\bibitem{icps-1997-3}
Eric Violard, St\'{e}phane Genaud et Guy-Ren\'{e} Perrin.
\newblock Refinement of data-parallel programs in pei.
\newblock In {\em IFIP Working Conference on Algorithmic Language and Calculi}, 
R.~Bird and L.~Meertens editors, Chapman \& Hall~Ed. ,
février 1997.
\newblock 25 pages.

\bibitem{icps-1995-1}
St\'{e}phane Genaud, Eric Violard, et Guy-Ren\'{e} Perrin.
\newblock Transformation techniques in \textsc{Pei}.
\newblock In P.~Magnusson S.~Haridi, K.~Ali, editor, {\em Europar'95}, LNCS
  966, pages 131--142. Springer-Verlag, août 1995.




\subsection*{Conf\'{e}rences avec actes et comité de lecture}
\bibitem{icps-2003-111}
Marc Grunberg et St\'{e}phane Genaud.
\newblock Calcul de rais en tomographie sismique : exploitation sur la grille.
\newblock In {\em Renpar2003}, pages 179--186. INRIA, octobre 2003.

\bibitem{icps-1995-6}
St\'{e}phane Genaud.
\newblock Techniques de tranformations d'\'{e}nonc\'{e}s \textsc{Pei} pour la
  production de programmes data-parall\`{e}les.
\newblock In {\em RenPar 7}, mai 1995, Mons, Belgique.

\bibitem{icps-1994-46}
Guy-Ren\'{e} Perrin, Eric Violard et St\'{e}phane Genaud.
\newblock \textsc{Pei} : a theoretical framework for data-parallel programming.
\newblock In {\em Workshop on Data-Parallel Languages and Compilers}, Lille, mai 1994.
\vspace{3mm}


\subsection*{Autres communications}


\bibitem{icps-2003-114}
St\'{e}phane Genaud.
\newblock Applications parallèles sur la grille: mieux vaut il être rapide ou résistant ? 
\newblock {\em Actes de GridUSe-2004}, Workshop "What we have learned", conférence invitée.
\newblock Supélec Metz, 24 juin 2004. 


\bibitem{icps-2003-113}
Marc Grunberg, St\'{e}phane Genaud, et Michel Granet.
\newblock Geographical {ISC} data characterization with parallel ray-tracing.
\newblock In {\em Eos Trans. AGU, 84(46), Fall-Meeting Suppl., Abstract
  S31E-0793}, décembre 2003.

\bibitem{icps-2002-50}
Romaric David, St\'{e}phane Genaud, Arnaud Giersch, \'{E}ric Violard, and
  Benjamin Schwarz.
\newblock Source-code transformations strategies to load-balance {G}rid
  applications.
\newblock Technical report, LSIIT-ICPS, Universit\'{e} Louis Pasteur, août
  2002.
\newblock RR 02-09.


\bibitem{icps-1998-5}
St\'{e}phane Genaud.
\newblock On deriving {HPF} code from \textsc{Pei} programs.
\newblock Technical Report RR 98-05, ICPS-Universit\'{e} Louis Pasteur, juin
  1998.

\bibitem{cds-2005}
A. Schaaff, F. Bonnarel, J.-J. Claudon, R. David, S. Genaud, M. Louys, C. Pestel and C. Wolf, 
\newblock Work around distributed image processing and workflow management, 
\newblock poster à ADASS 2005, Madrid.



\end{thebibliography}

%---------------------------------------- collab ---------------------------------------------
%\newpage
\bigskip
\rubrique{\large\bf Détails}


\section{Activités d'enseignement}

\label{sc:ensgnt-univ}

\begin{itemize}

\item[$\diamond$] \`A la rentrée 2009, maître de conférences en informatique rattaché à l'EM%
\footnote{Ecole de Management, établissement créé fin 2007 par la fusion des établissement IECS et IAE de l'Université R. Schuman}, 
Université de Strasbourg. Service d'enseignement partagé entre les établissements:
\begin{itemize}
	\item EM Strasbourg
	\item ENSIIE\footnote{Ecole Nationale Supérieure d'Informatique pour l'Industrie et l'Entreprise, basée à Evry (ex-IEE du CNAM)}, antenne de Strasbourg. Responsable du cours \textit{Systèmes Informatiques}, première année.
      \item UFR mathématiques-informatique. \textit{Systèmes distribués, calcul parallèles, grilles}, Master ILC M2. \\[2mm]
\end{itemize}

\item[$\diamond$] Maitre de conférences en informatique à l'IECS, université Robert Schuman, Strasbourg (1998-2007).\\
	Principaux cours assurés en Master International à la Gestion:
\begin{itemize}
	\item Technologies des systèmes d'information
	\item Algorithmique - Programmation (application avec Java)
	\item Architecture des applications web (architecture, JavaScript, PHP, base de données)
	\item Outils pour la gestion de projet
	\item Réseaux \\[2mm]
\end{itemize}

\item [$\diamond$]
Vacataire 
\begin{itemize}
\item DESS puis Master informatique pro Université Louis Pasteur (ULP). Cours de systèmes distribués. (2001-2007) 
\item Master recherche informatique ULP. Cours d'options {\em Grilles informatiques}. (2002-2003)
\item licence informatique ULP. TD du cours système d'exploitation. (1999-2001). 
\item CNAM de Strasbourg, filière informatique. 
cours de programmation parallèlele dans l'unité de valeur {\em conception et développement du logiciel} 
du cycle B (1/3 de l'UV). (1996-2000).
Cours magistral et TD (TD assurés avec Franco Zaroli).
\item école d'ingénieur \textit{ENSPS}. Outils de gestion de projets. (2000)
\item école d'ingénieur \emph{Ecole et Observatoire de Physique du Globe} (1997-1999).
Calcul parallèle. Généralités et application à des méthodes de résolution directe de systèmes d'équations 
linéaires en data-paralléle. 
\item DESS informatique du parallélisme (1993).\\[2mm]
\end{itemize}

\item [$\diamond$]
ATER plein-temps à l'IUT d'informatique de Strasbourg. (1996-1998). 
Principaux cours assurés: l'algorithmique, avec comme support la programmation en langage C et C++, 
et responsable de la première partie du cours système d'exploitation. \\[2mm]


\end{itemize}





\section{Responsabilités administratives}

%------------------------------------

\begin{itemize}
\item[$\bullet$] 
2004--2008: Membre de plusieurs commissions de spécialistes (section CNU 27).
\begin{itemize}
\item titulaire à l'Université Louis Pasteur (Strasbourg) entre 2004 et 2008,
\item suppléant à l'Université de Franche-Comté (Besançon) entre 2005 et 2008,
\item suppléant à l'Université Henri Poincaré (Nancy) entre 2006 et 2008.\\
\end{itemize}
2010-- \'Elu membre du comité d'experts (9 membres) pour la section 27 Université de Strasbourg en 2010.
Membre des comités de sélection:
\begin{itemize}
\item poste MC 210 UdS Réseaux et Protocole, 2010,
\item poste MC 1207 Université de Franche-Comté, IUT Belfort Montbéliard, 2010,
\end{itemize}



\item[$\bullet$] Responsable de la filière d'enseignement \emph{système d'information} à l'IECS, Strasbourg. (2001--2007)
\end{itemize}
\vspace{1cm}




\section{Animation scientifique}

\subsection{Responsabilités éditoriales}
%------------------------------------
\begin{itemize}
\item[$\bullet$] 
Membre des comité de programmes des conférences internationales:
IEEE/ACM International Conference on Grid Computing, 2008 (Tsukuba, Japon), et 2010 (Bruxelles, Belgique),
International Symposium on Grid and Distributed Computing, 2008 (Hainan Island, Chine). 
\item [$\bullet$]
Membre du comité de rédaction de la revue Technique et Science Informatiques (2005--2009).
\item [$\bullet$]
Membre du comité scientifique du département \emph{Expertise pour la recherche de l'UdS}. 
Le comité comprend 17 membres nommés, représentants les équipes scientifiques les plus impliquées 
par rapport aux équipements de calcul de l'Université. Le rôle du comité est de piloter
l'investissement en matière de calcul, et de promouvoir les projets présentant le plus 
d'intérêt scientifique par attribution de ressources.
\end{itemize}


\subsection{Animation de projets}
%------------------------------------
\begin{itemize}
\item[$\bullet$]
\textbf{Co-animateur} d'une action d'animation scientifique 
dans le cadre de l'action de développement technologique (ADT) de Aladdin de l'INRIA, 
visant à péreniser l'outil scientifique Grid5000. (07/2008--06/2012). 
Conjointement à l'ADT, l'animation scientifique est organisée autour de neuf actions d'animations baptisées \emph{défis}.
Je co-anime avec Nouredine Melab (LIFL, Université de Lille) le défi 
``{\em scalable application for large scale systems (algorithm, programming, execution models)}''.\\

\item[$\bullet$]
\textbf{Responsable du thème} \emph{programmation parallèle sur les grilles} au sein l'équipe ICPS,
du laboratoire LSIIT. Ce thème, parmi trois autres, est représenté par 3 à 4 permanents (2002--2007).\\


\item[$\bullet$]
\textbf{Porteur d'un projet d'Action Concertée Incitative}. Projet TAG, 
pluri-disciplinaire dans de l'ACI GRID (Globalisation des ressources informatiques et des données) 
du ministère de la recherche. Doté d'un budget de 182 000 \euro{} et d'un poste d'ingénieur). 
(12/2001 -- 12/2003).\\

\end{itemize}


\subsection{Participation à des projets scientifiques}

\begin{itemize}
\item[$\bullet$]
Participant au projet USS-SimGrid coordonné par Martin Quison, LORIA, Nancy. 
Projet accepté dans le cadre de l'appel d'offre ARPEGE-08 (systèmes embarqués et grandes infrastructures).
L'objectif est d'élargir les capacités de l'environnement de simulation SimGrid pour satisfaire des besoins plus divers.
Je devrai prendre part à mi-projet, à un work package dont l'objectif est d'enregistrer des traces d'exécutions (des programmes
MPI en particulier) et de les rejouer dans le simulateur. (2009 --2011).\\


\item[$\bullet$]
Participant au projet SPADES, coordonné par Eddy Caron, LIP-ENS Lyon.
L'objectif est de concevoir et construire un intergiciel capable de gérer un environnement 
dans lequel la disponibilité des
ressources change très rapidement. En particulier, cet intergiciel doit donner accès de manière fugace à 
des équipements de calculs très haute performance. (2009 -- 2011).\\


\item[$\bullet$]
Participant au projet HOUPIC coordonné par \'{E}ric Sonnedrücker à l'Institut de Recherche 
Mathématique Avancée (IRMA) à Stasbourg. 
Projet accepté dans le cadre de l'appel d'offre ANR-CIS-06 (Calcul Intensif et Simulation). 
L'objectif est le développement de code scientifiques pour la simulation de nombreux phénomènes physiques mettant en jeu des particules chargées, et leurs exécutions sur des systèmes parallèles ou distribués à large échelle.
Je prendrai part en dernière partie de projet, à l'expérimentation et à l'adaptation des codes développés 
sur des architectures massivement parallèles et/ou distribuées à large échelle.
(2007 -- 2009).\\


\item[$\bullet$] Participant au projet MDA (Masse de Données Astronomiques) coordonné par l'observatoire de Strasbourg dans le cadre de l'ACI MD (Masse de Données). (2004 -- 2006).\\

%\item Orateur dans diverses manifestations scientifiques sur ma thématique de recherche 
% * Journées \textit{Grilles} du CINES en mars 2003 à Montepllier, 
% * conférence et papier invité à GridUSe-2004 au workshop "What we have learned" 
%   de l'école thématique sur la \textit{globalisation des ressources informatiques et des données : Utilisation et Services},
%   juin 2004, Supélec Metz). 
% * Exposé Invité, Paristic 2006, 23 nov 2006, LORIA. Track 'Grid5000'. 
%   "Extensibilité d'un tracé de rais en tomographie sismique sur Grid5000."

% * Démonstrateur à SuperComputing'95 de l'environnement de développement Pei/VPei
% * Démonstrateur à SuperComputing'08 de l'environnement de développement P2P-MPI

\item[$\bullet$]
Correspondant pour l'équipe ICPS du groupe RGE (Réseau Grand Est), action géographique du groupe de recherche
\textit{Architecture, Système et Réseaux} du CNRS (GDR 725 ASR). RGE organise trois fois par an, une journée
d'exposés scientifiques sur le thème réseau et parallélisme, et grilles. Les équipes participantes appartiennent
à des laboratoires situés dans l'Est de la France.
%J'ai organisé la journée RGE du 01/06/2006 à Strasbourg.

\end{itemize}

\subsection{Participation à des jury de thèse}


\begin{itemize}
\item[$\bullet$] 
Examinateur de la thèse d'Heithem Abbès (soutenance décembre 2009), 
\textit{Approches de décentralisation de la gestion des ressources dans les Grilles},
rapporteurs Mohamed Jmaiel (Université de Sfax) et Franck Capello (INRIA-U. Urbana-Champain), 
encadrants Christophe Cérin (U. Paris 13) et Mohamed Jemni (École Supérieure des Sciences et Techniques de Tunis).
\end{itemize}





\section{Encadrements}

\subsection{Thèses}
\begin{enumerate}

\item 2000/2006 : co-encadrement de Marc Grunberg avec Catherine Mongenet 
(inscrit en thèse parallèlement à sa fonction d'ingénieur d'études au Réseau National de Surveillance Sismique). 
Taux de co-encadrement: \~{}70\%.
Directeurs de thèse: Catherine Mongenet et Michel Granet (Physicien, ULP).
La thèse \underline{soutenue en septembre 2006} est intitulée \textit{Conception d'une méthode de maillage 3D parallèle pour la construction d'un modèle de Terre réaliste par la tomographie sismique} - rapporteurs : Thierry Priol (IRISA, Rennes) et Denis Trystram (INPG, Grenoble).
Marc Grunberg occupe toujours aujour'hui un poste d'ingénieur d'études au Réseau National de Surveillance Sismique, 
\'{E}cole et Observatoire de Géophysique du Globe.\\

\item 2001/2004 : co-encadrement d'Arnaud Giersch avec Frédéric Vivien. Taux de co-encadrement: \~{}40\%.
Directeur de thèse Guy-René Perrin (ULP).
La thèse \underline{soutenue en décembre 2004} est intitulée \textit{Ordonnancement sur plates-formes hétérogènes de tâches partageant des données} - rapporteurs : Denis Trystram (INPG, Grenoble) et Henri Casanova (UCSD, San Diego).
Arnaud Giersch a aujourd'hui un poste de maître de conférences à l'IUT d'informatique de Belfort, université de Franche-Comté.\\



\item 2004/2008 : encadrement de Choopan Rattanapoka. Taux d'encadrement: 100\%. Directeur de thèse: Catherine Mongenet. 
La thèse \underline{soutenue en avril 2008} est intitulée \textit{P2P-MPI: A Fault-tolerant Message Passing Interface Implementation for Grids} - rapporteurs : Franck Cappello (INRIA, Orsay) et Thilo Kielmann (Vrije Universiteit, Amsterdam). 
Choopan Rattanapoka a aujourd'hui un poste permanent d'assistant professor au
Department of Eletronics Engineering Technology du King Mongkut's University of Technology, à Bangkok (Thailande).



\end{enumerate}

\subsection{Stages de DEA/Master}
\begin{enumerate}
\item 2002 : encadrement de Dominique Stehly en DEA. Mémoire intitulé \textit{Ordonnancement d'applications parallèles sur la grille}, soutenu 07/2002.

\item 2004 : encadrement de Choopan Rattanapoka en stage DEA. Mémoire intitulé  \textit{P2P-MPI : A Peer-to-Peer Framework for Robust Execution of Message Passing Parallel Programs on Grids}, soutenu 07/2004.

\item 2006 : co-encadrement de Constantinos Makassikis en stage de Master recherche, avec Jean-Jacques Pansiot et Guillaume Latu. Mémoire intitulé  \textit{Modèle de coût des communications TCP à un niveau applicatif}, soutenu 06/2006.

\item 2006 : encadrement de Ghazi Bouabene en stage de Master recherche. Mémoire intitulé  \textit{Sélection de pairs et allocation de tâches dans P2P-MPI}, soutenu 06/2006.

\end{enumerate}


\subsection{Autres}
Encadrement internship INRIA\\
\begin{itemize}
\item[$\bullet$]  {\it Optimisation de l'opération collective MPI all-to-all}, Antonio Grassi, Master de Facultad de Ingeniería, Universidad de la República, Montevideo, Uruguay. 4 mois, avril-août 2008, co-encadré avec Emmanuel Jeannot (AlGorille, LORIA).
\item[$\bullet$]  {\it Data Management in P2P-MPI}, Jagdish Achara, B-Tech de LNMIIT Jaipur, Inde. 3 mois, mai-août 2009. 
\end{itemize}
~\\
 
J'ai encadré ou co-encadré des stages ou projets tutorés d'étudiants de l'ULP (150h, un mois plein). 
Parmi les plus récents:\\
\begin{itemize}
\item[$\bullet$]  {\it Mise en {\oe}uvre d'une technique de segmentation par ligne de partage des eaux dans un environnement distribué hétérogène}, Lionel Ketterer, projet tutoré, février 2007, co-encadré avec Sebastien Lefèvre (LSIIT).
\item[$\bullet$]  {\it Distribution de calculs de Pricing d'options au modele Europeen sur grille}, Nabil Michraf et Khalid Souissi, projet tutoré, février 2006, co-encadré avec Stéphane Vialle (Supelec).

\item[$\bullet$]  {\it Parallélisation de la méthode Adaboost}, Abdelaziz Gacemi, projet tutoré, février 2007.
\item[$\bullet$] {\it Réalisation d'un portail web pour P2P-MPI avec SOAP}, David Michea, projet tutoré, février 2005.

\item[$\bullet$] {\it Parallelisation de la méthode MACLAW}, stage master, avril-juillet 2006, Damien Vouriot, co-encadré avec Pierre Gancarski (LSIIT).
\item[$\bullet$] {\it Heursitiques d'ordonnancement basées sur des traces d'exécution pour pour programmes parallèles}, Ghazi Bouabene, stage master, juillet-septembre 2006.
\item[$\bullet$]  {\it Outil de visualisation du réseau P2P dans P2P-MPI}, Ghazi Bouabene, stage licence, juin-août 2005.
\end{itemize}

