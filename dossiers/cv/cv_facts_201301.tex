
\setlength{\tabcolsep}{5pt}
\noindent
\begin{tabular}{lp{13.7cm}}
\textsc{NOM}		&{\bf Stéphane \textsc{Genaud}}, \texttt{\url{http://icps.u-strasbg.fr/~genaud}}\\
				& Né le 10 mars 1969. Français. Marié, 2 enfants. \\
				  & Pôle API, Blvd S. Brant,\\
				  & F-67400 Illkirch\\
				  & Tél. : 03 68 \\ 
				 & \texttt{genaud@unistra.fr}\\[5mm]

\textsc{Formation}	& 2009:  Habilitation à Diriger les Recherches en Informatique, Université Henri Poincaré Nancy.
      			  Titre: {\em Exécutions de programmes parallèles à passage de messages sur grille de calcul.}
   				 Rapporteurs: 	Christophe Cérin (Univ. Paris Nord), 
				   		   	Frédéric Desprez (INRIA),
							Thierry Priol (INRIA). 
							Examinateur: Pascal Bouvry (Univ. Luxembourg)
							Garant: Jens Gustedt (INRIA).\\[5mm]
%
				& 1997:  Doctorat en Sciences, mention Informatique, Université Louis Pasteur, Strasbourg.
      			  Titre: {\em Transformations de programmes \textsc{Pei} : applications au parall\'{e}lisme de donn\'{e}es}.
		              Rapporteurs: Luc Bougé (\'{E}cole Normale Supérieure Lyon) et Patrice Quinton (Université de Rennes). Examinateur : Christian Lengauer (Université Passau, Allemagne).\\[2mm]
			      & 1993: Diplôme d'\'{E}tudes supérieures Spécialisées (DESS) en Informatique du parallélisme, 
				  Université de Franche-Comté, Besançon. mention {\em très bien}.\\[2mm]
				& 1991: Bachelor of Sciences (BSc) in European Informatics, Sheffield Hallam University.\\[5mm]
%{\bf 1989}  & {\bf Diplôme Universitaire de Technologie (DUT)} en informatique, Institut Universitaire de Technologies de Nantes, Université de Nantes.\\
%\end{tabular}

\textsc{Emplois} 		& depuis sept. 2012: Maître de conférences mise à disposition de l'ENSIIE 
                                  pour exercer la co-direction du campus Strasbourgeois de l'école. 
                                  Membre du LSIIT (Laboratoire des Sciences de l'Image, de l'Informatique et de la Télédétection) 
                                  UMR 7005 CNRS-U. Strasbourg.\\[2mm]

                                  & 2009--2012: Maître de conférences à l'Université de Strasbourg. Membre du LSIIT, équipe ICPS 
                                  (Image et Caclul Parallèle Scientifique).\\[2mm]
 	                          & 2007--2009: Détaché Chargé de Recherches au LORIA, équipe projet 
				  INRIA \textsc{AlGorille}.\\[2mm]

				& 1998--2007: Maître de conférences à l'IECS, Université Robert Schuman de Strasbourg (aujourd'hui \'{E}cole de Management de Strasbourg).\\[2mm]
				& 1998--2007: Membre du LSIIT, UMR de l'Université Louis Pasteur et du CNRS (UMR 7005), 
équipe ICPS (Image et Caclul Parallèle Scientifique).\\[2mm]
				& 1996--1998: Attaché Temporaire d'Enseignement et de Recherches (ATER) au département d'informatique 
de l'IUT de Strasbourg Sud.\\[2mm]
				& 1994--1996: Doctorant à l'Université Louis Pasteur (Strasbourg). Directeur de thèse: Guy-René Perrin.\\[2mm]
				& 1993--1994: Doctorant à l'Université de Franche-Comté (Besançon). Directeur de thèse: Guy-René Perrin.\\[5mm]


\textsc{Responsabilités} & 2002--2007: Responsable du thème de recherche \textit{Grilles} dans l'équipe ICPS du LSIIT.\\[2mm]
				 & 2001-2007 : Responsable de la filière \textit{Système d'Information} du master Management International de l'IECS.\\[2mm]
\end{tabular}
%------------- P H O T O ---------------------------
%\pgfdeclareimage[interpolate=true,width=2.3cm]{idphoto}{stephane_genaud_gray_picture}
%\pgfputat{\pgfxy(14.5,19)}{\pgfuseimage{idphoto}}
%          pgfxy(width,height) from      



%---------------------------------------------------------------------------------------------
\newpage
\small
\bibliographystyle{plain}
\begin{thebibliography}{99}

\subsection*{Thèses}

\bibitem{hdr}
\textbf{Stéphane Genaud}.
\newblock 
{\em Exécutions de programmes parallèles à passage de messages sur grille de 
calcul}.
\newblock 
Habilitation à diriger des recherches de l'université Henri Poincaré, 
Nancy. Décembre 2009.
\newblock 
Rapporteurs : C. Cérin (Paris 13), F. Desprez (INRIA Rhônes-Alpes), 
T. Priol (INRIA Bretagne-Atlantique).\\[2mm]

\bibitem{icps-1997-4}
\textbf{Stéphane Genaud}.
\newblock {\em Transformations de programmes \textsc{Pei} : applications au
  parallélisme de données}.
\newblock Thèse de doctorat de l'université Louis Pasteur, Strasbourg, Janvier 1997.
\newblock Rapporteurs : Luc Bougé et Patrice Quinton. 

\subsection*{Chapitre de livre}

\bibitem{icps-book}
\textbf{Stéphane Genaud} et Choopan Rattanapoka.
\newblock 
\emph{A Peer-to-Peer Framework for Message Passing Parallel Programs.}
\newblock 
Parallel Programming and Applications in Grid, P2P and Network-based System,
in {\em Advances In Parallel Computing} Series. Editor G. R. Joubert.
IOS Press, juin 2009. 
 

\subsection*{Articles en revues internationales}

\setlength{\itemsep}{1.5mm}


\bibitem{icps-2009-217}
\newblock \textbf{Stéphane Genaud}, Emmanuel Jeannot et Choopan Rattanapoka.
\newblock Fault-Management in P2P-MPI.
\newblock {\em International Journal of Parallel Programming}, Springer, 
37(5):433--461, août 2009.


\bibitem{icps-2008-188}
\textbf{Stéphane Genaud}, Pierre Gançarski, Guillaume Latu, Alexandre Blansché, 
Choopan Rattanapoka et Damien Vouriot. \newblock Exploitation of a parallel 
clustering algorithm on commodity hardware with P2P-MPI.
\newblock 
{\em The Journal of SuperComputing}, Springer, 43(1):21--41, jan. 2008.


\bibitem{icps-2007-182}
\textbf{Stéphane Genaud} et Choopan Rattanapoka.
\newblock P2P-MPI: A Peer-to-Peer Framework for Robust Execution of Message 
Passing Parallel Programs on Grids.
\newblock {\em Journal of Grid Computing}, Springer, 5(1):27--42, mai 2007.


\bibitem{icps-2004-125}
\textbf{Stéphane Genaud}, Arnaud Giersch, et Frédéric Vivien.
\newblock Load-balancing scatter operations for Grid computing.
\newblock {\em Parallel Computing}, Elsevier, 30(8):923--946, août 2004.

\bibitem{icps-2004-107}
Marc Grunberg, \textbf{Stéphane Genaud} et Catherine Mongenet.
\newblock Seismic ray-tracing and Earth mesh modeling on various parallel
  architectures.
\newblock 
{\em The Journal of Supercomputing}, Kluwer, 29(1):27--44, juillet 2004.


\subsection*{Articles en revues nationales}
\bibitem{icps-2005-146}
\textbf{Stéphane Genaud} et Marc Grunberg. 
\newblock  Calcul de rais en tomographie sismique : exploitation sur la grille.
\newblock {\em Technique et Science Informatiques}, numéro spécial Renpar, 
Hermès-Lavoisier, 24(5), pages 591--608, décembre 2005.

\bibitem{icps-1996-2}
\textbf{Stéphane Genaud}.
\newblock Transformations d'énoncés \textsc{Pei}.
\newblock {\em Technique et Science Informatiques}, 15(5), pages 601--618, 
Hermès, avril 1996.

\bibitem{icps-1993-69}
\textbf{Stéphane Genaud} et Guy-René Perrin.
\newblock Une expérience d'implantation d'un algorithme systolique sur
  hypercube.
\newblock {\em La Lettre du Transputer et des calculateurs parallèles},
  (17), mars 1993.


\subsection*{Conférences internationales avec actes et comité de lecture}

\bibitem{FrincuGG13}
Marc Frincu, \textbf{Stéphane Genaud} et Julien Gossa.
\newblock Comparing Provisioning and Scheduling Strategies for Workflows on Clouds
\newblock {\em 2nd IEEE International Workshop on Workflow Models, Systems, Services 
and Applications in the Cloud (CloudFlow), IPDPS 2013}, mai 2013.


\bibitem{michon2012}
Etienne Michon, Julien Gossa, \textbf{Stéphane Genaud}.
\newblock Free elasticity and free CPU power for scientific workloads on IaaS Clouds
\newblock {\em 18th IEEE International Conference on Parallel and Distributed Systems}, 
IEEE, déc. 2012.
\newblock \small{\textit{(papiers acceptés/soumis:87/294, taux: 29\%)}}


\bibitem{icps-2011-225}
\newblock \textbf{Stéphane Genaud} et Julien Gossa,
\newblock Cost-wait Trade-offs in Client-side Resource Provisioning with 
Elastic Clouds.
\newblock {\em 4th IEEE International Conference on Cloud Computing (CLOUD 
2011)}, juillet 2011.
\newblock \small{\textit{(papiers acceptés/soumis{:}36/198, taux: 18\%)}}


\bibitem{icps-2011-224}
\newblock Pierre-Nicolas Clauss, Mark Stillwell, \textbf{Stéphane Genaud}, 
Fr\'ed\'eric Suter, Henri Casanova and  Martin Quinson.
\newblock Single Node On-Line Simulation of MPI Applications with SMPI.
\newblock {\em 25th IEEE International Parallel \& Distributed Processing 
Symposium (IPDPS 2011)}, mai 2011.
\newblock \small{\textit{(papiers acceptés/soumis{:}112/571, taux: 19\%)}}

\bibitem{icps-2009-219}
\newblock Virginie Galtier, \textbf{Stéphane Genaud} et Stéphane Vialle.
\newblock Implementation of the AdaBoost Algorithm for Large Scale Distributed 
Environments: Comparing JavaSpace and MPJ.
\newblock {\em 15th IEEE International Conference on Parallel and Distributed Systems}, 
IEEE, déc. 2009.
\newblock \small{\textit{(papiers acceptés/soumis:91/305, taux: 29\%)}}


\bibitem{icps-2009-214}
\textbf{Stéphane Genaud} and Choopan Rattanapoka.
\newblock Evaluation of Replication and Fault Detection in P2P-MPI.
\newblock 
{\em 6th IEEE International Workshop on Grid Computing (HPGC), IPDPS 2009}, 
mai 2009.
\newblock \textit{(Papier invité)}.

\bibitem{icps-2008-193}
\textbf{Stéphane Genaud} and Choopan Rattanapoka. 
\newblock Large-Scale Experiment of Co-allocation Strategies for Peer-to-Peer 
Supercomputing in P2P-MPI,
\newblock 
{\em 5th IEEE International Workshop on Grid Computing (HPGC), IPDPS 2008}, 
avril 2008.

\bibitem{icps-2007-192}
Ludovic Hablot and Olivier Glück and Jean-Christophe Mignot and \textbf{Stéphane Genaud} and Pascale Vicat-Blanc Primet.
\newblock Comparison and tuning of MPI implementation in a grid context.
\newblock {\em Proceedings of 2007 IEEE International Conference on Cluster Computing (CLUSTER)}, 458--463, september 2007.
\newblock \small{\textit{(papiers acceptés/soumis:42/106, taux: 39\%)}}

\bibitem{icps-2007-185}
\newblock \textbf{Stéphane Genaud} et Choopan Rattanapoka.
\newblock Fault Management in {\pmpi}. 
\newblock International Conference on {\em Grid and Pervasive Computing, 
(GPC 2007)}, LNCS, Springer, mai 2007.
\newblock \small{\textit{(papiers acceptés/soumis:56/217, taux: 25\%)}}
%submitted papers: 217; accepted : 56 full papers, 12 oral-short papers;  acceptance rate: 25\%

\bibitem{icps-2007-184}
\newblock \textbf{Stéphane Genaud}, Marc Grunberg et Catherine Mongenet.
\newblock Experiments in running a scientific {MPI} Application on GRID'5000. 
\newblock distingué par le \textsc{Intel} \textit{best paper award}.
\newblock {\em 4th IEEE International Workshop on Grid Computing (HPGC), IPDPS 
2007}, mars 2007.


\bibitem{icps-2005-155}
\textbf{Stéphane Genaud} et Choopan Rattanapoka.
\newblock A Peer-to-peer Framework for Robust Execution of Message Passing 
Parallel Programs.
\newblock 
In {\em EuroPVM/MPI 2005}, LNCS 3666, Springer-Verlag, pages 276--284, 
septembre 2005.
\newblock \small{\textit{(papiers acceptés/soumis:61/126, taux: 48\%)}}


\bibitem{icps-2004-124}
Marc Grunberg, \textbf{Stéphane Genaud}, et Catherine Mongenet.
\newblock Parallel adaptive mesh coarsening for seismic tomography.
\newblock In {\em SBAC-PAD 2004, 16th Symposium on Computer Architecture and
  High Performance Computing}. IEEE Computer Society Press, octobre 2004.
\newblock \small{\textit{(papiers acceptés/soumis:32/93, taux: 34\%)}}

\bibitem{icps-2003-75}
\textbf{Stéphane Genaud}, Arnaud Giersch, et Frédéric Vivien.
\newblock Load-balancing scatter operations for Grid computing.
\newblock In {\em Proceedings of 12th Heterogeneous Computing Workshop 
(HCW), IPDPS 2003}. IEEE Computer Society Press, avril 2003.

\bibitem{icps-2002-62}
Romaric David, \textbf{Stéphane Genaud}, Arnaud Giersch, \'{E}ric Violard, et 
  Benjamin Schwarz.
\newblock Source-code transformations strategies to load-balance Grid
  applications.
\newblock In {\em International Conference on Grid Computing - GRID'2002}, 
LNCS 2536, pages 82--87. Springer-Verlag, novembre 2002.

\bibitem{icps-2002-20}
Marc Grunberg, \textbf{Stéphane Genaud}, et Catherine Mongenet.
\newblock Parallel seismic ray-tracing in a global {E}arth mesh.
\newblock In {\em Proceedings of the 2002 Parallel and Distributed Processing
  Techniques and Applications (PDPTA'02)}, pages 1151--1157, juin 2002.

\bibitem{icps-1997-3}
Eric Violard, \textbf{Stéphane Genaud} et Guy-René Perrin.
\newblock Refinement of data-parallel programs in pei.
\newblock In {\em IFIP Working Conference on Algorithmic Language and Calculi}, 
R.~Bird and L.~Meertens editors, Chapman \& Hall~Ed., février 1997.
\newblock 25 pages.

\bibitem{icps-1995-1}
\textbf{Stéphane Genaud}, Eric Violard, et Guy-René Perrin.
\newblock Transformation techniques in \textsc{Pei}.
\newblock In P.~Magnusson S.~Haridi, K.~Ali, editor, {\em Europar'95}, LNCS
  966, pages 131--142. Springer-Verlag, août 1995.
\newblock \small{\textit{(papiers acceptés/soumis:50/180, taux: 27\%)}}




\subsection*{Conférences nationales avec actes et comité de lecture}
\bibitem{icps-2003-111}
Marc Grunberg et \textbf{Stéphane Genaud}.
\newblock Calcul de rais en tomographie sismique : exploitation sur la grille.
\newblock In {\em Renpar2003}, pages 179--186. INRIA, octobre 2003.

\bibitem{icps-1995-6}
\textbf{Stéphane Genaud}.
\newblock Techniques de tranformations d'énoncés \textsc{Pei} pour la
  production de programmes data-parallèles.
\newblock In {\em RenPar 7}, mai 1995, Mons, Belgique.

\bibitem{icps-1994-46}
Guy-René Perrin, Eric Violard et \textbf{Stéphane Genaud}.
\newblock \textsc{Pei} : a theoretical framework for data-parallel programming.
\newblock In {\em Workshop on Data-Parallel Languages and Compilers}, Lille, 
mai 1994.
\vspace{3mm}


\subsection*{Autres communications}

\bibitem{iphc-2011}
Christine Carapito, Jérôme Pansanel, Patrick Guterl, Alexandre Burel, Fabrice 
Bertile, \textbf{Stéphane Genaud}, Alain Van Dorsselaer, Christelle Roy.
\newblock Une suite logicielle pour la protéomique interfacée sur une grille de 
calcul. Utilisation d'algorithmes libres pour l'identification MS/MS, le 
séquençage de novo et l'annotation fonctionnelle.
\newblock Rencontres Scientifiques France Grilles 2011, Lyon.


\bibitem{ketterlin11}
Alain Ketterlin, \textbf{Stéphane Genaud}, Matthieu Kuhn.
\newblock Loop-Nest Recognition for the Extraction of Communication Patterns 
and the Compression of Message-Passing Parallel Traces.
\newblock Research Report ICPS 11-01. Université de Strasbourg. déc. 2011.


\bibitem{cds-2005}
A. Schaaff, F. Bonnarel, J.-J. Claudon, R. David, \textbf{S. Genaud}, M. Louys, 
C. Pestel and C. Wolf.
\newblock Work around distributed image processing and workflow management, 
\newblock poster à ADASS 2005, Madrid.


\bibitem{icps-2003-113}
Marc Grunberg, \textbf{Stéphane Genaud}, et Michel Granet.
\newblock Geographical {ISC} data characterization with parallel ray-tracing.
\newblock In {\em Eos Trans. AGU, 84(46), Fall-Meeting Suppl., Abstract
  S31E-0793}, décembre 2003.

\bibitem{icps-2003-114}
\textbf{Stéphane Genaud}.
\newblock Applications parallèles sur la grille: mieux vaut il être rapide ou résistant ? 
\newblock {\em Actes de GridUSe-2004}, Workshop "What we have learned", conférence invitée.
\newblock Supélec Metz, juin 2004. 

\subsection*{En cours de soumission}

\bibitem{itpro-cs11}
\newblock \textbf{Stéphane Genaud}, Julien Gossa et Etienne Michon.
\newblock Provisioning Cloud resources on the client-side: a cost-performance trade-off approach.
IEEE IT Professional Magazine on Cloud Computing. Nov 2011

\bibitem{tomacs-11}
Olivier Beaumont, Laurent Bobelin, Henir Casanova, Pierre-Nicolas Clauss, 
Bruno Donassolo, Lionel Eyraud-Dubois, \textbf{Stéphane Genaud}, Sacha Hunold, 
Arnaud Legrand, Martin Quinson.
\newblock
Towards Scalable, Accurate, and Usable Simulations of Distributed Applications and Systems,
ACM Transactions on Modeling and Computer Simulation. Oct 2011.
\end{thebibliography}



%---------------------------------------- collab ---------------------------------------------
%\newpage
\bigskip
\rubrique{\large\bf Détails}


\section{Activités d'enseignement}

\label{sc:ensgnt-univ}

\begin{itemize}

\item[$\diamond$] \`A la rentrée 2009, maître de conférences en informatique rattaché à l'EM%
\footnote{Ecole de Management, établissement créé fin 2007 par la fusion des établissement IECS et IAE de l'Université R. Schuman}, 
Université de Strasbourg. Service d'enseignement partagé entre les établissements:
\begin{itemize}
	\item EM Strasbourg
	\item ENSIIE\footnote{Ecole Nationale Supérieure d'Informatique pour l'Industrie et l'Entreprise, basée à Evry (ex-IEE du CNAM)}, antenne de Strasbourg. Responsable du cours \textit{Systèmes Informatiques}, première année.
      \item UFR mathématiques-informatique. \textit{Systèmes distribués, calcul parallèles, grilles}, Master ILC M2. \\[2mm]
\end{itemize}

\item[$\diamond$] Maitre de conférences en informatique à l'IECS, université Robert Schuman, Strasbourg (1998-2007).\\
	Principaux cours assurés en Master International à la Gestion:
\begin{itemize}
	\item Technologies des systèmes d'information
	\item Algorithmique - Programmation (application avec Java)
	\item Architecture des applications web (architecture, JavaScript, PHP, base de données)
	\item Outils pour la gestion de projet
	\item Réseaux \\[2mm]
\end{itemize}

\item [$\diamond$]
Vacataire 
\begin{itemize}
\item DESS puis Master informatique pro Université Louis Pasteur (ULP). Cours de systèmes distribués. (2001-2007) 
\item Master recherche informatique ULP. Cours d'options {\em Grilles informatiques}. (2002-2003)
\item licence informatique ULP. TD du cours système d'exploitation. (1999-2001). 
\item CNAM de Strasbourg, filière informatique. 
cours de programmation parallèlele dans l'unité de valeur {\em conception et développement du logiciel} 
du cycle B (1/3 de l'UV). (1996-2000).
Cours magistral et TD (TD assurés avec Franco Zaroli).
\item école d'ingénieur \textit{ENSPS}. Outils de gestion de projets. (2000)
\item école d'ingénieur \emph{Ecole et Observatoire de Physique du Globe} (1997-1999).
Calcul parallèle. Généralités et application à des méthodes de résolution directe de systèmes d'équations 
linéaires en data-paralléle. 
\item DESS informatique du parallélisme (1993).\\[2mm]
\end{itemize}

\item [$\diamond$]
ATER plein-temps à l'IUT d'informatique de Strasbourg. (1996-1998). 
Principaux cours assurés: l'algorithmique, avec comme support la programmation en langage C et C++, 
et responsable de la première partie du cours système d'exploitation. \\[2mm]


\end{itemize}





\section{Responsabilités administratives}

%------------------------------------

\begin{itemize}
\item[$\bullet$] \textbf{Comissions Spécialistes / Comité d'experts}:
\begin{itemize}
\item 2004--2008: Membre de plusieurs commissions de spécialistes (section CNU 27).
	\begin{itemize}
		\item titulaire à l'Université Louis Pasteur (Strasbourg) entre 2004 et 2008,
		\item suppléant à l'Université de Franche-Comté (Besançon) entre 2005 et 2008,
		\item suppléant à l'Université Henri Poincaré (Nancy) entre 2006 et 2008.\\
	\end{itemize}
\item 2010-- \'Elu membre du comité d'experts (9 membres) pour la section 27 Université de Strasbourg en 2010.
Membre des comités de sélection:
	\begin{itemize}
		\item poste MC 210 UdS Réseaux et Protocole, 2010,
		\item poste MC 1207 Université de Franche-Comté, IUT Belfort Montbéliard, 2010.\\
	\end{itemize}
\end{itemize}



\item[$\bullet$] \textbf{Pédagogique}:
Responsable de la filière d'enseignement \emph{système d'information} à l'IECS, Strasbourg. (2001--2007)
\end{itemize}
\vspace{1cm}




\section{Animation scientifique}

\begin{itemize}
\item[$\bullet$] Obtention de la prime d'excellence scientifique (PES) à partir d'octobre 2009.
\end{itemize}

\subsection{Responsabilités éditoriales}
%------------------------------------
\begin{itemize}
\item[$\bullet$] 
Membre des comité de programmes des conférences internationales:
12th International Conference on Algorithms and Architectures for Parallel Processing (ICA3PP-12), 2012 (Fukuoka, Japan),
13th IEEE International Conference on High Performance Computing and Communications, 2011 (Banff, Canada),
IEEE/ACM International Conference on Grid Computing, 2008 (Tsukuba, Japon), et 2010 (Bruxelles, Belgique),
21th IASTED International Conference on Parallel and Distributed Computing and Systems, 2010 (Marina del Rey, USA),
International Symposium on Grid and Distributed Computing, 2008 (Hainan Island, Chine). 
\item [$\bullet$]
Membre du comité de rédaction de la revue Technique et Science Informatiques (2005--2009).
\item [$\bullet$]
Membre du comité scientifique du département \emph{Expertise pour la recherche de l'UdS} (sept 2010--).
Le comité comprend 17 membres nommés, représentants les équipes scientifiques les plus impliquées 
par rapport aux équipements de calcul de l'Université. Le rôle du comité est de piloter
l'investissement en matière de calcul, et de promouvoir les projets présentant le plus 
d'intérêt scientifique par attribution de ressources.
\end{itemize}


%-----------------A N I M A T I O N -------------------
\subsection{Activités scientifiques}

\subsubsection{Niveau international}
Membre des comité de programmes des conférences internationales:\\[-3mm]
\begin{itemize}
\item[$\bullet$]
15th IEEE International Conference on Computational Science and Engineering (CSE 2012) 
Cluster, Grid, Cloud and P2P Computing track. Paphos, Cyprus, October 3-5, 2012. 
http://www.cse2012.cs.ucy.ac.cy/

\item[$\bullet$] 
14th IEEE International Conference on High Performance Computing and Communications (HPCC 2012), 
2012 (Liverpool, England),
\item[$\bullet$] 
13th IEEE International Conference on High Performance Computing and Communications (HPCC 2011), 
2011 (Banff, Canada),
\item[$\bullet$] 
IEEE/ACM International Conference on Grid Computing (GRID'10), 2010 (Bruxelles, Belgique),
\item[$\bullet$] 
20th IASTED International Conference on Parallel and Distributed Computing and Systems, 2010, (Marina Del Rey, USA),
\item[$\bullet$] 
IEEE/ACM International Conference on Grid Computing (GRID'08), 2008 (Tsukuba, Japon), 
\item [$\bullet$]
International Symposium on Grid and Distributed Computing, 2008 (Hainan Island, Chine),\\
\end{itemize}

Relecteur pour de nombreuses revues ou conférences internationales: IEEE Trans. on Distr. and Parallel Systems, 
J. of SuperComputing, J. of Grid Computing, IEEE Conference on Grid Computing, IEEE CCGrid conference, Europar,
IEEE IPDPS conference, \ldots.


\subsubsection{Niveau national}
%------------------------------------
\begin{itemize}

\item[$\bullet$] Obtention de la prime d'excellence scientifique (PES) à partir d'octobre 2009.\\

\item [$\bullet$]
membre du comité de rédaction de la revue Technique et Science Informatiques (2005--2009).\\
\end{itemize}

\underline{Projets en cours}\\
\begin{itemize}


\item[$\bullet$]
\textbf{Porteur local} pour le projet ANR SONGS (ANR 11 INFR 013-03)  (taux déclaré 40\%)
coordonné par Martin Quinson, LORIA, Nancy (2012-2015)  poursuivant le projet 
Uss-SimGrid (voir ci-dessous). Le projet vise à affiner les objets modélisés pour la 
simulation (processeurs multi-c{\oe}urs, mémoire) ou en ajouter (disque, réseaux spécialisés
comme Infiniband) et à fournir des interfaces adaptées à la représentation de systèmes
complexes comme des machines HPC ou des Clouds. Je suis responsable du work package
sur les clouds.\\

\item[$\bullet$]
Participant au projet blanc ANR E2T2 (ANR 11 SIMI 9) (taux déclaré 15\%) coordonné par 
Peter Beyer, laboratoire PIIM, Université de Provence (2011-2014). L'objectif du projet 
est d'améliorer la modélisation physique des plasmas de bord dans un tokamak. Dans ce
projet, ma tâche est de co-encadrer un doctorant, Matthieu Kuhn avec Guillaume Latu et
Nicolas Crouseille (IRMA) pour paralléliser les codes développés par le CEA Cadarache 
(IRFM) et les physiciens du PIIM. \\

\item[$\bullet$]
\textbf{Co-animateur} d'une action d'animation scientifique 
dans le cadre de l'action de développement technologique (ADT) de Aladdin de l'INRIA, 
visant à pérenniser l'outil scientifique Grid5000. (07/2008--06/2012). 
Conjointement à l'ADT, l'animation scientifique est organisée autour de neuf actions d'animations baptisées \emph{défis}.
Je co-anime avec Nouredine Melab (LIFL, Université de Lille) le défi 
``{\em scalable application for large scale systems (algorithm, programming, execution models)}''.\\
\end{itemize}


\underline{Projets passés}\\
\begin{itemize}
\item[$\bullet$]
Participant (taux déclaré 20\%) au projet ANR USS-SimGrid (ANR 08 SEGI 022) coordonné par 
Martin Quinson, LORIA, Nancy (2009 -- 2011). Ce projet a été labelisé projet \textit{phare}
par l'ANR.
L'objectif général du projet était  d'élargir les capacités 
de l'environnement de simulation SimGrid pour satisfaire des besoins plus divers, comme la
simulation de systèmes pair-à-pair ou d'environnements de calcul intensif.
Mes tâches ont concerné l'enregistrement des traces d'exécutions (des programmes MPI en 
particulier) afin de les rejouer dans le simulateur. J'ai redémarré
le travail commencé à l'université de Hawaï sur l'interface SMPI, qui permet de simuler des 
programmes MPI sans modification des codes sources. Elle est maintenant fonctionnelle depuis
la version 3.5 de SimGrid.\\


\item[$\bullet$]
Participant (taux déclaré 20\%) au projet SPADES (ANR 08 SEGI 025) coordonné par Eddy Caron, 
LIP-ENS Lyon (2009 -- 2011). L'objectif était de concevoir et construire un intergiciel capable 
de gérer un environnement dans lequel la disponibilité des ressources change très rapidement. 
En particulier, cet intergiciel doit donner accès de manière fugace à des équipements de calculs 
très haute performance. Mes tâches ont concerné la conception et l'évaluation de l'ordonnanceur
travaillant en collaboration avec un système pair-à-pair utilisé pour recenser dynamiquement
les ressources disponibles.\\


\item[$\bullet$] Participant (10\%) au projet Masse de Données Astronomiques (ACI Masse de données) 
coordonné par Françoise Genova, observatoire de Strasbourg (2004 -- 2006). \\

\item[$\bullet$]
\textbf{Porteur d'un projet d'Action Concertée Incitative}. Projet TAG, 
pluridisciplinaire dans de l'ACI GRID (Globalisation des ressources informatiques et des données) 
du ministère de la recherche. Doté d'un budget de 182 K\euro{} et d'un poste d'ingénieur). 
(12/2001 -- 12/2003).\\
\end{itemize}


%\item Orateur dans diverses manifestations scientifiques sur ma thématique de recherche 
% * Journées \textit{Grilles} du CINES en mars 2003 à Montepllier, 
% * conférence et papier invité à GridUSe-2004 au workshop "What we have learned" 
%   de l'école thématique sur la \textit{globalisation des ressources informatiques et des données : Utilisation et Services},
%   juin 2004, Supélec Metz). 
% * Exposé Invité, Paristic 2006, 23 nov 2006, LORIA. Track 'Grid5000'. 
%   "Extensibilité d'un tracé de rais en tomographie sismique sur Grid5000."

% * Démonstrateur à SuperComputing'95 de l'environnement de développement Pei/VPei
% * Démonstrateur à SuperComputing'08 de l'environnement de développement P2P-MPI

\subsubsection{Niveau local}
\begin{itemize}


\item[$\bullet$]
\textbf{Responsable du thème} \emph{programmation parallèle sur les grilles} au sein l'équipe ICPS,
du laboratoire LSIIT. Ce thème a compté parmi ses membres~: Guillaume Latu (MC), 
Eric Violard (MC), Romaric David (IR), Benjamin Schwarz (IE en CDD), Arnaud Giersch (doctorant), 
Choopan Rattanapoka (doctorant) sur la période 2002 à 2007.\\

\item [$\bullet$]
Membre du conseil scientifique du département \emph{Expertise pour la recherche de l'UdS} (sept 2010--).
Le comité comprend 17 membres nommés, représentants les équipes scientifiques les plus impliquées
par rapport aux équipements de calcul de l'Université. Le rôle du comité est de piloter
l'investissement en matière de calcul, et de promouvoir les projets présentant le plus
d'intérêt scientifique par attribution de ressources.\\

\item[$\bullet$]
Correspondant depuis 2002 pour l'équipe ICPS auprès du groupe RGE (Réseau Grand Est),
action géographique regroupant 9 sites du GDR \textit{Architecture, Système et Réseaux} 
CNRS (GDR 725 ASR). RGE organise trois fois par an, une journée consacrée à des exposés 
scientifiques des équipes et à une conférence par un industriel invité.\\
%J'ai organisé la journée RGE du 01/06/2006 à Strasbourg.


\end{itemize}

\subsubsection{Jurys de thèse}

\begin{itemize}

\item[$\bullet$] 
Rapporteur de la thèse d'Adrian Muresan, \'Ecole Normale Supérieure de Lyon
(soutenance déc. 2012), \textit{Scheduling and deployment of large-scale applications on 
Cloud platforms},
rapporteur J. F. Méhaut (U. de Grenoble), 
encadrants F. Desprez (INRIA Rhône-Alpes) et E. Caron (ENS Lyon)\\
\item[$\bullet$] 
Rapporteur de la thèse de Sébastien Miquée, Univ. Franche-Comté (soutenance 
jan. 2012), \textit{Exécution d'applications parallèles en environnements 
hétérogènes et volatils~: déploiement et virtualisation},
rapporteur C. Cérin (U. Paris 13), 
encadrants R. Couturier et D. Laiymani (U. Franche-Comté)\\
\item[$\bullet$] 
Rapporteur de la thèse de Fabrice Dupros, Univ. Bordeaux 1 (soutenance déc. 2010), 
\textit{Contribution à la modélisation numérique de la propagation des ondes 
sismiques sur architectures multic{\oe}urs et hiérarchiques},
rapporteur S. Lanteri (INRIA Sophia-Antipolis), 
encadrants D. Komatitsch (U. Pau) et J. Roman (Institut Polytechnique de 
Bordeaux).\\

\item[$\bullet$] 
Examinateur de la thèse d'Heithem Abbès (soutenance déc. 2009), 
\textit{Approches de décentralisation de la gestion des ressources dans les 
Grilles}, rapporteurs Mohamed Jmaiel (Université de Sfax) et Franck Capello 
(INRIA-U. Urbana-Champain), encadrants Christophe Cérin (U. Paris 13) et 
Mohamed Jemni (École Supérieure des Sciences et Techniques de Tunis).
\end{itemize}




\subsection{Encadrements}

\subsubsection{Thèses}
\begin{enumerate}

\item 10/2011-- : encadrement d'Etienne Michon. Taux d'encadrement: 50\%,
avec Julien Gossa. Financement DGA. La thèse porte sur les problématiques
d'allocation de ressources de cloud côté client.\\

\item 02/2011-- : encadrement de Matthieu Kuhn. Financement ANR E2T2. 
Taux d'encadrement prévisionnel: 20\%. Co-encadrants Guillaume Latu pour 
l'informatique, Nicolas Crouseille (HDR) pour les mathématiques appliquées. 
La thèse porte sur la parallélisation de modèles 
numériques pour la simulation de plasmas de bord.\\

\item 2004--2008 : encadrement de Choopan Rattanapoka. Taux d'encadrement: 100\%. 
Directeur de thèse: Catherine Mongenet. La thèse \underline{soutenue en avril 2008} 
est intitulée \textit{P2P-MPI: A Fault-tolerant Message Passing Interface Implementation 
for Grids} - rapporteurs : Franck Cappello (INRIA, Orsay) et Thilo Kielmann (Vrije 
Universiteit, Amsterdam). Choopan Rattanapoka a aujourd'hui un poste permanent 
d'assistant professor au Department of Eletronics Engineering Technology du King 
Mongkut's University of Technology, à Bangkok (Thailande).
Publications associées: 
\cite{icps-2005-155,icps-2007-182,icps-2007-185,
      icps-2008-188,icps-2008-193,icps-2009-214,
	icps-2009-217,icps-book}\\


\item 2001--2004 : co-encadrement d'Arnaud Giersch avec Frédéric Vivien. 
Taux de co-encadrement: $\sim$40\%. Directeur de thèse Guy-René Perrin (ULP).
La thèse \underline{soutenue en décembre 2004} est intitulée \textit{Ordonnancement 
sur plates-formes hétérogènes de tâches partageant des données} - rapporteurs : Denis 
Trystram (INPG, Grenoble) et Henri Casanova (UCSD, San Diego). Arnaud Giersch a 
aujourd'hui un poste de maître de conférences à l'IUT d'informatique de Belfort, 
université de Franche-Comté.
Publications associées: \cite{icps-2002-62,icps-2003-75,icps-2004-125}\\


\item 2000--2006 : co-encadrement de Marc Grunberg avec Catherine Mongenet 
(inscrit en thèse parallèlement à sa fonction d'ingénieur d'études au Réseau 
National de Surveillance Sismique). Taux de co-encadrement: $\sim$70\%.
Directeurs de thèse: Catherine Mongenet et Michel Granet (Physicien, ULP).
La thèse \underline{soutenue en septembre 2006} est intitulée \textit{Conception 
d'une méthode de maillage 3D parallèle pour la construction d'un modèle de Terre 
réaliste par la tomographie sismique} - rapporteurs : Thierry Priol (IRISA, Rennes) 
et Denis Trystram (INPG, Grenoble).
Marc Grunberg occupe toujours aujourd'hui un poste d'ingénieur d'études au Réseau 
National de Surveillance Sismique, \'{E}cole et Observatoire de Géophysique du Globe.
Publications associées: 
\cite{icps-2002-20,icps-2003-111,icps-2003-113,
      icps-2004-107,icps-2004-124,icps-2005-146,icps-2007-184}.\\



\end{enumerate}

\subsubsection{Stages de DEA/Master}
\begin{enumerate}

\item 2011 : Etienne Michon. Co-encadrement avec Julien Gossa. Mémoire 
intitulé  \textit{Allocation de ressources et ordonnancement côté client dans 
un environnement de Clouds}, soutenu 06/2011.
\item 2006 : Constantinos Makassikis. Co-encadrement avec Jean-Jacques Pansiot 
et Guillaume Latu. Mémoire intitulé \textit{Modèle de coût des communications 
TCP à un niveau applicatif}, soutenu 06/2006.
\item 2006 : Ghazi Bouabene. Mémoire intitulé  \textit{Sélection de pairs et 
allocation de tâches dans P2P-MPI}, soutenu 06/2006.
\item 2004 : Choopan Rattanapoka. Mémoire intitulé  \textit{P2P-MPI : A 
Peer-to-Peer Framework for Robust Execution of Message Passing Parallel 
Programs on Grids}, soutenu 07/2004.
\item 2002 : Dominique Stehly. Mémoire intitulé \textit{Ordonnancement 
d'applications parallèles sur la grille}, soutenu 07/2002.
\end{enumerate}

\subsubsection{Autres}
Encadrement internship INRIA
\smallskip
\begin{itemize}
\item[$\bullet$]  {\it Data Management in P2P-MPI}, Jagdish Achara, B-Tech de 
LNMIIT Jaipur, Inde. 3 mois, mai-août 2009. 
\item[$\bullet$]  {\it Optimisation de l'opération collective MPI all-to-all}, 
Antonio Grassi, Master de Facultad de Ingeniería, Universidad de la República, 
Montevideo, Uruguay. 4 mois, avril-août 2008, co-encadré avec Emmanuel Jeannot 
(AlGorille, LORIA).

\end{itemize}
~\\
Encadrement stage ENSIIE
\smallskip
\begin{itemize}
\item[$\bullet$]  {\it Experimentation de cluster virtualisé ave Nimbus}, 
Marien Ritzenthaler, ENSIIE 1A. 2,5 mois, juin-août 2010. 
\end{itemize}
~\\
%Encadrement apprentissage
%\begin{itemize}
%\item[$\bullet$]  
%{\it Conception et développement d'un outil de gestion de production}, Aymen Bouslama, Master 1 ILC, maître d'apprentissage David Damand, EM Strasbourg. 2010-2012.
%\end{itemize}

\label{sc:encadre-autres} 
Encadrements stages ou projets tutorés d'étudiants de l'UFR d'informatique, 
université de Strasbourg (150h, un mois plein). Parmi les plus récents:
\smallskip
\begin{itemize}
\item[$\bullet$]  {\it Interfaçage d'un batch scheduler un système de gestion 
de cloud IaaS}, Vincent Kerbache, TER, 2012, co-encadré avec J. Gossa. 
\item[$\bullet$]  {\it Mise en {\oe}uvre d'une technique de segmentation par 
ligne de partage des eaux dans un environnement distribué hétérogène}, 
Lionel Ketterer, projet tutoré, fév. 2007, co-encadré avec Sebastien 
Lefèvre (LSIIT).
\item[$\bullet$]  {\it Distribution de calculs de Pricing d'options au modele 
Europeen sur grille}, Nabil Michraf et Khalid Souissi, projet tutoré, fév. 2006, 
co-encadré avec Stéphane Vialle (Supelec).

\item[$\bullet$]  {\it Parallélisation de la méthode Adaboost}, Abdelaziz 
Gacemi, projet tutoré, fév. 2007.
\item[$\bullet$] {\it Réalisation d'un portail web pour P2P-MPI avec SOAP}, 
David Michea, projet tutoré, fév. 2005.

\item[$\bullet$] {\it Parallelisation de la méthode MACLAW}, stage master, 
avril-juillet 2006, Damien Vouriot, co-encadré avec Pierre Gancarski (LSIIT).
\item[$\bullet$] {\it Heursitiques d'ordonnancement basées sur des traces 
d'exécution pour pour programmes parallèles}, Ghazi Bouabene, stage master, 
juillet-septembre 2006.
\item[$\bullet$]  {\it Outil de visualisation du réseau P2P dans P2P-MPI}, 
Ghazi Bouabene, stage licence, juin-août 2005.
\end{itemize}

