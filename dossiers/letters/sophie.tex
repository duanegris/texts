\documentclass[a4paper,10pt]{letter}
\usepackage[francais]{babel}
\usepackage[utf8]{inputenc}
\usepackage[T1]{fontenc}
\selectlanguage{francais}
	  \date{Le 10 juillet 2018}
\begin{document}
%\name{Stéphane \textsc{Genaud}}
\begin{letter}{}
\opening{Ma petite femme,}

Pour tes 45 ans, je crois n'avoir rien à t'offrir. 
J'ai bien essayé de t'emmener dans des magasins pour choisir une petite fringue, mais de manière raisonnable, tu as décliné car tu as déjà pu t'offrir ce qui te plaisait tout au long de l'année, au gré des bonnes affaires à faire.

Pour tes 45 ans, tu as déjà presque tout. 
Une splendide maison à laquelle il ne manque qu'un bambou pour terminer la haie, une profession que tu exerces avec passion, une famille unie, avec laquelle tu as connu le grand bonheur de partir dans des voyages lointains ces dernières années, et surtout deux filles merveilleuses.

Autant de réalisations auxquelles tu as aspiré depuis toute petite. 
A ce stade de la vie, je suis certain que tu dresses un bilan plus que positif par rapport à toutes les espérances qu'on construit quand on est enfant et qu'on imagine notre vie de "quand on sera grands".

Alors que se passe t-il quand on a presque tout ?
Qu'est ce qui devient le plus important ?

Peut être refait on le tour des choses. Veut on revenir sur ce qu'on a mais qui pourrait être mieux ?
J'en ai l'impression. 

Il y a ce bambou qui manque, cela devient insupportable. 
Il y a ce mari, qui a participé à construire tout ça, mais qui a toujours les mêmes défauts.
Après toutes ces années, c'est toujours aussi difficile de comprendre comment il fonctionne.
Ses réactions sont prévisibles, mais c'est souvent à côté de ce que j'attends. 
Je sais que n'importe quelle petite phrase peut devenir un sujet de débât compliqué. 
Je suis agacée, et fatiguée d'avance de savoir ce qui ne lui convient pas dans ce que je viens de dire, 
qui m'est juste passé par la tête. Il ne lâche rien, je ne comprends pas pourquoi, 
il suffirait qu'il dise \emph{oui} et tout irait bien.

Il a des qualités mais depuis longtemps j'ai fait le deuil d'arriver à le convaincre que mes valeurs 
devraient aussi être les siennes. J'ai presque tout mais ce manque là est énorme. J'ai aussi intégré dans mes rêves 
d'enfant, un mari qui me regarderait comme mes parents se regardent. Un couple en osmose, qui réagit à l'unisson.

Alors c'est ça qu'il me reste à faire. Exiger plus de compréhension.  Je ne dois pas avoir peur de lui. 
On est un vieux couple, alors au pire, on s'accrochera un peu et puis tout rentrera dans l'ordre.
Je ne risque pas grand-chose, il finira peut être par être en symbiose avec moi.
Ce n'est pas très grave si je ne suis pas très attentive à ses petites blagues pas très marrantes. 
Ca ne m'interesse pas et si je rentre dans son jeu, il va m'entretenir des heures sur la référence subtile qui est derrière.
Il ne s'intéresse pas aux mêmes choses, il est curieux de tout, et peut débattre de tout et n'importe quoi.
Moi, je n'en vois pas l'intérêt. On m'a toujours appris à ne pas la ramener, de garder mes distances, on ne sait jamais
ce que les gens pourraient penser de moi. Pourquoi ne lui fais-je pas confiance ? Pourquoi je ne veux pas rentrer dans le débat?
Je ne sais pas exactement, probablement un relent d'éducation.

Oui, certainement, à 45 ans, on remet beaucoup de choses en question. On sent déjà poindre le moment où les enfants vont partir.
Que restera t-il quand le centre d'intérêt commun va d'un coup presque disparaître ?
Peut être pas grand-chose car finalement les enfants sont un formidable ciment qui masque 
beaucoup des dissensions qui peuvent se creuser.
Ce sont eux qui apportent la nouveauté et qui font qu'on les accompagne dans une direction commune au gré des rebondissements de leurs vies.

On est donc à un tournant à notre âge. Ce tournant peut aller dans la même direction, ou dans deux directions divergentes.
À nous de nous écouter, de reconstruire une complicité, car sans cela, nul projet ou obligation ne nous amènera à continuer l'aventure commune.
Mais moi j'ai envie de continuer, je ne renonce pas à te faire comprendre ce qui me fait vibrer dans la vie.
C'est une volonté de partager, par exemple la musique que j'aime et que tu trouves dorénavant ringuarde, 
et beaucoup d'autres choses importantes. Bien sûr, tout le monde est différent, personne ne partage
exactement les mêmes passions. Cependant, le mur se dresse quand on a renoncé à vouloir la partager ou l'expliquer
à l'autre, en sachant d'avance que c'est peine perdue.

J'ai appris beaucoup de toi tout au long de ses années. Je savais dès le départ que nous étions différents, que le chemin
pour se rejoindre était assez conséquent, mais au fond de moi, j'ai choisi cette différence avec l'intuition que c'était un
défi qui me sortirait de ma trajectoire, qui me ferait progresser. 
Avec l'espoir aussi de dévier la tienne, en te gardant avec toutes tes qualités,
et en t'aidant à t'émanciper de tous tes interdits qui te rivent à cette trajectoire, pour 
se rencontrer finalement et nous épanouir pleinement, à deux, en osmose. Pour cela, il n'y a
qu'une solution : accepter d'analyser notre propre façon de fonctionner, puis de se dévoiler à l'autre, 
en toute confiance. 

Comme symbole de tout ça, je t'offre, ma petite femme, une petite surprise pour ton anniversaire. 
Tu me connais par coeur, donc tu sais que cette surprise est une façon de sortir de mes rails. 
Je t'emmène ici\footnote{lat=47.216545, long= -1.551365} 
le 17 juillet, une journée et une soirée pour nous deux.

Voici donc ma deuxième lettre en vingt ans. Le score est de 2-0. 
J'en suis presque à souhaiter que ce soit la dernière car le contenu est pesant à chaque fois.
Peut être que ces questionnements surgissent dans tous les couples, mais je ne sais pas si tout le
monde fait ce travail d'introspection. Tout ce qu'on a vécu et construit ensemble le mérite bien.


	  \closing{Tpm.}
\end{letter}


\end{document}
                  

