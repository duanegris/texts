\documentclass[a4paper,10pt]{article}
\usepackage[francais]{babel}
\usepackage{url}

\selectlanguage{francais}
\input{letterLSIIT-0.04.sty}

\selectlanguage{francais}
\begin{document}


\name{St�phane \textsc{Genaud}}
%\fonction{Professeur des Universit�s}
\telephoneillkirch{+33 (0)390244542}
\email{genaud@icps.u-strasbg.fr}

\begin{letter}[Attestation of PhD continuation]{to}{Director of College of Industrial Technology\\King Mongkut's Institute of Technology,\\ North Bangkok}


I have been the advisor of Choopan Rattanapoka during his master thesis on which Choopan has worked from January to July 2004. 
Afterwards, Choopan has registered for a PhD cursus in computing science and his official advisor is professor Catherine Mongenet. 


As an assistant professor, i am working with Choopan since his master thesis and i know very well his work. Since Choopan arrived in our laboratory, we have been investigating the interest of a peer-to-peer middleware that would allow to run parallel programs on computational grids. 
Choopan has played a key role in the design and the development of this project. A first public version of the software is available since May 2005 (\url{http://grid.u-strasbg.fr/p2pmpi/}).
I must say i am fully satisfied with the work of Choopan.
Furthermore, a paper about this project, entitled \emph{A Peer-to-Peer Framework for Robust Execution of Message Passing Parallel Programs on Grids} has been accepted at the well-known international conference Euro PVM/MPI and will be published in the proceedings of the conference (in the series \emph{Lecture Notes in Computer Science} of Springer-Verlag).


Choopan Rattanapoka is currently in his first year of PhD and will continue working next year (2005-2006) on the same subject. 
I have no doubt Choopan will be able to get the PhD degree in computer science from University Louis Pasteur in approximatively two years.

Sincerely Yours,
\end{letter}
%\newpage
%\bibliographystyle{plain}
%\bibliography{arno}


\end{document}
