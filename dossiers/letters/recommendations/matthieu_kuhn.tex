\documentclass[a4paper,10pt]{article}
\usepackage[francais]{babel}
\usepackage{url}

\selectlanguage{francais}
\input{letter-LSIIT-UdS.sty}

\selectlanguage{francais}
\begin{document}


%\name{St�phane \textsc{Genaud}}
%\fonction{Professeur des Universit�s}
\telephoneillkirch{+33 (0)3688 54542}
\email{genaud@unistra.fr}

\begin{letter}[]{�}{Ecole Doctorale MSII}


\vspace{-3cm}
Le projet d'ANR E2T2 d�butant en 2011 pour 3 ans a pour but
d'am�liorer des codes de simulation de la physique des plasmas.
Dans le cadre de ce projet, dont une partie de l'activit� va 
�tre men�e au LSIIT au sein de l'�quipe ICPS, un financement 
de th�se a �t� obtenu.
Nous avons selectionn� un candidat, Matthieu Kuhn, que nous souhaiterions
pouvoir inscrire en th�se en informatique.

Matthieu Kuhn a obtenu le Master professionnel 
\textit{Calcul Scientifique et S�curit� Informatique} de l'Universit� de Strasbourg
en juillet 2010, apr�s avoir obtenu une licence de math�matiques avec la 
mention bien. Il a de bonnes recommandations de ses enseignants de Master,
Philippe Helluy et Eric Sonnendr�cker (jointes au dossier).
Il a effectu� son stage de Master au sein de l'ICPS
sur la parall�lisation d'un code scientifique de physique des fluides sur GPUs. 
Ses encadrants, tant du c�t� informatique, que de celui de l'astrophysique
ont �t� totalement satisfaits de son travail (voir la recommandation
de Dominique Aubert jointe au dossier).
Sa formation au calcul scientifique est un atout indispensable pour
travailler sur le sujet propos� par l'ANR.
Dans ce travail, il sera encadr� par moi m�me et Guillaume Latu du CEA Cadarache,
aid� de Nicolas Crouseille, IRMA et �quipe-projet INRIA CalVi.
Matthieu Kuhn serait h�berg� au LSIIT (ICPS). Les d�placements n�cessaires pour
le suivi r�gulier du travail par ses co-encadrants sont pr�vus dans le financement
du projet ANR.


Ce travail s'inscrit dans une collaboration de longue date entre l'ICPS et
CalVi. Matthieu Kuhn me semble pr�senter toutes les
qualit�s pour le travail de th�se propos�.

\begin{flushright}
St�phane \textsc{Genaud}
\end{flushright}

\noindent
Avis du directeur de l'unit� de recherche, Fabrice \textsc{Heitz}:\\[3mm]
%\vspace{3mm}
\framebox{
\begin{minipage}{.9\linewidth}
\hspace{.9\linewidth}
\vspace{4.3cm}

\end{minipage}
}

\end{letter}
%\newpage
%\bibliographystyle{plain}
%\bibliography{arno}


\end{document}
