\documentclass[a4paper,10pt]{article}
\usepackage[francais]{babel}
\usepackage{url}
\usepackage[utf8]{inputenc}

\selectlanguage{francais}
\usepackage[latin1]{inputenc}
\usepackage[T1]{fontenc}
\usepackage{babel}
%\usepackage{graphicx}
\usepackage{xspace}
\usepackage{color}
\usepackage{setspace}
\usepackage{geometry}
\usepackage{helvet}
\usepackage{calc}
\usepackage{ifthen}


\pagestyle{empty}

%%%%%
%%%%%

% Package pour la gestion des couleurs
%
\usepackage{color}

% Les commandes pour passer de la version nb � la version couleur 
% (et r�ciproquement)
%
\newboolean{enNoir}
\setboolean{enNoir}{true}
\newboolean{enCouleur}
\setboolean{enCouleur}{false}
\newcommand{\enNoir}{\definecolor{couleur}{rgb}{0,0,0}
  \setboolean{enCouleur}{false}
  \setboolean{enNoir}{true}}
\newcommand{\enGris}{\definecolor{couleur}{rgb}{0,0,0}
  \setboolean{enCouleur}{false}
  \setboolean{enNoir}{false}}
% D�finition du bleu ULP
\newcommand{\enCouleur}{\definecolor{couleur}{rgb}{0,0.2,0.6}
  \setboolean{enCouleur}{true}
  \setboolean{enNoir}{false}}
% La version par d�faut en en noir
%
\enNoir

%%%%%%
% Gestion de pdf
%%%%%%
%--------------------------------------
%pour compiler avec latex ou pdflatex
\newif\ifpdf
\ifx\pdfoutput\undefined
  \pdffalse
\else
  \pdfoutput=1
  \pdftrue
\fi
%--------------------------------------
%pour eviter de generer des fonts de type 3
\ifpdf 
\usepackage{pslatex}
\usepackage[pdftex]{graphicx}
%\DeclareGraphicsExtensions{.gif}
\else
\usepackage[dvips]{graphicx}
\fi
%--------------------------------------


\addtolength{\parskip}{\baselineskip}

\newlength{\firstcolumnwidth}
\setlength{\firstcolumnwidth}{1.65in}
\newlength{\inbetween}
\setlength{\inbetween}{5mm}
\newlength{\headermargin}
\setlength{\headermargin}{49mm}

\geometry{left=0.47in+\firstcolumnwidth+\inbetween,right=1in,top=15mm,bottom=20mm,noheadfoot}

\newenvironment{letter}[3][]
{
\thispagestyle{empty}

\newsavebox{\dimargin}
\sbox{\dimargin}{%
  \begin{minipage}[t]{\firstcolumnwidth}
    \vspace{0pt}
    \sffamily
    \begin{flushright}
      %%% Ent�te du DI
      \newsavebox{\dihead}
      \sbox{\dihead}{%
        \begin{minipage}{\linewidth}
          \flushright
          {\bfseries
            \color{couleur}
            \begin{spacing}{1}
              \large{\'Equipe EDPTC}
              
              \large{Laboratoire IRMA}

              \normalsize{UMR 7501} 
            
              {\centering \large{\&}}

              \large{\'Equipe ICPS} 

              \large{Laboratoire LSIIT} 
        
              \normalsize{UMR 7005} 
            \end{spacing}
            }
          \vspace{-2mm}
          
          \rule{36.5mm}{0.5pt}
          
%          \vspace{-4mm}
          %%% Inclusion du Logo ULP
          \ifthenelse{\boolean{enNoir}}
          {
            \flushright{\includegraphics[height=19mm]{logo-uds-couleur-800x433}}
          }
          {
            \ifthenelse{\boolean{enCouleur}}
            {
              \includegraphics[height=19mm]{}
              }
            {
              \includegraphics[height=19mm]{logo-uds-couleur-800x433}
              }
            }
          %%% Fin inclusion logo
        \end{minipage}%
        }
      \newlength{\diheadheight}
      \newlength{\diheaddepth}
      \newlength{\diheadspace}
      \settoheight{\diheadheight}{\usebox{\dihead}}
      \settodepth{\diheaddepth}{\usebox{\dihead}}
      \setlength{\diheadspace}{\diheadheight+\diheaddepth}
      %
      %
     %
      %
      %%% Coordonn�es 
      \newsavebox{\difoot}
      \sbox{\difoot}{%
        \begin{minipage}{\linewidth}
          \flushright
          \begin{spacing}{1.25}
            \footnotesize
            
            \textbf{IRMA-EDPTC}
            
            {7, rue Descartes}
            
            {F-67000 Strasbourg}
            
            {T�l. : +33 (0) 3 68 85 01 29}
            
            {Fax : +33 (0) 3 68 85 03 28}
            
            {http\string://www-irma.u-strasbg.fr}

            \vspace{1cm} 
            \textbf{LSIIT-ICPS}
            
            
            {P�le API -Bd S. Brant}
            
            {F-67400 Illkirch}
            
            {T�l. : +33 (0) 3 90 24 45 42}
            
            {Fax : +33 (0) 3 90 24 45 47}
            
            {http\string://icps.u-strasbg.fr}
          \end{spacing}
        \end{minipage}%
        }
      \newlength{\difootheight}
      \newlength{\difootdepth}
      \newlength{\difootspace}
      \settoheight{\difootheight}{\usebox{\difoot}}
      \settodepth{\difootdepth}{\usebox{\difoot}}
      \setlength{\difootspace}{\difootheight+\difootdepth}
      %
      %
      \newlength{\devantsender}
      \setlength{\devantsender}{57mm}
%
      \newlength{\devantdifoot}
%
      \setlength{\devantdifoot}{\textheight-\diheadspace-\devantsender-\difootspace-8mm-13pt}
%
      \vspace{8mm}

      \usebox{\dihead}

      \vspace{\devantsender}

      \vspace{\devantdifoot}

      \usebox{\difoot}
    \end{flushright}
  \end{minipage}%
}

\begin{spacing}{1}

  \noindent
  \begin{picture}(0,0)
    \setlength{\unitlength}{\firstcolumnwidth+\inbetween}
    \put(-1,0){\usebox{\dimargin}}
  \end{picture}%
%
\begin{minipage}[t]{\headermargin}

\vspace{26mm}

~
\end{minipage}%
\begin{minipage}[t]{\linewidth-\headermargin}
  Strasbourg, le \today
\end{minipage}

\noindent
\begin{minipage}[t]{\headermargin}
\vspace{35mm}
~
\end{minipage}%
%-----------------------------------------
% Champ destinataire (#2=To/A , #3 <nom>)   
%-----------------------------------------
\ifthenelse{\equal{#3}{}}{%
\begin{minipage}[t]{\linewidth-\headermargin}

\vspace{2mm}

\end{minipage}
}%
{%
\begin{minipage}[t]{\linewidth-\headermargin} #2
  
  

  \vspace{2mm}

  #3
}
\end{minipage}
\ifthenelse{\equal{#1}{}} {} {Objet : {#1}}
\vspace{20mm}
}
{
%\vspace{20mm}

\newlength{\taillesignature}
\settowidth{\taillesignature}{\fromname}
\newlength{\possignature}
\setlength{\possignature}{0.667\linewidth-0.5\taillesignature}

\hspace{\possignature}\fromname


\end{spacing}
}



\newcommand*{\name}[1]{\def\fromname{#1}}
\name{}
\newcommand*{\fonction}[1]{\def\fonction{#1}}
\newcommand*{\telephoneesplanade}[1]{\def\phoneespla{#1}}
\telephoneesplanade{}
\newcommand*{\telephoneillkirch}[1]{\def\phoneillkirch{#1}}
\telephoneillkirch{}
\newcommand*{\email}[1]{\def\emailadd{#1}}
\email{}
\newcommand*{\objet}[1]{\def\emailadd{#1}}





\selectlanguage{francais}
\begin{document}


\name{Philippe \textsc{Clauss} et Philippe \textsc{Helluy}}

\begin{letter}[Recommandation de Mathieu Faverge (poste CR2 n°07/06)]{à}
		  {Bruno Durand, \\Président de la section 7 du CoCNRS}

En tant que représentants des équipes de mathématiques (EDPTC, 
laboratoire IRMA UMR CNRS 7501) et informatique (ICPS, laboratoire 
LSIIT UMR CNRS 7005), nous souhaitons apporter notre soutien le plus
fort à la candidature de Mathieu Faverge au poste de chargé de recherche 
n°07/06.   

Les compétences de M. Faverge sont à l'interface entre mathématiques
appliquées et informatique. Ces compétences sont rares et ont été 
acquises dans un laboratoire de pointe au niveau mondial (équipe de 
Jack Dongarra). Par ailleurs, ses expériences avec le calcul intensif 
appliquées aux méthodes numériques sont nombreuses et variées. 

Il a, dès son stage de fin d'études d'ingénieur au CEA, travaillé à la 
parallélisation de codes pour la physique utilisant les mêmes méthodes 
numériques que celles que nous étudions pour modéliser la physique des 
plasmas. \`A travers son travail de thèse, qui étudie des moyens 
d'ordonnancer des codes d'algèbre linéaire creuse sur des supercalculateurs
modernes, il a acquis des connaissances indispensables à la compréhension
des effets de l'architecture des machines, aujourd'hui très hiérarchique,
sur le choix ou la conception des schémas numériques. M. Faverge est
actuellement en Post-Doc dans un des laboratoires les plus réputés
dans le domaine du calcul intensif, le laboratoire ICL de l'Université 
du Tennessee, USA, dirigé par Jack Dongarra. Son expérience sur le développement
de la bibliothèque PLASMA, noyaux de calcul d'algèbre linéaire dense
sur GPU, est un atout pour appuyer nos efforts dans l'expérimentation
de codes pour GPUs. Enfin, les techniques d'ordonnancement sur lesquelles
il travaille, dans la lignée des travaux menés dans l'équipe-projet
RunTime à Bordeaux, qui visent à trouver les meilleures configurations
matérielles (type de processeur) dynamiquement lors de l'exécution,
sont très proches des travaux développés dans notre équipe-projet 
INRIA Camus (sous-ensemble de l'ICPS).


Nos échanges avec M. Faverge nous ont convaincu que ces compétences et 
son projet de recherche constitueraient des atouts majeurs relativement 
au projet que nous comptons développer. Nous démarrons le projet de
Labex IRMIA dans lesquel les interactions entre calcul parallèle et 
methodes numériques doivent être très fortes.  M. Faverge possède donc 
des compétences entièrement convergentes avec cet objectif. Elles  % mettre 2 ou 3 exemples
permettent d'envisager de nombreuses perspectives de recherche qui 
renforceraient grandement la collaboration menée entre nos deux équipes 
depuis 2002.


\end{letter}
\begin{flushright}
\includegraphics[width=.30\textwidth]{sign_clauss.jpg}
\includegraphics[width=.30\textwidth]{sign_helluy.jpg} 
\end{flushright}
%\newpage
%\bibliographystyle{plain}
%\bibliography{julien_blaise_ED}


\end{document}
