\documentclass[a4paper,12pt]{article}

\usepackage[utf8]{inputenc}
\usepackage[T1]{fontenc}
\usepackage[francais]{babel}
\usepackage{mathptmx}
\usepackage[scaled]{helvet}
\usepackage{courier}

\usepackage[margin=2.2cm,nohead,foot=1cm]{geometry}
\usepackage[compact,sf,small]{titlesec}
\usepackage{parskip}
\usepackage{comment}
\usepackage{graphicx}
\usepackage{fancyvrb}

\begin{document}

{\large\sffamily
\begin{tabular}[t]{l}
  ENSIIE 1\textsuperscript{e} année\\
  Systèmes
\end{tabular}%
\hfill%
\begin{tabular}[t]{r}
  Année 2009-2010
  %\\ A. Ketterlin
\end{tabular}}

\vspace{2em}
\begin{center}
  \Large\bfseries\sffamily TP \no12
\end{center}
\vspace{1em}


\section{\texttt{dlopen()}}

\textbf{Question 1} : Il s'agit de compléter main1.c en remplaçant les zones en pointillés 
pour qu'il fasse effectivement le travail demandé.

\textbf{Explications} : La fonction \texttt{main()} permet au choix de compresser ou de
décompresser un fichier (de taille inférieure à 4096 octets).
Elle doit être appelée avec trois arguments : le premier est soit compress
soit uncompress (nom de l'opération à exécuter), le second est le nom du fichier
source sur lequel l'opération doit porter et le troisième est le nom du fichier
dans lequel le résultat de l'opération doit être placé.
Pour procéder à l'opération demandée, la fonction main doit ajouter à son
image mémoire la fonction demandée (\texttt{compress()} ou \texttt{uncompress()}) qui ne
figure pas dans le programme au début de l'exécution, mais qui doit se trouver
dans le fichier compress.so ou dans le fichier uncompress.so.
Cette opération peut être effectuée à l'aide de la fonction \texttt{dlopen()} qui
ajoute dans l'image mémoire d'un processus, pendant son exécution, le contenu
d'un fichier partageable d'extension .so. Il faut donc, avant d'exécuter le pro-
gramme, produire le fichier à inclure (c'est-à-dire compress.so ou uncompress.so),
ce qui peut être fait par la commande suivante (à partir d'un fichier fich.o) :
\begin{verbatim}
gcc -shared -o fich.so fich.c.
\end{verbatim}
Il faut ensuite trouver l'adresse virtuelle d'implantation de la fonction à
utiliser grâce à la fonction \texttt{dlsym()}.
Nota : il est totalement superflu de comprendre comment fonctionnent exac-
tement les fonctions compress et uncompress. On peut se limiter aux interfaces
et à l'entête compress.h.
Pour que l'édition de liens dynamique puisse être opérationnelle, il faut que
la variable d'environnement \verb#LD_LIBRARY_PATH# contienne les chemins d'accès
au répertoire contenant les fichiers .so. La compilation de \texttt{main1.c} doit être
effectuée avec l'option -ldl.

\begin{comment}
Correction :
Les lignes complétées sont (dans l'ordre) :
operation(buffer_in, length_in, buffer_out, &length_out);
write(out, buffer_out, sizeof(UBYTE)*length_out);
sprintf(partageable, "%s.so", function);
if ((so_handle = dlopen(partageable,RTLD_LAZY)) == NULL)
dlsym(so_handle, function);
operation = loader(argv[1]);
treatment(argv[2], argv[3]);
dlopen() permet d'ouvrir la bibliothèque dynamique. RTLD_LAZY indique que les
symboles doivent être résolus quand c'est nécessaire (et non pas immédiatement).
dlsym() cherche un symbole dans une bibliothèque dynamique ouverte. 
\end{comment}
\end{document}
